% ===================================
% CODICE LaTeX PER GRAFICI E TABELLE
% Tesi GDO - Capitoli 1 e 2
% ===================================
% filepath: c:\Users\saint\tesi\nuovi grafi 1 e 2.tex
\documentclass[border=10pt]{standalone}
% Pacchetti necessari
\usepackage[utf8]{inputenc}
\usepackage[T1]{fontenc}
\usepackage{amsmath}
\usepackage{amssymb}    
\usepackage{graphicx}
\usepackage{caption}
\usepackage{subcaption} % Per sottotitoli nelle figure
% Preambolo necessario:
\usepackage{tikz}
\usepackage{pgfplots}
\usepackage{booktabs}
\usepackage{multirow}
\usetikzlibrary{shadows}
\pgfplotsset{compat=1.17}
\usetikzlibrary{pgfplots.polar}

% FIGURA 1.4 - VERSIONE 1: Schema a flusso verticale con gradiente blu
\begin{document}


\centering
\begin{tikzpicture}[
    node distance=2.8cm,
    chapter/.style={rectangle, draw=blue!70, fill=blue!10, text width=5.5cm, 
                text centered, rounded corners=8pt, minimum height=2cm, 
                font=\normalsize, line width=1.2pt, drop shadow},
    framework/.style={rectangle, draw=blue!90, fill=blue!5, text width=4.5cm, 
                text centered, rounded corners=8pt, minimum height=8cm,
                font=\normalsize\bfseries, line width=2pt, drop shadow},
    arrow/.style={->, >=latex, line width=1.5pt, draw=blue!60},
    feedback/.style={->, >=latex, line width=1pt, draw=orange!70, dashed},
    label/.style={font=\footnotesize, text=black!80, fill=white, inner sep=3pt, rounded corners=2pt}
]

% Gradiente di intensità per i capitoli
\node[chapter, fill=blue!5] (cap1) {
    \textbf{Capitolo 1: Introduzione}\\[3pt]
    \footnotesize Contesto, problematiche e obiettivi\\
    Framework metodologico
};

\node[chapter, fill=blue!15, below of=cap1] (cap2) {
    \textbf{Capitolo 2: Threat Landscape}\\[3pt]
    \footnotesize Analisi quantitativa minacce\\
    Metriche di vulnerabilità GDO
};

\node[chapter, fill=blue!25, below of=cap2] (cap3) {
    \textbf{Capitolo 3: Infrastruttura}\\[3pt]
    \footnotesize Evoluzione fisica → cloud\\
    Pattern architetturali
};

\node[chapter, fill=blue!35, below of=cap3] (cap4) {
    \textbf{Capitolo 4: Compliance}\\[3pt]
    \footnotesize Integrazione PCI-DSS/GDPR/NIS2\\
    Modello economico
};

\node[chapter, fill=blue!45, below of=cap4] (cap5) {
    \textbf{Capitolo 5: Framework GIST}\\[3pt]
    \footnotesize Sintesi e validazione\\
    Roadmap strategica
};

% Framework GIST con icone
\node[framework, right=5cm of cap3.east] (gist) {
    \textbf{GIST Framework}\\[0.5cm]
    \begin{tikzpicture}[scale=0.8]
        \node[circle, fill=orange!30, draw=orange!70, minimum size=1cm] (P) at (0,0) {P};
        \node[circle, fill=green!30, draw=green!70, minimum size=1cm] (A) at (2,0) {A};
        \node[circle, fill=red!30, draw=red!70, minimum size=1cm] (S) at (0,-2) {S};
        \node[circle, fill=purple!30, draw=purple!70, minimum size=1cm] (C) at (2,-2) {C};
        \draw[-, line width=0.5pt, gray!50] (P) -- (A) -- (C) -- (S) -- (P);
        \draw[-, line width=0.5pt, gray!50] (P) -- (C);
        \draw[-, line width=0.5pt, gray!50] (A) -- (S);
    \end{tikzpicture}\\[0.3cm]
    \footnotesize
    \textcolor{orange!70}{P}: Physical\\
    \textcolor{green!70}{A}: Architectural\\
    \textcolor{red!70}{S}: Security\\
    \textcolor{purple!70}{C}: Compliance
};

% Frecce principali
\foreach \i in {1,...,4} {
    \draw[arrow] (cap\i) -- (cap\the\numexpr\i+1\relax);
}

% Feedback al framework
\draw[feedback] (cap2.east) -| node[label, pos=0.3] {Requisiti} (gist.160);
\draw[feedback] (cap3.east) -- node[label, above] {Pattern} (gist.west);
\draw[feedback] (cap4.east) -| node[label, pos=0.3] {Vincoli} (gist.200);
\draw[feedback, line width=1.5pt] (gist.south) |- node[label, pos=0.7] {Validazione} (cap5.east);

\end{tikzpicture}
%\caption{Struttura della tesi con approccio incrementale. L'intensità cromatica crescente riflette l'approfondimento progressivo verso il Framework GIST finale.}
%\label{fig:struttura_v1}
\end{document}
