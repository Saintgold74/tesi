\documentclass{beamer}
\usetheme{Madrid}
\usepackage[utf8]{inputenc}
\usepackage[italian]{babel}
\usepackage{amsmath}
\usepackage{graphicx}

\title[GIST nella GDO]{GDO Integrated Security Transformation (GIST)}
\subtitle{Un framework per infrastrutture IT sicure, resilienti e compliant}
\author{Marco [Cognome]}
\institute{Corso di Laurea in Ingegneria Informatica}
\date{Anno Accademico 2024/2025}

\begin{document}

\begin{frame}
  \titlepage
\end{frame}

% Slide 2
\begin{frame}{Obiettivo della Ricerca}
\textbf{Problema centrale:}
\begin{itemize}
  \item Come bilanciare sicurezza, performance, compliance e sostenibilità economica nell'IT GDO?
\end{itemize}

\textbf{Soluzione proposta:}
\begin{itemize}
  \item Framework integrato GIST
\end{itemize}
\end{frame}

% Slide 3
\begin{frame}{Contesto e Complessità}
\textbf{La GDO italiana:}
\begin{itemize}
  \item 27.000+ punti vendita
  \item 45M transazioni/giorno
  \item SLA > 99.9\%
\end{itemize}

\textbf{Vincoli:}
\begin{itemize}
  \item Ambienti distribuiti
  \item Sistemi legacy + cloud-native
  \item Normative sovrapposte (GDPR, PCI-DSS, NIS2)
\end{itemize}
\end{frame}

% Slide 4
\begin{frame}{Ipotesi di Ricerca}
\begin{itemize}
  \item \textbf{H1}: Architettura cloud-ibrida \(\Rightarrow\) SLA ≥ 99.95\%, TCO -30\%
  \item \textbf{H2}: Zero Trust \(\Rightarrow\) -35\% superficie di attacco, latenza < 50ms
  \item \textbf{H3}: Compliance integrata \(\Rightarrow\) -30/40\% costi, overhead IT < 10\%
\end{itemize}
\end{frame}

% Slide 5
\begin{frame}{Metodologia}
\textbf{Approccio mixed-methods:}
\begin{itemize}
  \item Studio longitudinale (15 GDO, 24 mesi)
  \item Quantitativa: regressione, simulazioni Monte Carlo
  \item Qualitativa: casi studio, digital twin
\end{itemize}

\textbf{Framework GIST:}  
\[
\text{GIST} = f(\text{Physical, Architectural, Security, Compliance}) \times \text{Context}_{\text{GDO}}
\]
\end{frame}

% Slide 6
\begin{frame}{Risultati H1 – Cloud Ibrido}
\begin{itemize}
  \item SLA ≥ 99.95\% (87\% dei casi)
  \item TCO medio: -33\%
  \item ROI positivo entro 3 anni
\end{itemize}

\textbf{→ Alta disponibilità e flessibilità operativa}
\end{frame}

% Slide 7
\begin{frame}{Risultati H2 – Zero Trust}
\begin{itemize}
  \item Riduzione ASSA: -41.2\%
  \item Latenza aggiuntiva: +19ms (POS-friendly)
  \item Tecniche: micro-segmentazione, verifica continua
\end{itemize}

\textbf{→ Sicurezza senza compromettere l’usabilità}
\end{frame}

% Slide 8
\begin{frame}{Risultati H3 – Compliance Integrata}
\begin{itemize}
  \item Costi di compliance: -37\%
  \item Overhead IT: < 9.5\%
  \item Matrice normativa: GDPR + PCI-DSS + NIS2
\end{itemize}
\end{frame}

% Slide 9
\begin{frame}{Contributi Originali}
\begin{enumerate}
  \item Framework GIST multilivello
  \item Modello GDO-Cloud per TCO/ROI
  \item Matrice integrata per compliance
  \item Dataset empirico (15 aziende, 24 mesi)
\end{enumerate}
\end{frame}

% Slide 10
\begin{frame}{Impatto e Applicazioni}
\textbf{Applicazioni pratiche:}
\begin{itemize}
  \item Supporto alle decisioni IT
  \item Roadmap in 3 fasi
  \item Riduzione rischio cyber, migliore protezione dati
\end{itemize}

\textbf{Impatto sociale:}
\begin{itemize}
  \item Resilienza dei servizi essenziali
  \item Ottimizzazione energetica
\end{itemize}
\end{frame}

% Slide 11
\begin{frame}{Limiti e Prospettive Future}
\textbf{Limiti:}
\begin{itemize}
  \item Contesto geografico: Italia/UE
  \item Orizzonte: 24 mesi
  \item Alcuni dati aggregati
\end{itemize}

\textbf{Prossimi sviluppi:}
\begin{itemize}
  \item Integrazione AI/ML nei SIEM
  \item Applicazione a sanità/logistica
  \item DevSecOps distribuito
\end{itemize}
\end{frame}

% Slide 12
\begin{frame}{Conclusione}
\textbf{In sintesi:}
\begin{quote}
  Il framework GIST consente una trasformazione digitale sicura, performante ed economicamente sostenibile per la GDO italiana.
\end{quote}

\textbf{Grazie per l’attenzione!}  
(Ringraziamenti a relatori, azienda partner, team)
\end{frame}

\end{document}
