% =====================================
% GRAFICI E DIAGRAMMI - CAPITOLO 1 INTRODUZIONE
% "Dall'Alimentazione alla Cybersecurity: Fondamenti di un'Infrastruttura IT Sicura nel Contesto della GDO"
% Compatibile con XeLaTeX
% =====================================

\documentclass[12pt,a4paper]{article}
\usepackage{fontspec}
\usepackage{polyglossia}
\setmainlanguage{italian}
\usepackage{amsmath}
\usepackage{amsfonts}
\usepackage{amssymb}
\usepackage{tikz}
\usepackage{pgfplots}
\usepackage{pgfplotstable}
\usepackage{booktabs}
\usepackage{array}
\usepackage{multirow}
\usepackage{longtable}
\usepackage{xcolor}
\usepackage{geometry}

% Configurazioni PGFPlots
\pgfplotsset{compat=1.18}
\usetikzlibrary{positioning,shapes,arrows,calc,patterns,decorations.pathreplacing,fit,backgrounds}

% Definizione colori tema GDO
\definecolor{gdoblu}{RGB}{25,118,188}
\definecolor{gdorosso}{RGB}{220,38,58}
\definecolor{gdoverde}{RGB}{0,150,108}
\definecolor{gdoarancio}{RGB}{255,165,0}
\definecolor{gdogrigio}{RGB}{128,128,128}
\definecolor{gdovioletto}{RGB}{128,0,128}

\geometry{margin=2cm}

\begin{document}

% =====================================
% FIGURA 1.1: EVOLUZIONE ARCHITETTURE GDO
% =====================================

\begin{figure}[htbp]
\centering
\begin{tikzpicture}[node distance=3cm]
% Definizione stili
\tikzset{
    fase/.style={rectangle, rounded corners, minimum width=4cm, minimum height=3cm, text centered, draw=black, thick},
    timeline/.style={->, thick, gdogrigio},
    anno/.style={rectangle, fill=white, draw=gdogrigio, text centered}
}

% Timeline orizzontale
\draw[timeline] (0,0) -- (15,0);

% Fasi evolutive
\node[fase, fill=gdorosso!20] (fase1) at (2.5,0) {
\begin{minipage}{3.5cm}
\centering
\textbf{Fase I}
\textbf{Architetture Centralizzate}

• Mainframe centrali
• Terminali stupidi
• Connettività dedicata
• Single point of failure
\end{minipage}
};

\node[fase, fill=gdoarancio!20] (fase2) at (7.5,0) {
\begin{minipage}{3.5cm}
\centering
\textbf{Fase II}
\textbf{Sistemi Distribuiti}

• Server locali
• Virtualizzazione
• Replica dati
• SD-WAN emergence
\end{minipage}
};

\node[fase, fill=gdoverde!20] (fase3) at (12.5,0) {
\begin{minipage}{3.5cm}
\centering
\textbf{Fase III}
\textbf{Cloud Ibrido}

• Cloud-first strategy
• Edge computing
• API-driven architecture
• Zero Trust security
\end{minipage}
};

% Anni sulla timeline
\node[anno] at (2.5,-2) {1990-2010};
\node[anno] at (7.5,-2) {2010-2020};
\node[anno] at (12.5,-2) {2020-Presente};

% Caratteristiche evolutive sopra
\node[rectangle, fill=gdoblu!10, draw=gdoblu, minimum width=2cm] at (2.5,4) {\footnotesize Controllo};
\node[rectangle, fill=gdoblu!10, draw=gdoblu, minimum width=2cm] at (7.5,4) {\footnotesize Flessibilità};
\node[rectangle, fill=gdoblu!10, draw=gdoblu, minimum width=2cm] at (12.5,4) {\footnotesize Elasticità};

% Frecce di transizione
\draw[->, thick, gdogrigio] (fase1.east) -- (fase2.west) node[midway, above] {\footnotesize Distribuzione};
\draw[->, thick, gdogrigio] (fase2.east) -- (fase3.west) node[midway, above] {\footnotesize Ibridazione};

\end{tikzpicture}
\caption{Evoluzione delle architetture IT nella GDO: dalla centralizzazione al cloud ibrido}
\label{fig:evoluzione_architetture}
\end{figure}

% =====================================
% FIGURA 1.2: FRAMEWORK VALUTAZIONE MULTI-CRITERIO
% =====================================

\begin{figure}[htbp]
\centering
\begin{tikzpicture}[scale=0.8]
% Pentagono per i 5 criteri
\def\radius{4}
\def\angles{90, 162, 234, 306, 18} % Angoli per i 5 criteri

% Disegno del pentagono di base
\draw[thick, gdogrigio] 
  (90:\radius) -- (162:\radius) -- (234:\radius) -- (306:\radius) -- (18:\radius) -- cycle;

% Griglia interna (3 livelli)
\foreach \r in {1.33, 2.66} {
  \draw[gdogrigio!50, thin] 
    (90:\r) -- (162:\r) -- (234:\r) -- (306:\r) -- (18:\r) -- cycle;
}

% Assi per ogni criterio
\foreach \angle in \angles {
  \draw[gdogrigio!70, thin] (0,0) -- (\angle:\radius);
}

% Etichette criteri
\node at (90:\radius+0.7) {\textbf{Sicurezza (S)}};
\node at (162:\radius+0.7) {\textbf{Scalabilità (Sc)}};
\node at (234:\radius+0.7) {\textbf{Compliance (C)}};
\node at (306:\radius+0.7) {\textbf{TCO}};
\node at (18:\radius+0.7) {\textbf{Resilienza (R)}};

% Esempio di valutazione architettura (area colorata)
\filldraw[fill=gdoblu!30, draw=gdoblu, thick] 
  (90:3.2) -- (162:2.8) -- (234:3.6) -- (306:2.4) -- (18:3.0) -- cycle;

% Valori numerici sui punti
\node[circle, fill=gdoblu, text=white, inner sep=2pt] at (90:3.2) {\footnotesize 8};
\node[circle, fill=gdoblu, text=white, inner sep=2pt] at (162:2.8) {\footnotesize 7};
\node[circle, fill=gdoblu, text=white, inner sep=2pt] at (234:3.6) {\footnotesize 9};
\node[circle, fill=gdoblu, text=white, inner sep=2pt] at (306:2.4) {\footnotesize 6};
\node[circle, fill=gdoblu, text=white, inner sep=2pt] at (18:3.0) {\footnotesize 7.5};

% Scala di valutazione
\node at (-6,3) {\begin{minipage}{3cm}
\textbf{Scala di Valutazione:}
\begin{itemize}
\item[\footnotesize 10] Eccellente
\item[\footnotesize 7-9] Buono  
\item[\footnotesize 4-6] Accettabile
\item[\footnotesize 1-3] Critico
\end{itemize}
\end{minipage}};

% Formula di utilità
\node at (-6,-2) {\begin{minipage}{4cm}
\footnotesize
\textbf{Funzione di Utilità:}

$U = w_1 \cdot S + w_2 \cdot Sc + w_3 \cdot C + w_4 \cdot TCO^{-1} + w_5 \cdot R$

Dove $\sum w_i = 1$
\end{minipage}};

\end{tikzpicture}
\caption{Framework di valutazione multi-criterio per architetture GDO}
\label{fig:framework_valutazione}
\end{figure}

% =====================================
% FIGURA 1.3: METODOLOGIA DI RICERCA
% =====================================

\begin{figure}[htbp]
\centering
\begin{tikzpicture}[node distance=2cm, auto]
% Definizione stili
\tikzset{
    process/.style={rectangle, rounded corners, minimum width=3cm, minimum height=1.5cm, text centered, draw=gdoblu, fill=gdoblu!20, thick},
    analysis/.style={diamond, minimum width=2.5cm, minimum height=1.5cm, text centered, draw=gdoverde, fill=gdoverde!20, thick},
    output/.style={rectangle, rounded corners, minimum width=2.5cm, minimum height=1cm, text centered, draw=gdorosso, fill=gdorosso!20, thick},
    data/.style={cylinder, shape border rotate=90, minimum width=2cm, minimum height=1.2cm, text centered, draw=gdoarancio, fill=gdoarancio!20, thick}
}

% Flusso metodologico principale
\node[process] (literature) at (0,8) {Revisione Letteratura};
\node[data] (sources) at (4,8) {
\begin{minipage}{2cm}
\centering
\footnotesize
IEEE/ACM
Gartner
Forrester
\end{minipage}
};

\node[process] (empirical) at (0,5.5) {Analisi Empirica};
\node[data] (datasets) at (4,5.5) {
\begin{minipage}{2cm}
\centering
\footnotesize
Dataset Pubblici
Benchmark
Case Study
\end{minipage}
};

\node[analysis] (quantitative) at (-3,3) {Analisi Quantitativa};
\node[analysis] (qualitative) at (3,3) {Analisi Qualitativa};

\node[process] (modeling) at (0,0.5) {Modellazione Matematica};

\node[output] (framework) at (-3,-2) {Framework Integrato};
\node[output] (principles) at (0,-2) {Design Principles};
\node[output] (roadmap) at (3,-2) {Strategic Roadmap};

% Connessioni principali
\draw[->] (literature) -- (empirical);
\draw[->] (sources) -- (literature);
\draw[->] (datasets) -- (empirical);

\draw[->] (empirical) -- (quantitative);
\draw[->] (empirical) -- (qualitative);

\draw[->] (quantitative) -- (modeling);
\draw[->] (qualitative) -- (modeling);

\draw[->] (modeling) -- (framework);
\draw[->] (modeling) -- (principles);
\draw[->] (modeling) -- (roadmap);

% Feedback loops
\draw[->, dashed, gdogrigio] (framework) to[bend right=20] (quantitative);
\draw[->, dashed, gdogrigio] (principles) to[bend left=20] (qualitative);

% Tecniche utilizzate (annotazioni)
\node[rectangle, fill=yellow!20, draw=none] at (6,3) {
\begin{minipage}{3cm}
\footnotesize
\textbf{Tecniche MCDM:}
• Fuzzy Logic
• Rough Set Theory
• AHP Analysis
• TOPSIS
\end{minipage}
};

\end{tikzpicture}
\caption{Metodologia di ricerca integrata: approccio multi-fase per l'analisi delle architetture GDO}
\label{fig:metodologia_ricerca}
\end{figure}

% =====================================
% FIGURA 1.4: ROADMAP DAL FISICO AL DIGITALE
% =====================================

\begin{figure}[htbp]
\centering
\begin{tikzpicture}[node distance=1.5cm]
% Definizione stili
\tikzset{
    layer/.style={rectangle, rounded corners, minimum width=13cm, minimum height=2cm, text centered, draw=black, thick},
    arrow/.style={->, thick, gdogrigio},
    chapter/.style={rectangle, rounded corners, minimum width=2.5cm, minimum height=1cm, text centered, draw=gdoblu, fill=gdoblu!20}
}

% Livelli dell'architettura (dal fisico al digitale)
\node[layer, fill=gdogrigio!20] (fisico) at (0,0) {
\textbf{Livello Fisico} - Alimentazione, Cooling, Connettività, Vincoli Ambientali
};

\node[layer, fill=gdoarancio!20] (infra) at (0,3) {
\textbf{Livello Infrastrutturale} - Data Center, Edge Computing, SD-WAN, Networking
};

\node[layer, fill=gdoverde!20] (platform) at (0,6) {
\textbf{Livello Piattaforma} - Virtualizzazione, Container, Cloud Services, Orchestrazione
};

\node[layer, fill=gdoblu!20] (application) at (0,9) {
\textbf{Livello Applicativo} - POS Systems, ERP, Analytics, Customer Experience
};

\node[layer, fill=gdovioletto!20] (security) at (0,12) {
\textbf{Livello Sicurezza} - Zero Trust, Compliance, Governance, Incident Response
};

% Frecce di trasformazione
\draw[arrow] (fisico.north) -- (infra.south);
\draw[arrow] (infra.north) -- (platform.south);
\draw[arrow] (platform.north) -- (application.south);
\draw[arrow] (application.north) -- (security.south);

% Mapping capitoli tesi
\node[chapter] (cap2) at (8,3) {Cap. 2 Threat Landscape};
\node[chapter] (cap3) at (8,6) {Cap. 3 Evoluzione Infrastrutturale};
\node[chapter] (cap4) at (8,9) {Cap. 4 Compliance};
\node[chapter] (cap5) at (8,12) {Cap. 5 Strategia};

% Connessioni capitoli-livelli
\draw[dashed, gdorosso] (cap2) -- (infra.east);
\draw[dashed, gdorosso] (cap3) -- (platform.east);
\draw[dashed, gdorosso] (cap4) -- (application.east);
\draw[dashed, gdorosso] (cap5) -- (security.east);

% Innovazioni tecnologiche (lato sinistro)
\node at (-8,6) {
\begin{minipage}{3cm}
\footnotesize
\textbf{Tecnologie Abilitanti:}
• AI/ML
• IoT
• Blockchain
• 5G/6G
• Quantum-Ready
• Sustainable IT
\end{minipage}
};

% Drivers di cambiamento (lato destro)
\node at (15,6) {
\begin{minipage}{3cm}
\footnotesize
\textbf{Drivers di Evoluzione:}
• Customer Experience
• Operational Efficiency  
• Regulatory Compliance
• Cyber Resilience
• Cost Optimization
• Innovation Speed
\end{minipage}
};

\end{tikzpicture}
\caption{Roadmap "Dal Fisico al Digitale": evoluzione stratificata dell'architettura GDO}
\label{fig:roadmap_fisico_digitale}
\end{figure}

% =====================================
% TABELLA 1.1: SFIDE SISTEMICHE GDO
% =====================================

\begin{table}[htbp]
\centering
\caption{Sfide Sistemiche della GDO Moderna: Impatti e Requisiti Architetturali}
\label{tab:sfide_sistemiche_gdo}
\begin{tabular}{@{}llcc@{}}
\toprule
\textbf{Categoria Sfida} & \textbf{Descrizione} & \textbf{Impatto Economico} & \textbf{Requisito SLA} \\
\midrule
\multirow{2}{*}{\begin{minipage}{3cm}Operatività Continua\end{minipage}} 
& Downtime sistemi critici & 100k-500k €/ora & Disponibilità > 99.9\% \\
& Interruzione pagamenti & 50k-200k €/ora & Latenza < 200ms \\
\addlinespace
\multirow{2}{*}{\begin{minipage}{3cm}Distribuzione Geografica\end{minipage}} 
& Coordinamento multi-sito & 20-30\% OPEX IT & Sincronizzazione < 1s \\
& Gestione configurazioni & 15-25\% effort IT & Standardizzazione 95\% \\
\addlinespace
\multirow{2}{*}{\begin{minipage}{3cm}Scalabilità Transazionale\end{minipage}} 
& Picchi stagionali & 300-500\% carico base & Scaling automatico \\
& Black Friday events & 1000\% throughput & Elasticità < 60s \\
\addlinespace
\multirow{2}{*}{\begin{minipage}{3cm}Dati Sensibili\end{minipage}} 
& Violazioni GDPR & 4\% fatturato annuo & Encryption 100\% \\
& Breach PCI-DSS & 50k-500k \$ multa & Audit readiness \\
\bottomrule
\end{tabular}
\end{table}

% =====================================
% FIGURA 1.5: ARCHITETTURA TIPICA GDO
% =====================================

\begin{figure}[htbp]
\centering
\begin{tikzpicture}[node distance=2cm, scale=0.9, transform shape]
% Definizione stili
\tikzset{
    datacenter/.style={rectangle, rounded corners, minimum width=4cm, minimum height=3cm, text centered, draw=gdoblu, fill=gdoblu!20, thick},
    store/.style={rectangle, rounded corners, minimum width=2.5cm, minimum height=2cm, text centered, draw=gdoverde, fill=gdoverde!20, thick},
    cloud/.style={ellipse, minimum width=3cm, minimum height=2cm, text centered, draw=gdoarancio, fill=gdoarancio!20, thick},
    connection/.style={->, thick, gdogrigio},
    secure/.style={->, thick, gdorosso, dashed}
}

% Data Center Centrale
\node[datacenter] (hq) at (0,0) {
\begin{minipage}{3.5cm}
\centering
\textbf{HQ Data Center}

• ERP Systems
• Data Warehouse
• Security Operations
• Backup \& DR
\end{minipage}
};

% Punti Vendita
\node[store] (store1) at (-6,-4) {
\begin{minipage}{2.2cm}
\centering
\textbf{Store A}

• POS Systems
• Local Server
• IoT Sensors
• Edge Computing
\end{minipage}
};

\node[store] (store2) at (-3,-4) {
\begin{minipage}{2.2cm}
\centering
\textbf{Store B}

• POS Systems
• Local Server
• IoT Sensors
• Edge Computing
\end{minipage}
};

\node[store] (store3) at (3,-4) {
\begin{minipage}{2.2cm}
\centering
\textbf{Store N}

• POS Systems
• Local Server
• IoT Sensors
• Edge Computing
\end{minipage}
};

% Cloud Services
\node[cloud] (cloud1) at (-4,4) {
\begin{minipage}{2.5cm}
\centering
\textbf{Public Cloud}

• Analytics
• CRM
• E-commerce
\end{minipage}
};

\node[cloud] (cloud2) at (4,4) {
\begin{minipage}{2.5cm}
\centering
\textbf{Hybrid Cloud}

• DR Site
• Development
• AI/ML Services
\end{minipage}
};

% Connessioni
\draw[connection] (hq) -- (store1) node[midway, left] {\footnotesize SD-WAN};
\draw[connection] (hq) -- (store2);
\draw[connection] (hq) -- (store3) node[midway, right] {\footnotesize SD-WAN};

\draw[secure] (hq) -- (cloud1) node[midway, above left] {\footnotesize VPN};
\draw[secure] (hq) -- (cloud2) node[midway, above right] {\footnotesize VPN};

% Indicazione "..." per altri store
\node at (0,-4) {\Large ...};

% Payment Processors (esterni)
\node[rectangle, draw=gdorosso, fill=gdorosso!20, minimum width=2cm] (payment) at (8,0) {
\begin{minipage}{1.8cm}
\centering
\footnotesize
\textbf{Payment Processors}
\end{minipage}
};

\draw[secure] (hq) -- (payment) node[midway, above] {\footnotesize Secure Gateway};

% Internet/External
\node[ellipse, draw=gdogrigio, minimum width=2cm] (internet) at (0,6) {\textbf{Internet}};
\draw[connection] (cloud1) -- (internet);
\draw[connection] (cloud2) -- (internet);

% Legenda sicurezza
\node at (-8,2) {
\begin{minipage}{2.5cm}
\footnotesize
\textbf{Legenda:}
\tikz \draw[connection] (0,0) -- (0.5,0); Connessioni Standard
\tikz \draw[secure] (0,0) -- (0.5,0); Connessioni Sicure
\end{minipage}
};

\end{tikzpicture}
\caption{Architettura tipica GDO: infrastruttura distribuita con componenti on-premise, edge e cloud}
\label{fig:architettura_tipica_gdo}
\end{figure}

% =====================================
% TABELLA 1.2: CONTRIBUTI ORIGINALI ATTESI
% =====================================

\begin{table}[htbp]
\centering
\caption{Contributi Originali della Ricerca: Metodologici, Analitici e Strategici}
\label{tab:contributi_originali}
\begin{tabular}{@{}llcp{5cm}@{}}
\toprule
\textbf{Tipo Contributo} & \textbf{Ambito} & \textbf{Novità} & \textbf{Impatto Atteso} \\
\midrule
\multirow{2}{*}{Metodologico} 
& Framework MCDM & Alto & Framework quantitativo per valutazione architetture GDO \\
& Metriche integrate & Medio & Standardizzazione criteri valutazione settore retail \\
\addlinespace
\multirow{2}{*}{Analitico} 
& Threat modeling & Alto & Modelli predittivi rischio per architetture cloud-ibride \\
& Pattern analysis & Medio & Identificazione vulnerabilità emergenti IT-OT \\
\addlinespace
\multirow{2}{*}{Progettuale} 
& Design principles & Alto & Principi architetturali sicurezza-per-progettazione \\
& Reference architecture & Medio & Template implementativo per transizione cloud \\
\addlinespace
\multirow{2}{*}{Strategico} 
& Strategic roadmap & Alto & Guida pianificazione trasformazione digitale GDO \\
& ROI models & Medio & Modelli economici investimenti sicurezza \\
\bottomrule
\end{tabular}
\end{table}

\end{document}