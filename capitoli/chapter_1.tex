% Capitolo 1 - Introduzione
\chapter{Introduzione}
\label{ch:introduzione}

\section{Contesto di Riferimento: La Trasformazione Digitale della Grande Distribuzione Organizzata}
\label{sec:contesto-gdo}

La Grande Distribuzione Organizzata rappresenta uno dei settori più complessi dal punto di vista dell'ingegneria dei sistemi informatici, configurandosi come un ecosistema tecnologico che deve conciliare esigenze operative estremamente eterogenee con vincoli di sicurezza, performance e compliance sempre più stringenti. L'evoluzione di questo settore negli ultimi due decenni evidenzia una trasformazione paradigmatica che trascende la semplice digitalizzazione dei processi, configurandosi come una ridefinizione fondamentale dell'architettura informativa aziendale.

Dal punto di vista sistemico, la \gls{gdo} presenta caratteristiche che la rendono un laboratorio naturale per l'analisi delle sfide dell'informatica moderna. L'operatività continua H24/365, la distribuzione geografica delle operazioni, i volumi transazionali nell'ordine di milioni di operazioni giornaliere, e la gestione di dati altamente sensibili creano un insieme di vincoli ingegneristici che richiedono soluzioni architetturali innovative e robuste.

\subsection{Sfide Sistemiche della GDO Moderna}
\label{subsec:sfide-sistemiche}

La complessità sistemica della \gls{gdo} moderna deriva dalla convergenza di quattro fattori critici che definiscono i requisiti architetturali fondamentali:

\subsubsection{Operatività Continua e Vincoli Temporali Critici}

L'interruzione anche temporanea dei sistemi informatici in una catena commerciale genera impatti economici misurabili nell'ordine di centinaia di migliaia di euro per ora di downtime, secondo le analisi di Retail Systems Research\footnote{RETAIL SYSTEMS RESEARCH, \textit{The True Cost of Downtime in Retail Operations: 2024 Analysis}, Miami, RSR Analytics Division, 2024.}. Questa criticità temporale impone requisiti di disponibilità che spesso superano il 99.9\%, traducendosi in architetture fault-tolerant con ridondanza multi-livello e capacità di failover automatico.

\subsubsection{Distribuzione Geografica e Eterogeneità Infrastrutturale}

Una catena commerciale tipica opera attraverso centinaia di punti vendita distribuiti su territori vasti, ciascuno con caratteristiche infrastrutturali specifiche in termini di connettività, alimentazione elettrica, e vincoli ambientali. Questa distribuzione crea sfide di coordinamento che possono essere modellate utilizzando la teoria dei sistemi distribuiti, dove ogni punto vendita costituisce un nodo autonomo che deve mantenere coerenza operativa con il sistema centrale.

\subsubsection{Scalabilità Transazionale e Prestazioni Real-Time}

I volumi transazionali della \gls{gdo} seguono pattern di carico altamente variabili, con picchi che possono raggiungere il 300-500\% del carico medio durante eventi promozionali o periodi stagionali\footnote{FORRESTER RESEARCH, \textit{Peak Load Management in Retail IT Infrastructure: Performance Benchmarks and Scaling Strategies}, Cambridge, Forrester Consulting, 2024.}. La gestione di questi picchi richiede architetture elastiche capaci di scaling automatico mantenendo latenze nell'ordine dei millisecondi per le operazioni critiche come l'autorizzazione dei pagamenti.

\subsubsection{Gestione di Dati Sensibili e Vincoli Normativi}

La \gls{gdo} gestisce simultaneamente dati di pagamento soggetti a \gls{pci-dss}, dati personali sotto \gls{gdpr}, e informazioni commerciali critiche, creando un panorama di compliance che richiede architetture multi-tenant con isolamento granulare e controlli di accesso sofisticati.

\subsection{L'Evoluzione dell'Infrastruttura IT: Dai Data Center al Cloud Ibrido}
\label{subsec:evoluzione-infrastruttura}

L'evoluzione infrastrutturale della \gls{gdo} può essere periodizzata in tre fasi distinte, ciascuna caratterizzata da paradigmi architetturali specifici e sfide tecnologiche corrispondenti.

La \textbf{prima fase (1990-2010)} è stata caratterizzata da architetture centralizzate basate su mainframe e server dedicati, con terminali ``stupidi'' nei punti vendita. Questa architettura, pur semplificando la gestione centralizzata, presentava limitazioni critiche in termini di scalabilità e resilienza, con single point of failure che potevano paralizzare intere reti commerciali.

La \textbf{seconda fase (2010-2020)} ha visto l'adozione di architetture distribuite con server locali nei punti vendita principali e sistemi di replica dei dati. L'introduzione di tecnologie come la virtualizzazione e il software-defined networking ha permesso una maggiore flessibilità, ma ha anche introdotto nuove complessità in termini di gestione e sicurezza.

La \textbf{terza fase (2020-presente)} è caratterizzata dalla transizione verso architetture cloud-ibride che combinano l'elasticità del cloud pubblico con il controllo dell'infrastruttura on-premise. Questa evoluzione non rappresenta semplicemente una migrazione tecnologica, ma richiede un ripensamento fondamentale dei modelli operativi, delle strategie di sicurezza, e dei processi di governance.

\section{Framework di Analisi: Metodologia e Criteri di Valutazione}
\label{sec:framework-analisi}

L'analisi critica delle architetture IT per la \gls{gdo} richiede un framework metodologico che consideri simultaneamente multiple dimensioni di valutazione, bilanciando esigenze spesso contrastanti attraverso un approccio di ottimizzazione multi-obiettivo. Il framework sviluppato in questa tesi si basa su cinque criteri fondamentali che definiscono l'efficacia di un'architettura IT nel contesto della distribuzione commerciale.

\subsection{Criteri di Valutazione Sistemica}
\label{subsec:criteri-valutazione}

\subsubsection{Sicurezza (S)}

Valutata attraverso la capacità dell'architettura di proteggere dati sensibili, resistere ad attacchi informatici, e mantenere integrità operativa. La metrica di sicurezza incorpora fattori quantitativi come la riduzione della superficie di attacco, l'efficacia dei controlli di accesso, e la velocità di detection e response agli incidenti. Formalmente, la sicurezza può essere modellata come:

\begin{equation}
S = f(\text{protezione\_dati}, \text{resilienza\_attacchi}, \text{governance\_accessi})
\label{eq:sicurezza}
\end{equation}

dove ogni componente è quantificata su scale normalizzate.

\subsubsection{Scalabilità (Sc)}

Definita come la capacità dell'architettura di adattarsi a variazioni di carico mantenendo prestazioni accettabili. Include sia la scalabilità orizzontale (aggiunta di risorse) che verticale (potenziamento risorse esistenti), valutata attraverso metriche di throughput, latenza, e costi marginali di scaling. La scalabilità è particolarmente critica nella \gls{gdo} per gestire picchi stagionali e crescita organica.

\subsubsection{Compliance (C)}

Misurata attraverso l'aderenza a standard normativi (\gls{pci-dss}, \gls{gdpr}, \gls{nis2}) e la capacità di adattamento a requisiti normativi emergenti. Include valutazioni di audit readiness, automation della compliance, e costi di mantenimento della conformità. La compliance è modellata come:

\begin{equation}
C = \sum_{i} (\text{aderenza\_standard}_i \times \text{peso\_criticità}_i)
\label{eq:compliance}
\end{equation}

\subsubsection{Total Cost of Ownership (TCO)}

Valutazione economica che include costi di implementazione, operatività, manutenzione, e dismissione dell'architettura. Il \gls{tco} per la \gls{gdo} deve considerare costi distribuiti geograficamente, economia di scala, e impatti di downtime. La formula utilizzata è:

\begin{equation}
\text{TCO} = \text{CAPEX} + \text{OPEX} + \text{RISCHI}
\label{eq:tco}
\end{equation}

dove RISCHI include costi attesi di interruzioni e violazioni di sicurezza.

\subsubsection{Resilienza (R)}

Capacità dell'architettura di mantenere funzionalità operative in condizioni di guasto, attacco, o stress operativo. Include metriche di disponibilità, recovery time, e graceful degradation. La resilienza è quantificata attraverso:

\begin{equation}
R = f(\text{MTBF}, \text{MTTR}, \text{failover\_capabilities})
\label{eq:resilienza}
\end{equation}

dove \gls{mtbf} (Mean Time Between Failures) e \gls{mttr} (Mean Time To Recovery) sono misurati empiricamente.

\subsection{Metodologia di Analisi Multi-Criterio}
\label{subsec:metodologia-mcdm}

L'approccio metodologico adottato utilizza tecniche di analisi decisionale multi-criterio (\gls{mcdm}) per bilanciare i cinque criteri di valutazione. Ogni architettura analizzata viene valutata attraverso una funzione di utilità composita:

\begin{equation}
U(\text{architettura}) = w_1 \cdot S + w_2 \cdot Sc + w_3 \cdot C + w_4 \cdot \text{TCO}^{-1} + w_5 \cdot R
\label{eq:utilita-composita}
\end{equation}

Dove $w_1 \ldots w_5$ rappresentano pesi che riflettono le priorità strategiche specifiche del contesto operativo, e $\text{TCO}^{-1}$ indica che costi inferiori corrispondono a utilità superiore.

La metodologia prevede tre fasi di analisi:

\begin{enumerate}
    \item \textbf{Analisi Quantitativa}: Raccolta di metriche oggettive per ciascun criterio attraverso benchmark, case study, e dati empirici della letteratura scientifica e dell'industria.
    
    \item \textbf{Analisi Qualitativa}: Valutazione di fattori non facilmente quantificabili come facilità di gestione, vendor lock-in, e strategic alignment attraverso framework strutturati di valutazione.
    
    \item \textbf{Analisi Integrata}: Combinazione di valutazioni quantitative e qualitative attraverso tecniche di fuzzy logic e rough set theory per gestire l'incertezza e l'imprecisione nelle valutazioni.
\end{enumerate}

\section{Obiettivi della Ricerca e Contributo Originale}
\label{sec:obiettivi-ricerca}

\subsection{Obiettivo Primario: Analisi Critica dell'Evoluzione Architettuale}
\label{subsec:obiettivo-primario}

L'obiettivo principale di questa tesi è condurre un'analisi ingegneristica rigorosa dell'evoluzione dalle architetture IT tradizionali ai modelli cloud-ibridi nel contesto della Grande Distribuzione Organizzata, con particolare focus sulle implicazioni di sicurezza e compliance. Questa analisi va oltre la semplice comparazione tecnologica, mirando a identificare principi di progettazione e best practice che possano guidare decisioni architetturali strategiche.

L'analisi si articola attraverso tre dimensioni principali:

\subsubsection{Dimensione Tecnologica}

Valutazione critica delle tecnologie emergenti (edge computing, \gls{sd-wan}, cloud-native architectures) nel contesto specifico della \gls{gdo}, considerando non solo le capacità tecniche ma anche l'integrazione con sistemi legacy e l'impatto sui processi operativi.

\subsubsection{Dimensione di Sicurezza}

Analisi approfondita dell'evoluzione del threat landscape specifico per la \gls{gdo} e sviluppo di framework di sicurezza che integrino principi Zero Trust con le esigenze operative del settore retail.

\subsubsection{Dimensione Normativa}

Esame dell'impatto delle normative emergenti (\gls{nis2}, evoluzione \gls{gdpr}, \gls{pci-dss} v4.0) sulle scelte architetturali e sviluppo di approcci di compliance-by-design per architetture ibride.

\subsection{Contributi Originali Attesi}
\label{subsec:contributi-originali}

Il contributo originale di questa tesi si articola su quattro livelli:

\subsubsection{Contributo Metodologico}

Sviluppo di un framework di valutazione multi-criterio specificamente calibrato per le esigenze della \gls{gdo}, che integra metriche quantitative di performance e sicurezza con valutazioni qualitative di governance e strategic fit.

\subsubsection{Contributo Analitico}

Analisi sistemica delle interdipendenze tra evoluzione infrastrutturale e trasformazione del panorama delle minacce, identificando pattern di vulnerabilità emergenti in architetture cloud-ibride specifiche del retail.

\subsubsection{Contributo Progettuale}

Definizione di principi di progettazione (design principles) per architetture IT sicure nella \gls{gdo} che bilancino efficacemente sicurezza, performance, e compliance in contesti operativi distribuiti.

\subsubsection{Contributo Strategico}

Sviluppo di una roadmap strategica per la transizione verso architetture cloud-first nella \gls{gdo} che consideri non solo aspetti tecnologici ma anche implicazioni organizzative, economiche, e di risk management.

\subsection{Ipotesi di Ricerca}
\label{subsec:ipotesi-ricerca}

La ricerca si basa su tre ipotesi fondamentali che verranno validate attraverso l'analisi:

\begin{hypothesis}[H1: Efficacia delle Architetture Cloud-Ibride]
L'adozione di architetture cloud-ibride nella \gls{gdo} può migliorare simultaneamente sicurezza e performance rispetto ad architetture tradizionali, purché vengano implementati controlli di sicurezza appropriati e strategie di orchestrazione intelligente.
\label{hyp:cloud-ibride}
\end{hypothesis}

\begin{hypothesis}[H2: Integrazione Zero Trust]
L'integrazione di principi Zero Trust in architetture distribuite per la \gls{gdo} può ridurre significativamente la superficie di attacco senza compromettere l'esperienza operativa, attraverso l'automazione intelligente dei controlli di accesso.
\label{hyp:zero-trust}
\end{hypothesis}

\begin{hypothesis}[H3: Compliance-by-Design]
L'implementazione di compliance-by-design in architetture cloud-ibride può potenzialmente ridurre i costi di conformità normativa del 30-50\% rispetto ad approcci retrofitting, mantenendo o migliorando l'efficacia dei controlli. \textit{Questa ipotesi quantitativa verrà validata attraverso l'analisi comparativa dei casi di studio presentati nei capitoli successivi.}
\label{hyp:compliance-by-design}
\end{hypothesis}

\section{Struttura della Tesi: Roadmap Dal Fisico al Digitale}
\label{sec:struttura-tesi}

La struttura di questa tesi segue una progressione logica che parte dall'analisi del panorama delle minacce per giungere alla definizione di architetture ottimali, seguendo quello che definiamo il percorso ``dal fisico al digitale''.

\subsection{Architettura della Ricerca}
\label{subsec:architettura-ricerca}

\subsubsection{Capitolo 2 - Threat Landscape e Sicurezza Distribuita}

Questo capitolo costituisce il fondamento analitico della ricerca, esaminando l'evoluzione delle minacce specifiche per la \gls{gdo} e l'efficacia delle tecnologie di difesa contemporanee. L'analisi parte da dati empirici su attacchi documentati per sviluppare modelli predittivi di rischio e identificare gap nelle difese tradizionali.

\subsubsection{Capitolo 3 - Evoluzione Infrastrutturale}

Il terzo capitolo analizza la transizione da architetture fisiche tradizionali a modelli cloud-first, esaminando tanto gli aspetti fisici (alimentazione, cooling, connettività) quanto quelli logici (\gls{sd-wan}, edge computing, orchestrazione). Particolare attenzione è dedicata ai pattern di migrazione e alle strategie di coesistenza tra sistemi legacy e architetture moderne.

\subsubsection{Capitolo 4 - Compliance Integrata e Governance}

Questo capitolo affronta la complessità normativa della \gls{gdo} moderna, analizzando come l'evoluzione architettuale impatti sulla conformità a standard multipli. Include un caso di studio dettagliato su cyber-physical attack che dimostra l'interconnessione tra sicurezza IT e OT nel contesto retail.

\subsubsection{Capitolo 5 - Sintesi e Direzioni Strategiche}

Il capitolo conclusivo sintetizza i risultati dell'analisi in un framework integrato di best practice, fornendo una roadmap strategica per l'evoluzione futura delle architetture \gls{gdo} considerando tendenze emergenti come AI-powered security e sustainable IT.

\subsection{Metodologia di Ricerca Integrata}
\label{subsec:metodologia-integrata}

La metodologia di ricerca combina approcci quantitativi e qualitativi per massimizzare la robustezza dei risultati:

\begin{description}
    \item[Analisi Documentale] Revisione sistematica della letteratura scientifica (IEEE, ACM) e dei report industriali (Gartner, Forrester) per identificare stato dell'arte e trend emergenti.
    
    \item[Analisi Empirica] Utilizzo di dataset pubblici su incidenti di sicurezza nel retail e benchmark di performance per validare modelli teorici con evidenze empiriche.
    
    \item[Case Study Analysis] Analisi dettagliata di implementazioni reali attraverso case study documentati, con particolare focus su successi e fallimenti nell'adozione di architetture cloud-ibride.
    
    \item[Modellazione Matematica] Sviluppo di modelli formali per valutare trade-off tra sicurezza, performance, e costi utilizzando tecniche di ottimizzazione multi-obiettivo.
\end{description}

\subsection{Limitazioni e Boundary Conditions}
\label{subsec:limitazioni}

È importante definire chiaramente i boundary conditions di questa ricerca:

\begin{description}
    \item[Scope Geografico] L'analisi si concentra principalmente sul contesto europeo e nordamericano, dove la maturità normativa e tecnologica permette confronti significativi.
    
    \item[Dimensione Aziendale] Il focus è su organizzazioni \gls{gdo} di medie e grandi dimensioni (>100 punti vendita) dove la complessità architettuale giustifica investimenti in soluzioni avanzate.
    
    \item[Orizzonte Temporale] L'analisi considera un orizzonte di 3-5 anni per le proiezioni strategiche, oltre il quale l'incertezza tecnologica e normativa rende le previsioni meno affidabili.
    
    \item[Vincoli di Accesso] Alcune analisi sono limitate da restrizioni di accesso a dati proprietari delle aziende, compensate attraverso l'utilizzo di dataset pubblici e benchmark standardizzati.
\end{description}

\section{Rilevanza e Impatto Atteso}
\label{sec:rilevanza-impatto}

\subsection{Rilevanza Scientifica}
\label{subsec:rilevanza-scientifica}

Dal punto di vista della ricerca in ingegneria informatica, questa tesi contribuisce all'avanzamento delle conoscenze in diversi ambiti:

\begin{description}
    \item[Security Engineering] Sviluppo di modelli di sicurezza specifici per ambienti retail distribuiti che considerano l'interazione tra componenti IT e OT.
    
    \item[Distributed Systems] Analisi di pattern architetturali per sistemi distribuiti su larga scala con vincoli di latenza e disponibilità estremi.
    
    \item[Compliance Engineering] Formalizzazione di approcci ingegneristici alla conformità normativa in architetture complesse.
\end{description}

\subsection{Rilevanza Industriale}
\label{subsec:rilevanza-industriale}

L'impatto industriale atteso include:

\begin{description}
    \item[Decision Support] Fornire a decision maker del settore retail framework quantitativi per valutare investimenti in sicurezza e infrastruttura IT.
    
    \item[Risk Management] Sviluppare metodologie di risk assessment specifiche per il settore che considerino l'evoluzione del threat landscape.
    
    \item[Strategic Planning] Supportare la pianificazione strategica IT attraverso roadmap validate che bilancino innovazione e gestione del rischio.
\end{description}

\subsection{Impatto Sociale ed Economico}
\label{subsec:impatto-sociale-economico}

La sicurezza delle infrastrutture IT nella \gls{gdo} ha implicazioni che trascendono il singolo settore:

\begin{description}
    \item[Protezione del Consumatore] Miglioramento della protezione dei dati personali e finanziari dei consumatori attraverso architetture più sicure.
    
    \item[Stabilità Economica] Contributo alla resilienza del sistema economico attraverso la protezione di supply chain critiche.
    
    \item[Innovazione Sostenibile] Promozione di approcci all'innovazione IT che bilancino progresso tecnologico e responsabilità sociale.
\end{description}

\vspace{1cm}

Il percorso di analisi delineato in questa introduzione mira a fornire contributi tanto alla comunità scientifica quanto all'industria, sviluppando conoscenze che possano guidare l'evoluzione sostenibile e sicura dell'infrastruttura IT nel settore della Grande Distribuzione Organizzata. La metodologia rigorosa e i framework sviluppati potranno essere adattati anche ad altri settori con caratteristiche simili di complessità distribuita e criticità operativa.