% =====================================
% GRAFICI E DIAGRAMMI - CAPITOLO 3 EVOLUZIONE INFRASTRUTTURALE
% "Da Data Center a Cloud-First"
% Compatibile con XeLaTeX
% =====================================

\documentclass[12pt,a4paper]{article}
\usepackage{fontspec}
\usepackage{polyglossia}
\setmainlanguage{italian}
\usepackage{amsmath}
\usepackage{amsfonts}
\usepackage{amssymb}
\usepackage{tikz}
\usepackage{pgfplots}
\usepackage{pgfplotstable}
\usepackage{booktabs}
\usepackage{array}
\usepackage{multirow}
\usepackage{longtable}
\usepackage{xcolor}
\usepackage{geometry}

% Configurazioni PGFPlots
\pgfplotsset{compat=1.18}
\usetikzlibrary{positioning,shapes,arrows,calc,patterns,decorations.pathreplacing,fit,backgrounds,matrix}

% Definizione colori tema
\definecolor{gdoblu}{RGB}{25,118,188}
\definecolor{gdorosso}{RGB}{220,38,58}
\definecolor{gdoverde}{RGB}{0,150,108}
\definecolor{gdoarancio}{RGB}{255,165,0}
\definecolor{gdogrigio}{RGB}{128,128,128}
\definecolor{gdovioletto}{RGB}{128,0,128}

\geometry{margin=2cm}

\begin{document}

% =====================================
% FIGURA 3.1: ARCHITETTURA SD-WAN GDO
% =====================================

\begin{figure}[htbp]
\centering
\begin{tikzpicture}[node distance=2cm, scale=0.85, transform shape]
% Definizione stili
\tikzset{
    controller/.style={rectangle, rounded corners, minimum width=3cm, minimum height=2cm, text centered, draw=gdoblu, fill=gdoblu!20, thick},
    store/.style={rectangle, rounded corners, minimum width=2cm, minimum height=1.5cm, text centered, draw=gdoverde, fill=gdoverde!20, thick},
    connection/.style={->, thick, gdogrigio},
    control/.style={->, thick, gdorosso, dashed},
    data/.style={-, thick, gdoblu},
    internet/.style={ellipse, minimum width=2.5cm, minimum height=1.5cm, text centered, draw=gdoarancio, fill=gdoarancio!20}
}

% Orchestratore Centrale
\node[controller] (orchestrator) at (0,6) {
\begin{minipage}{2.8cm}
\centering
\textbf{SD-WAN Orchestrator}

• Policy Management
• Path Optimization
• Centralized Control
\end{minipage}
};

% Controller Regionali
\node[controller, minimum width=2.5cm] (controller1) at (-4,3) {
\begin{minipage}{2.2cm}
\centering
\textbf{Regional Controller A}

• Local Policy
• Backup Control
\end{minipage}
};

\node[controller, minimum width=2.5cm] (controller2) at (4,3) {
\begin{minipage}{2.2cm}
\centering
\textbf{Regional Controller B}

• Local Policy  
• Backup Control
\end{minipage}
};

% Store Network
\node[store] (store1) at (-6,0) {
\begin{minipage}{1.8cm}
\centering
\textbf{Store 1}

SD-WAN Edge
\end{minipage}
};

\node[store] (store2) at (-3,0) {
\begin{minipage}{1.8cm}
\centering
\textbf{Store 2}

SD-WAN Edge
\end{minipage}
};

\node[store] (store3) at (3,0) {
\begin{minipage}{1.8cm}
\centering
\textbf{Store N}

SD-WAN Edge
\end{minipage}
};

\node[store] (store4) at (6,0) {
\begin{minipage}{1.8cm}
\centering
\textbf{Store N+1}

SD-WAN Edge
\end{minipage}
};

% Internet/Cloud
\node[internet] (internet) at (0,0) {\textbf{Internet/MPLS}};

% Data Center
\node[controller, fill=gdovioletto!20, draw=gdovioletto] (datacenter) at (0,-3) {
\begin{minipage}{2.8cm}
\centering
\textbf{Data Center HQ}

• Core Applications
• Analytics Platform
• Security Services
\end{minipage}
};

% Connessioni di controllo
\draw[control] (orchestrator) -- (controller1) node[midway, above left] {\footnotesize Control Plane};
\draw[control] (orchestrator) -- (controller2) node[midway, above right] {\footnotesize Control Plane};
\draw[control] (controller1) -- (store1);
\draw[control] (controller1) -- (store2);
\draw[control] (controller2) -- (store3);
\draw[control] (controller2) -- (store4);

% Connessioni dati
\draw[data] (store1) -- (internet);
\draw[data] (store2) -- (internet);
\draw[data] (store3) -- (internet);
\draw[data] (store4) -- (internet);
\draw[data] (internet) -- (datacenter);

% Path alternativi (backup)
\draw[data, dashed] (store1) -- (store2) node[midway, above] {\footnotesize Backup Path};
\draw[data, dashed] (store3) -- (store4) node[midway, above] {\footnotesize Backup Path};

% QoS Indicators
\node at (-8,1.5) {
\begin{minipage}{2cm}
\footnotesize
\textbf{QoS Classes:}
• Real-time (EF)
• Business (AF31)
• Best Effort (BE)
\end{minipage}
};

% Performance Metrics
\node at (8,1.5) {
\begin{minipage}{2cm}
\footnotesize
\textbf{Metrics:}
• Latenza < 100ms
• Jitter < 10ms
• Loss < 0.1\%
\end{minipage}
};

\end{tikzpicture}
\caption{Architettura SD-WAN per la GDO: controllo centralizzato con resilienza distribuita}
\label{fig:sdwan_architecture}
\end{figure}

% =====================================
% FIGURA 3.2: MATRICE DECISIONALE MIGRAZIONE CLOUD
% =====================================

\begin{figure}[htbp]
\centering
\begin{tikzpicture}[scale=1.1]
% Assi della matrice
\draw[->] (0,0) -- (8,0) node[right] {\textbf{Complessità Applicativa}};
\draw[->] (0,0) -- (0,8) node[above] {\textbf{Benefici Cloud Attesi}};

% Griglia di sfondo
\foreach \x in {1,2,3,4,5,6,7} {
  \draw[gray!30] (\x,0) -- (\x,8);
}
\foreach \y in {1,2,3,4,5,6,7} {
  \draw[gray!30] (0,\y) -- (8,\y);
}

% Zone strategiche
\fill[gdoverde!20] (0,0) rectangle (3,4);
\fill[gdoarancio!20] (3,0) rectangle (6,4);
\fill[gdoarancio!20] (0,4) rectangle (3,8);
\fill[gdorosso!20] (3,4) rectangle (8,8);
\fill[gdovioletto!20] (6,0) rectangle (8,4);

% Etichette zone
\node at (1.5,2) {\begin{minipage}{2.5cm}\centering\textbf{Lift-and-Shift}\footnotesize Migrazione rapida\end{minipage}};
\node at (4.5,2) {\begin{minipage}{2.5cm}\centering\textbf{Replatforming}\footnotesize Ottimizzazioni minime\end{minipage}};
\node at (1.5,6) {\begin{minipage}{2.5cm}\centering\textbf{Refactoring}\footnotesize Ristrutturazione\end{minipage}};
\node at (5.5,6) {\begin{minipage}{2.5cm}\centering\textbf{Rebuild}\footnotesize Ricostruzione completa\end{minipage}};
\node at (7,2) {\begin{minipage}{1.5cm}\centering\textbf{Retire}\footnotesize Legacy\end{minipage}};

% Esempi applicazioni
\node[circle, fill=gdoblu, text=white, inner sep=2pt] at (1,1.5) {\footnotesize POS};
\node[circle, fill=gdoblu, text=white, inner sep=2pt] at (2.5,1) {\footnotesize Web};
\node[circle, fill=gdoblu, text=white, inner sep=2pt] at (4,2.5) {\footnotesize ERP};
\node[circle, fill=gdoblu, text=white, inner sep=2pt] at (5.5,3) {\footnotesize CRM};
\node[circle, fill=gdoblu, text=white, inner sep=2pt] at (2,5.5) {\footnotesize Analytics};
\node[circle, fill=gdoblu, text=white, inner sep=2pt] at (4.5,6.5) {\footnotesize AI/ML};
\node[circle, fill=gdoblu, text=white, inner sep=2pt] at (7,1.5) {\footnotesize Legacy};

% Scale
\foreach \x in {0,2,4,6,8} {
  \node[below] at (\x,-0.3) {\x};
}
\foreach \y in {0,2,4,6,8} {
  \node[left] at (-0.3,\y) {\y};
}

% Legenda
\node at (10,6) {
\begin{minipage}{3cm}
\footnotesize
\textbf{Legenda Strategie:}

\tikz \fill[gdoverde!20] (0,0) rectangle (0.3,0.2); Lift-and-Shift

\tikz \fill[gdoarancio!20] (0,0) rectangle (0.3,0.2); Replatforming

\tikz \fill[gdorosso!20] (0,0) rectangle (0.3,0.2); Refactoring

\tikz \fill[gdovioletto!20] (0,0) rectangle (0.3,0.2); Retire/Replace
\end{minipage}
};

\end{tikzpicture}
\caption{Matrice decisionale per la selezione della strategia di migrazione cloud}
\label{fig:migration_decision_matrix}
\end{figure}

% =====================================
% TABELLA 3.1: CONFRONTO STRATEGIE MIGRAZIONE
% =====================================

\begin{table}[htbp]
\centering
\caption{Confronto Strategie di Migrazione Cloud: Tempi, Costi e Benefici}
\label{tab:migration_strategies_comparison}
\begin{tabular}{@{}lccccc@{}}
\toprule
\textbf{Strategia} & \textbf{Timeline} & \textbf{Costo Iniziale} & \textbf{Benefici Cloud} & \textbf{ROI 3 anni} & \textbf{Rischio} \\
\midrule
Lift-and-Shift & 3-6 mesi & Basso & 20-30\% & 150-200\% & Basso \\
\addlinespace
Replatforming & 6-12 mesi & Medio & 30-40\% & 200-300\% & Medio \\
\addlinespace
Refactoring & 12-24 mesi & Alto & 50-70\% & 300-500\% & Alto \\
\addlinespace
Rebuild & 24-48 mesi & Molto Alto & 70-90\% & 400-800\% & Molto Alto \\
\addlinespace
Retire/Replace & 6-18 mesi & Variabile & 60-80\% & 250-400\% & Medio \\
\bottomrule
\end{tabular}
\end{table}

% =====================================
% FIGURA 3.3: ROADMAP STRATEGICA TRANSIZIONE
% =====================================

\begin{figure}[htbp]
\centering
\begin{tikzpicture}[node distance=1.5cm, scale=0.9, transform shape]
% Definizione stili
\tikzset{
    phase/.style={rectangle, rounded corners, minimum width=3cm, minimum height=2.5cm, text centered, draw=black, thick},
    milestone/.style={diamond, minimum width=1.5cm, minimum height=1cm, text centered, draw=gdorosso, fill=gdorosso!20, thick},
    dependency/.style={->, thick, gdogrigio, dashed}
}

% Timeline principale
\draw[->] (0,0) -- (15,0) node[right] {\textbf{Timeline (mesi)}};

% Fasi della roadmap
\node[phase, fill=gdorosso!20] (fase1) at (2.5,3) {
\begin{minipage}{2.8cm}
\centering
\textbf{Fase 1: Foundation}
\footnotesize (0-12 mesi)

• Virtualizzazione
• SD-WAN deployment
• Infrastructure upgrade
• Team training
\end{minipage}
};

\node[phase, fill=gdoarancio!20] (fase2) at (5.5,3) {
\begin{minipage}{2.8cm}
\centering
\textbf{Fase 2: Hybrid}
\footnotesize (12-24 mesi)

• Cloud migration pilot
• Edge computing
• Automation tools
• Multi-cloud setup
\end{minipage}
};

\node[phase, fill=gdoverde!20] (fase3) at (8.5,3) {
\begin{minipage}{2.8cm}
\centering
\textbf{Fase 3: Cloud-First}
\footnotesize (24-36 mesi)

• Core apps migration
• Container orchestration
• AI/ML integration
• Zero Trust security
\end{minipage}
};

\node[phase, fill=gdoblu!20] (fase4) at (11.5,3) {
\begin{minipage}{2.8cm}
\centering
\textbf{Fase 4: Autonomous}
\footnotesize (36+ mesi)

• Self-healing infra
• Predictive scaling
• AI-driven optimization
• Sustainable IT
\end{minipage}
};

% Milestone
\node[milestone] (m1) at (2.5,1) {\footnotesize M1};
\node[milestone] (m2) at (5.5,1) {\footnotesize M2};
\node[milestone] (m3) at (8.5,1) {\footnotesize M3};
\node[milestone] (m4) at (11.5,1) {\footnotesize M4};

% Connessioni milestone-fasi
\draw[->] (m1) -- (fase1);
\draw[->] (m2) -- (fase2);
\draw[->] (m3) -- (fase3);
\draw[->] (m4) -- (fase4);

% Dipendenze tra fasi
\draw[dependency] (fase1.east) -- (fase2.west);
\draw[dependency] (fase2.east) -- (fase3.west);
\draw[dependency] (fase3.east) -- (fase4.west);

% Metriche di successo
\node at (2.5,6) {
\begin{minipage}{2.8cm}
\centering
\footnotesize
\textbf{Target M1:}
• 90\% virtualizzazione
• 99.9\% uptime
• 25\% cost reduction
\end{minipage}
};

\node at (5.5,6) {
\begin{minipage}{2.8cm}
\centering
\footnotesize
\textbf{Target M2:}
• 30\% cloud workload
• 60\% deployment speed
• 1h DR RTO
\end{minipage}
};

\node at (8.5,6) {
\begin{minipage}{2.8cm}
\centering
\footnotesize
\textbf{Target M3:}
• 70\% cloud adoption
• 40\% OPEX reduction
• 30\% CX improvement
\end{minipage}
};

\node at (11.5,6) {
\begin{minipage}{2.8cm}
\centering
\footnotesize
\textbf{Target M4:}
• 90\% automation
• 75\% MTTR reduction
• Carbon neutral IT
\end{minipage}
};

% Timeline markers
\foreach \x in {1,2,...,12} {
  \draw (\x,-0.1) -- (\x,0.1);
  \node[below] at (\x,-0.3) {\footnotesize \x};
}

% Investment curve
\begin{scope}[yshift=-3cm]
\begin{axis}[
    width=12cm,
    height=4cm,
    xlabel={\footnotesize Mesi},
    ylabel={\footnotesize Investimento Cumulativo (M€)},
    xmin=0, xmax=48,
    ymin=0, ymax=8,
    grid=major,
    no marks
]
\addplot[gdoblu, thick] coordinates {
    (0,0) (6,1.5) (12,3) (18,4) (24,5.5) (30,6.5) (36,7) (42,7.5) (48,8)
};
\addplot[gdoverde, thick] coordinates {
    (0,0) (12,0) (18,0.5) (24,1.5) (30,3) (36,5) (42,7) (48,9)
};
\legend{\footnotesize Investimenti, \footnotesize Savings Cumulativi}
\end{axis}
\end{scope}

\end{tikzpicture}
\caption{Roadmap strategica per la transizione cloud-first: timeline, dipendenze e target}
\label{fig:strategic_roadmap}
\end{figure}

% =====================================
% FIGURA 3.4: EDGE COMPUTING ARCHITECTURE
% =====================================

\begin{figure}[htbp]
\centering
\begin{tikzpicture}[node distance=2cm, scale=0.9, transform shape]
% Definizione stili
\tikzset{
    device/.style={rectangle, rounded corners, minimum width=1.5cm, minimum height=1cm, text centered, draw=gdoarancio, fill=gdoarancio!20, thick},
    edge/.style={rectangle, rounded corners, minimum width=2.5cm, minimum height=1.8cm, text centered, draw=gdoverde, fill=gdoverde!20, thick},
    regional/.style={rectangle, rounded corners, minimum width=3cm, minimum height=2.5cm, text centered, draw=gdoblu, fill=gdoblu!20, thick},
    cloud/.style={ellipse, minimum width=3cm, minimum height=2cm, text centered, draw=gdovioletto, fill=gdovioletto!20, thick},
    dataflow/.style={<->, thick, gdogrigio}
}

% Device Edge Layer
\node[device] (sensor1) at (0,0) {\footnotesize IoT Sensor};
\node[device] (camera1) at (2,0) {\footnotesize Smart Camera};
\node[device] (pos1) at (4,0) {\footnotesize POS System};
\node[device] (sensor2) at (6,0) {\footnotesize RFID Reader};

% Infrastructure Edge Layer
\node[edge] (edge1) at (1,3) {
\begin{minipage}{2.2cm}
\centering
\textbf{Store Edge Server}

• Local processing
• Caching layer
• Real-time analytics
\end{minipage}
};

\node[edge] (edge2) at (5,3) {
\begin{minipage}{2.2cm}
\centering
\textbf{Store Edge Server}

• Local processing
• Caching layer  
• Real-time analytics
\end{minipage}
};

% Regional Edge Layer
\node[regional] (regional) at (3,6.5) {
\begin{minipage}{2.8cm}
\centering
\textbf{Regional Data Center}

• Aggregated analytics
• ML model training
• Regional services
• Backup \& DR
\end{minipage}
};

% Cloud Layer
\node[cloud] (cloud) at (3,10) {
\begin{minipage}{2.8cm}
\centering
\textbf{Public Cloud}

• Global analytics
• AI/ML platforms
• Enterprise services
• Long-term storage
\end{minipage}
};

% Connessioni
\draw[dataflow] (sensor1) -- (edge1);
\draw[dataflow] (camera1) -- (edge1);
\draw[dataflow] (pos1) -- (edge2);
\draw[dataflow] (sensor2) -- (edge2);

\draw[dataflow] (edge1) -- (regional);
\draw[dataflow] (edge2) -- (regional);

\draw[dataflow] (regional) -- (cloud);

% Latency indicators
\node at (-2,1.5) {
\begin{minipage}{2cm}
\footnotesize
\textbf{Device→Edge:}
< 5ms
\end{minipage}
};

\node at (-2,4.5) {
\begin{minipage}{2cm}
\footnotesize
\textbf{Edge→Regional:}
< 50ms
\end{minipage}
};

\node at (-2,8) {
\begin{minipage}{2cm}
\footnotesize
\textbf{Regional→Cloud:}
< 200ms
\end{minipage}
};

% Processing capabilities
\node at (8,1.5) {
\begin{minipage}{2.5cm}
\footnotesize
\textbf{Device Edge:}
• Simple filtering
• Basic aggregation
• Local decisions
\end{minipage}
};

\node at (8,4.5) {
\begin{minipage}{2.5cm}
\footnotesize
\textbf{Infrastructure Edge:}
• Complex analytics
• ML inference
• Local storage
\end{minipage}
};

\node at (8,8) {
\begin{minipage}{2.5cm}
\footnotesize
\textbf{Regional/Cloud:}
• ML training
• Big data analytics
• Global coordination
\end{minipage}
};

% Data flow volumes
\draw[<-, thick, gdorosso] (0.5,11) -- (0.5,9.5) node[left, midway] {\footnotesize 1GB/day};
\draw[<-, thick, gdorosso] (0.5,7.5) -- (0.5,6) node[left, midway] {\footnotesize 100GB/day};
\draw[<-, thick, gdorosso] (0.5,4.5) -- (0.5,3) node[left, midway] {\footnotesize 1TB/day};

\end{tikzpicture}
\caption{Architettura edge computing per la GDO: elaborazione distribuita multi-tier}
\label{fig:edge_computing_architecture}
\end{figure}

% =====================================
% TABELLA 3.2: ANALISI ROI TRANSIZIONE CLOUD
% =====================================

\begin{table}[htbp]
\centering
\caption{Analisi ROI della Transizione Cloud: Proiezione Quinquennale}
\label{tab:roi_analysis_cloud_transition}
\begin{tabular}{@{}lccccc@{}}
\toprule
\textbf{Anno} & \textbf{Investimenti (M€)} & \textbf{Savings OPEX (M€)} & \textbf{Benefici Strategici (M€)} & \textbf{Cash Flow (M€)} & \textbf{NPV Cumulativo (M€)} \\
\midrule
0 & -4.0 & 0.0 & 0.0 & -4.0 & -4.0 \\
1 & -1.5 & 0.8 & 0.2 & -0.5 & -4.5 \\
2 & -0.5 & 1.5 & 0.5 & 1.5 & -3.1 \\
3 & -0.3 & 2.2 & 1.1 & 3.0 & -0.6 \\
4 & -0.2 & 2.8 & 1.8 & 4.4 & 3.2 \\
5 & -0.1 & 3.2 & 2.5 & 5.6 & 8.1 \\
\midrule
\multicolumn{4}{l}{\textbf{NPV Totale (8\% discount rate):}} & \textbf{€2.1M} \\
\multicolumn{4}{l}{\textbf{IRR:}} & \textbf{24\%} \\
\multicolumn{4}{l}{\textbf{Payback Period:}} & \textbf{2.8 anni} \\
\bottomrule
\end{tabular}
\end{table}

% =====================================
% FIGURA 3.5: POWER EFFICIENCY ANALYSIS
% =====================================

\begin{figure}[htbp]
\centering
\begin{tikzpicture}
\begin{axis}[
    title={Analisi Efficienza Energetica: UPS vs Load Factor},
    xlabel={Load Factor (\%)},
    ylabel={Efficienza (\%)},
    width=12cm,
    height=8cm,
    xmin=10,
    xmax=100,
    ymin=85,
    ymax=98,
    grid=major,
    legend style={at={(0.02,0.98)}, anchor=north west},
]

% Curve di efficienza per diverse configurazioni UPS
\addplot[color=gdoblu, mark=*, line width=2pt] coordinates {
    (10, 87) (20, 91) (30, 93) (40, 95) (50, 96.5) (60, 97) (70, 97.2) (80, 97) (90, 96.5) (100, 95.8)
};

\addplot[color=gdoverde, mark=square*, line width=2pt] coordinates {
    (10, 89) (20, 93) (30, 95) (40, 96.5) (50, 97.5) (60, 98) (70, 98.2) (80, 98.1) (90, 97.8) (100, 97.2)
};

\addplot[color=gdorosso, mark=triangle*, line width=2pt] coordinates {
    (10, 85) (20, 88) (30, 90) (40, 92) (50, 94) (60, 95) (70, 95.5) (80, 95.2) (90, 94.8) (100, 94)
};

% Zona operativa ottimale
\fill[gdoverde!20, opacity=0.3] (axis cs:40,85) rectangle (axis cs:80,98);
\node at (axis cs:60,87) {\footnotesize Zona Operativa Ottimale};

% Linea target efficiency
\draw[dashed, gdoarancio, line width=2pt] (axis cs:10,95) -- (axis cs:100,95) node[right] {\footnotesize Target 95\%};

\legend{UPS Online Standard, UPS Eco-Mode, UPS Legacy}

\end{axis}
\end{tikzpicture}
\caption{Curve di efficienza energetica per diverse configurazioni UPS in funzione del load factor}
\label{fig:ups_efficiency_analysis}
\end{figure}

% =====================================
% TABELLA 3.3: COMPARISON COOLING STRATEGIES
% =====================================

\begin{table}[htbp]
\centering
\caption{Confronto Strategie di Cooling per Ambienti IT Retail}
\label{tab:cooling_strategies_comparison}
\begin{tabular}{@{}lcccc@{}}
\toprule
\textbf{Strategia} & \textbf{PUE Tipico} & \textbf{CAPEX Relativo} & \textbf{OPEX Annuo} & \textbf{Efficienza Energetica} \\
\midrule
Traditional CRAC & 2.2-2.5 & 100\% & 100\% & Baseline \\
\addlinespace
Hot/Cold Aisle & 1.8-2.2 & 120\% & 85\% & +15\% \\
\addlinespace
Containment & 1.6-1.9 & 140\% & 70\% & +30\% \\
\addlinespace
Free Cooling & 1.4-1.7 & 160\% & 60\% & +40\% \\
\addlinespace
Liquid Cooling & 1.2-1.4 & 200\% & 50\% & +50\% \\
\bottomrule
\end{tabular}
\end{table}

% =====================================
% FIGURA 3.6: RISK ASSESSMENT MATRIX
% =====================================

\begin{figure}[htbp]
\centering
\begin{tikzpicture}[scale=1.0]
% Matrice di rischio
\draw[->] (0,0) -- (6,0) node[right] {\textbf{Probabilità}};
\draw[->] (0,0) -- (0,6) node[above] {\textbf{Impatto}};

% Griglia
\foreach \x in {1,2,3,4,5} {
  \draw[gray!30] (\x,0) -- (\x,6);
}
\foreach \y in {1,2,3,4,5} {
  \draw[gray!30] (0,\y) -- (6,\y);
}

% Zone di rischio
\fill[gdoverde!30] (0,0) rectangle (2,2);
\fill[gdoarancio!30] (2,0) rectangle (4,2);
\fill[gdoarancio!30] (0,2) rectangle (2,4);
\fill[gdorosso!30] (2,2) rectangle (6,6);
\fill[gdoverde!30] (4,0) rectangle (6,2);
\fill[gdoarancio!30] (0,4) rectangle (2,6);

% Labels zone
\node at (1,1) {\footnotesize Basso};
\node at (3,1) {\footnotesize Medio};
\node at (1,3) {\footnotesize Medio};
\node at (4,4) {\footnotesize Alto};
\node at (5,1) {\footnotesize Basso};
\node at (1,5) {\footnotesize Medio};

% Posizionamento rischi specifici
\node[circle, fill=gdoblu, text=white, inner sep=1pt] at (1.5,1.5) {\footnotesize 1};
\node[circle, fill=gdoblu, text=white, inner sep=1pt] at (2.5,3) {\footnotesize 2};
\node[circle, fill=gdoblu, text=white, inner sep=1pt] at (3.5,2.5) {\footnotesize 3};
\node[circle, fill=gdoblu, text=white, inner sep=1pt] at (1.5,4.5) {\footnotesize 4};
\node[circle, fill=gdoblu, text=white, inner sep=1pt] at (4.5,3.5) {\footnotesize 5};
\node[circle, fill=gdoblu, text=white, inner sep=1pt] at (3,4.5) {\footnotesize 6};

% Scale
\foreach \x in {0,1,2,3,4,5} {
  \node[below] at (\x,-0.2) {\footnotesize \x};
}
\foreach \y in {0,1,2,3,4,5} {
  \node[left] at (-0.2,\y) {\footnotesize \y};
}

% Legenda rischi
\node at (8,4) {
\begin{minipage}{4cm}
\footnotesize
\textbf{Rischi Identificati:}

1. Performance degradation
2. Skills gap organizzativo  
3. Vendor lock-in
4. Security vulnerabilities
5. Integration complexity
6. Regulatory changes

\textbf{Livelli di Rischio:}
\tikz \fill[gdoverde!30] (0,0) rectangle (0.3,0.15); Accettabile
\tikz \fill[gdoarancio!30] (0,0) rectangle (0.3,0.15); Monitoraggio
\tikz \fill[gdorosso!30] (0,0) rectangle (0.3,0.15); Mitigazione
\end{minipage}
};

\end{tikzpicture}
\caption{Matrice di valutazione rischi per la transizione cloud: probabilità vs impatto}
\label{fig:risk_assessment_matrix}
\end{figure}

\end{document}