% !TEX program = xelatex
\documentclass[12pt,a4paper,oneside]{book}

% ==========================================
% PACCHETTI ESSENZIALI PER XeLaTeX
% ==========================================
\usepackage{fontspec}
\usepackage[italian]{babel}

% Font Arial nativo dal sistema (REGOLA UNIVERSITÀ)
\setmainfont{Arial}[
    Ligatures=TeX,
    Numbers=Lining,
    BoldFont=Arial Bold,
    ItalicFont=Arial Italic,
    BoldItalicFont=Arial Bold Italic,
]

% MARGINI ESATTI SECONDO REGOLAMENTO UNIVERSITÀ
\usepackage[
    top=3.5cm,        % Superiore: 3,5 cm
    bottom=3.5cm,     % Inferiore: 3,5 cm  
    left=4.5cm,       % Sinistro: 4,5 cm
    right=3cm         % Destro: 3 cm
]{geometry}

% INTERLINEA 1,5 RIGHE (REGOLA UNIVERSITÀ)
\usepackage{setspace}
\onehalfspacing

% GESTIONE PARAGRAFI - RIENTRO PRIMA RIGA 1,25 CM
\usepackage{indentfirst}
\setlength{\parindent}{1.25cm}

% NOTE A PIÈ PAGINA - NUMERATE PER CAPITOLO (REGOLA UNIVERSITÀ)
\usepackage{footmisc}
\usepackage{chngcntr}
\counterwithout{footnote}{chapter}
\counterwithin{footnote}{chapter}

% PERSONALIZZAZIONE FORMATO NOTE (PARENTESI TONDE)
\renewcommand{\thefootnote}{(\arabic{footnote})}

% FORMATO NOTE: Arial 10pt, rientro 0,6cm, interlinea singola
\renewcommand{\footnotesize}{\fontsize{10pt}{10pt}\selectfont}
\renewcommand{\footnoterule}{\kern-3pt\hrule width 0.4\columnwidth \kern 2.6pt}

% Configurazione layout note
\setlength{\footnotemargin}{0.6cm}
\setlength{\footnotesep}{0pt}

% GESTIONE INDICE - TITOLI A SINISTRA, NUMERI A DESTRA
\usepackage{tocloft}
\renewcommand{\cftchapfont}{\normalfont\bfseries\fontsize{11pt}{13pt}\selectfont}
\renewcommand{\cftsecfont}{\normalfont\bfseries\fontsize{11pt}{13pt}\selectfont}
\renewcommand{\cftsubsecfont}{\normalfont\bfseries\fontsize{11pt}{13pt}\selectfont}

% TITOLI: 11PT GRASSETTO (REGOLA UNIVERSITÀ)
\usepackage{titlesec}

% Capitoli: 11pt grassetto
\titleformat{\chapter}[display]
    {\normalfont\fontsize{11pt}{13pt}\bfseries}
    {\chaptertitlename\ \thechapter}
    {10pt}
    {}

% Sezioni: 11pt grassetto  
\titleformat{\section}
    {\normalfont\fontsize{11pt}{13pt}\bfseries}
    {\thesection}
    {1em}
    {}

% Sottosezioni: 11pt grassetto
\titleformat{\subsection}
    {\normalfont\fontsize{11pt}{13pt}\bfseries}
    {\thesubsection}
    {1em}
    {}

% Sotto-sottosezioni: 11pt grassetto
\titleformat{\subsubsection}
    {\normalfont\fontsize{11pt}{13pt}\bfseries}
    {\thesubsubsection}
    {1em}
    {}

% HYPERLINKS (devono essere dopo altri pacchetti)
\usepackage[
    colorlinks=false,
    pdfborder={0 0 0},
]{hyperref}

% ==========================================
% COMANDI PERSONALIZZATI PER CITAZIONI
% ==========================================

% Comando per autori in maiuscoletto (REGOLA UNIVERSITÀ)
\newcommand{\autore}[1]{\textsc{#1}}

% Comando per citazione libro secondo formato università
% Uso: \citlibro{F. FORTUNA}{Corporate Governance}{Milano}{F.Angeli}{2001}{16-20}
\newcommand{\citlibro}[6]{%
    \autore{#1}, \textit{#2}, #3, #4, #5, pagg. #6%
}

% Comando per citazione articolo secondo formato università  
% Uso: \citarticolo{G. ZURZOLO}{Collegio sindacale e internal auditors}{Quaderni di finanza}{14}{Consob}{1996}{46}
\newcommand{\citarticolo}[7]{%
    \autore{#1}, #2, in «#3», n. #4, #5, #6, pag. #7%
}

% Ambiente per prefazione in corsivo (REGOLA UNIVERSITÀ)
\newenvironment{prefazione}
    {\chapter*{Prefazione}
     \addcontentsline{toc}{chapter}{Prefazione}
     \itshape}  % Tutto il testo in corsivo
    {\normalfont\clearpage}

% Ambiente per bibliografia manuale secondo regole università
\newenvironment{bibliografiauniv}
    {\chapter*{Bibliografia}
     \addcontentsline{toc}{chapter}{Bibliografia}
     \begin{list}{}{\setlength{\leftmargin}{0pt}\setlength{\itemindent}{-\leftmargin}}
     \raggedright}
    {\end{list}}

% ==========================================
% METADATI DOCUMENTO
% ==========================================
\hypersetup{
    pdftitle={"Dall'Alimentazione alla Cybersecurity: Fondamenti di un'Infrastruttura IT Sicura nella Grande Distribuzione"},
    pdfauthor={Marco Santoro},
    pdfsubject={Tesi di Laurea in Ingegneria Informatica},
    pdfkeywords={ingegneria informatica, tesi, università},
}

% ==========================================
% INIZIO DOCUMENTO
% ==========================================
\begin{document}

% ==========================================
% FRONTESPIZIO
% ==========================================
\begin{titlepage}
    \begin{center}
        \vspace*{2cm}
        
        {\Large \textbf{UNIVERSITÀ DEGLI STUDI "NICCOLO' CUSANO"]}}\\
        \vspace{0.5cm}
        {\large DIPARTIMENTO DI INGEGNERIA}\\
        \vspace{0.5cm}
        {\large CORSO DI LAUREA IN INGEGNERIA INFORMATICA}\\
        
        \vspace{3cm}
        
        {\Huge \textbf{TITOLO DELLA TESI}}\\
        
        \vspace{3cm}
        
        \begin{flushleft}
            \begin{tabular}{ll}
                \textbf{Relatore:} & Prof. [Giovanni Farina] \\
                & \\
                \textbf{Correlatore:} & Dott. [Nome Cognome] \\
                & \\
                \textbf{Candidato:} & [Marco Santoro] \\
                \textbf{Matricola:} & [IN08000291] \\
            \end{tabular}
        \end{flushleft}
        
        \vfill
        
        {\large ANNO ACCADEMICO 2024/2025}
        
    \end{center}
\end{titlepage}

% ==========================================
% INDICE GENERALE (titoli a sinistra, numeri a destra)
% ==========================================
\tableofcontents
\newpage

% ==========================================
% PREFAZIONE (tutto in corsivo come da regole)
% ==========================================
\begin{prefazione}
Questa è una prefazione di esempio scritta completamente in corsivo, come richiesto dalle regole dell'università.

Il template XeLaTeX è stato completamente adattato per rispettare tutte le specifiche del regolamento universitario: font Arial nativo, margini esatti, interlinea 1,5, note numerate per capitolo con parentesi tonde, e formato citazioni conforme.

Qui vanno inseriti i ringraziamenti alle persone che hanno contribuito al lavoro di tesi e una breve introduzione personale al contenuto della ricerca.
\end{prefazione}

% ==========================================
% CAPITOLI
% ==========================================

\chapter{Introduzione}

Questo è un esempio di testo formattato secondo le regole esatte della tua università. Il carattere Arial è caricato nativamente dal sistema operativo, dimensione 12pt normale per il testo.

I paragrafi sono giustificati con rientro della prima riga di 1,25 cm e interlinea 1,5 righe, esattamente come specificato nel regolamento.

Esempio di citazione di libro secondo le regole universitarie: secondo \citlibro{F. Fortuna}{Corporate Governance}{Milano}{F.Angeli}{2001}{16-20}(1), la governance aziendale rappresenta un elemento fondamentale.

Esempio di citazione di articolo: come evidenziato da \citarticolo{G. Zurzolo}{Collegio sindacale e internal auditors}{Quaderni di finanza}{14}{Consob}{1996}{46}(2), il controllo interno è cruciale.

\section{Obiettivi della Ricerca}

I titoli delle sezioni utilizzano Arial 11pt grassetto, come specificato nelle regole. Le note sono numerate progressivamente dall'inizio del capitolo e utilizzano parentesi tonde.

\subsection{Obiettivi Specifici}

Anche i sotto-paragrafi seguono il formato 11pt grassetto.

\section{Struttura della Tesi}

La tesi è organizzata nei seguenti capitoli principali...

\chapter{Stato dell'Arte}

Secondo capitolo di esempio che dimostra la corretta numerazione delle note che ricomincia da (1) per ogni nuovo capitolo.

Esempio di altra citazione: \citlibro{D. E. Knuth}{The Art of Computer Programming}{Boston}{Addison-Wesley}{1997}{1-50}(1).

\section{Tecnologie Esistenti}

Analisi delle tecnologie attualmente disponibili...

\subsection{Confronto Metodologie}

Dettaglio del confronto tra le diverse metodologie...

\chapter{Metodologia Proposta}

Descrizione dettagliata della metodologia sviluppata in questa tesi.

\chapter{Implementazione}

Dettagli tecnici dell'implementazione del sistema proposto.

\chapter{Risultati Sperimentali}

Presentazione e discussione dei risultati ottenuti attraverso la sperimentazione.

\chapter{Conclusioni}

Riepilogo dei contributi della tesi e indicazioni per sviluppi futuri.

% ==========================================
% BIBLIOGRAFIA SECONDO REGOLE UNIVERSITÀ
% ==========================================
\begin{bibliografiauniv}

% Ordinata alfabeticamente per cognome autore
% Formato: COGNOME NOME, Titolo, luogo, editore, anno

\item \autore{Airoldi G.}, Gli assetti istituzionali d'impresa: inerzia, funzioni e leve, in \autore{Airoldi G.-Forestieri G.} (a cura di), Corporate governance. Analisi e prospettive nel caso italiano, Milano, Etas Libri, 1998.

\item \autore{Fortuna F.}, Corporate Governance, Milano, F.Angeli, 2001.

\item \autore{Knuth Donald E.}, The Art of Computer Programming, volume 1, Boston, Addison-Wesley, 1997.

\item \autore{Zurzolo G.}, Collegio sindacale e internal auditors, in «Quaderni di finanza», n. 14, Consob, 1996.

% Esempio di più opere dello stesso autore (ordine cronologico)
% \item \autore{Rossi M.}, Prima opera, Milano, Editore, 1995.
% \item \autore{Rossi M.}, Seconda opera, Roma, Altro Editore, 2000.

\end{bibliografiauniv}

\end{document}