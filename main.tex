% !TEX program = xelatex % Questa riga è importante per VS Code/LaTeX Workshop


\documentclass[12pt, a4paper]{book} % 12pt per la dimensione del testo, a4paper per il formato carta

% --- PACCHETTI FONDAMENTALI ---
\usepackage[utf8]{inputenc} % Già implicito con XeLaTeX/LuaLaTeX ma buona pratica
\usepackage{microtype} % Migliora l'estetica del testo
\usepackage{csquotes} % Per le citazioni
\usepackage{graphicx} % Per includere immagini (es. .pdf, .png, .jpg)
\usepackage{amsmath, amssymb, amsfonts} % Pacchetti AMS per simboli matematici avanzati
\usepackage{booktabs} % Per tabelle di qualità professionale
\usepackage{hyperref} % Per creare collegamenti cliccabili nel PDF
\usepackage[nameinlink, capitalize]{cleveref} % Per riferimenti incrociati intelligenti (es. "Figura 1.1")

% --- IMPOSTAZIONI TIPOGRAFICHE (REGOLE UNIVERSITARIE) ---

% 1. Font: Arial
% Richiede XeLaTeX o LuaLaTeX per funzionare correttamente
\usepackage{fontspec}
\setmainfont{Arial} % Carattere principale
\setsansfont{Arial} % Carattere sans-serif (spesso usato per titoli, ma qui è Arial)
%\setmonofont{Consolas} % Un font monospace per il codice (opzionale)

% 2. Margini: Superiore/Inferiore 3.5cm, Destro 3cm, Sinistro 4.5cm
\usepackage[
    top=3.5cm,
    bottom=3.5cm,
    left=4.5cm,
    right=3cm,
    % showframe % Decommenta per visualizzare i margini durante la fase di debug
]{geometry}

% 3. Interlinea: 1.5 righe
\usepackage{setspace}
\onehalfspacing

% 4. Rientro prima riga: 1.25 cm
% \setlength{\parindent}{1.25cm} % Decommenta se vuoi un rientro esplicito. LaTeX ha un rientro di default.
% \usepackage{indentfirst} % Assicura che il primo paragrafo dopo un titolo abbia un rientro.

% 5. Titoli dei paragrafi: 11, grassetto
% Le classi standard gestiscono le dimensioni dei titoli. Potrebbe non essere esattamente 11pt,
% ma molto vicino. Per un controllo preciso servirebbe una riconfigurazione più complessa.

% --- IMPOSTAZIONI NOTE A PIE' PAGINA (REGOLE UNIVERSITARIE) ---

% Note a piè di pagina: numerate progressivamente all'inizio di ogni capitolo.
% Per questo serve il pacchetto chngcntr, e la classe document 'book' o 'report'.
\usepackage{chngcntr}
\counterwithout{footnote}{chapter} % La numerazione riparte per ogni nuovo \chapter.
% Se usi \section come unità più alta (documentclass{article}), usa:
% \counterwithout{footnote}{section}

% Carattere note: Arial, dimensione 10, stile normale, rientro 0.6cm, interlinea singola
% Questo è più complesso da impostare in modo robusto. Ecco un tentativo:
\usepackage{etoolbox}
\makeatletter
\patchcmd{\@footnotetext}{\normalfont}{\normalfont\fontsize{10}{12}\selectfont}{}{} % Dimensione 10, interlinea 12pt (singola)
% Per il rientro di 0.6cm, si può modificare \footnotesep o usare footmisc
\usepackage[hang, flushmargin]{footmisc} % Per controllare rientro note, default è 0.6cm per hang
\makeatother

% --- GESTIONE BIBLIOGRAFIA ---
% Usiamo biblatex con biber per maggiore flessibilità e compatibilità con i font di sistema
\usepackage[
    backend=biber,
    style=verbose-ibid, % Stile per citazioni complete in nota e "ibid"
    citestyle=numeric, % o authoryear, come preferisci per le citazioni nel testo
    sorting=nyt, % Ordina per nome, anno, titolo (come richiesto per la bibliografia finale)
    url=false, doi=false, eprint=false % Disabilita URL, DOI, ePrint se non richiesto
]{biblatex}
\addbibresource{bibliografia/riferimenti.bib} % Assicurati che il percorso sia corretto

% --- INIZIO DEL DOCUMENTO ---
\begin{document}

% Frontespizio (da adattare al tuo caso)
\begin{titlepage}
    \centering
    \vspace*{\fill}
    {\Huge\bfseries Titolo della Tesi di Laurea\par}
    \vspace{1cm}
    {\Large Nome Cognome Candidato\par}
    \vspace{0.5cm}
    {\large Matricola: XXXXXX\par}
    \vspace{1cm}
    {\Large Relatore: Prof. Nome Cognome Relatore\par}
    \vspace{0.5cm}
    {\large Correlatore: Dott. Nome Cognome Correlatore (se presente)\par}
    \vfill
    {\Large Dipartimento di Informatica\par}
    {\large Università degli Studi di [Nome Università]\par}
    {\large Anno Accademico 20XX/20YY\par}
    \vspace*{\fill}
\end{titlepage}

% Prefazione (se presente)
\chapter*{Prefazione} % Il * nasconde il numero del capitolo nell'indice e nel testo
\addcontentsline{toc}{chapter}{Prefazione} % Aggiunge la prefazione all'indice
\textit{
    Qui va il testo della prefazione, che deve essere in corsivo come da regole.
    Ad esempio, potresti ringraziare il relatore, la tua famiglia, e descrivere brevemente il motivo della tua scelta dell'argomento.
}

% Indice generale
\tableofcontents

\newpage % Per iniziare il primo capitolo su una nuova pagina

% Includi i tuoi capitoli qui.
% Assicurati che i file dei capitoli si trovino nella cartella 'sezioni/'.
% Puoi commentare o decommentare i capitoli mentre li scrivi.

%\input{sezioni/introduzione.tex}
%\input{sezioni/capitolo1.tex} % Es. Revisione della letteratura
%\input{sezioni/capitolo2.tex} % Es. Metodologia
%\input{sezioni/capitolo3.tex} % Es. Risultati e Discussione
%\input{sezioni/conclusioni.tex} % Conclusioni

% Appendici (se presenti)
%\appendix
%\input{appendici/appendiceA.tex}
%\input{appendici/appendiceB.tex}

% Bibliografia finale
\printbibliography[heading=bibintoc, title={Bibliografia}] % Aggiunge la bibliografia all'indice generale

\end{document}