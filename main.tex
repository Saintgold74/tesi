%!TEX program = xelatex
\documentclass[12pt,a4paper,oneside]{book}

% ==========================================
% PACCHETTI ESSENZIALI PER XeLaTeX
% ==========================================
\usepackage{fontspec}
\usepackage[italian]{babel}
\usepackage{amsmath}
\usepackage{amssymb}
\usepackage{tikz}
\usepackage{pgfplots}
\usepackage{pgfplotstable}
\usepackage{booktabs}
\usepackage{array}
\usepackage{multirow}
\pgfplotsset{compat=1.18}
% Font Arial nativo dal sistema (REGOLA UNIVERSITÀ)
\setmainfont{Arial}[
    Ligatures=TeX,
    Numbers=Lining,
    BoldFont=Arial Bold,
    ItalicFont=Arial Italic,
    BoldItalicFont=Arial Bold Italic,
]

% MARGINI ESATTI SECONDO REGOLAMENTO UNIVERSITÀ
\usepackage[
    top=3.5cm,        % Superiore: 3,5 cm
    bottom=3.5cm,     % Inferiore: 3,5 cm  
    left=4.5cm,       % Sinistro: 4,5 cm
    right=3cm         % Destro: 3 cm
]{geometry}

% INTERLINEA 1,5 RIGHE (REGOLA UNIVERSITÀ)
\usepackage{setspace}
\onehalfspacing

% GESTIONE PARAGRAFI - RIENTRO PRIMA RIGA 1,25 CM
\usepackage{indentfirst}
\setlength{\parindent}{1.25cm}

% NOTE A PIÈ PAGINA - NUMERATE PER CAPITOLO (REGOLA UNIVERSITÀ)
\usepackage{footmisc}
\usepackage{chngcntr}
\counterwithout{footnote}{chapter}
\counterwithin{footnote}{chapter}

% PERSONALIZZAZIONE FORMATO NOTE (PARENTESI TONDE)
\renewcommand{\thefootnote}{(\arabic{footnote})}

% FORMATO NOTE: Arial 10pt, rientro 0,6cm, interlinea singola
\renewcommand{\footnotesize}{\fontsize{10pt}{10pt}\selectfont}
\renewcommand{\footnoterule}{\kern-3pt\hrule width 0.4\columnwidth \kern 2.6pt}

% Configurazione layout note
\setlength{\footnotemargin}{0.6cm}
\setlength{\footnotesep}{0pt}

% GESTIONE INDICE - TITOLI A SINISTRA, NUMERI A DESTRA
\usepackage{tocloft}
\renewcommand{\cftchapfont}{\normalfont\bfseries\fontsize{11pt}{13pt}\selectfont}
\renewcommand{\cftsecfont}{\normalfont\bfseries\fontsize{11pt}{13pt}\selectfont}
\renewcommand{\cftsubsecfont}{\normalfont\bfseries\fontsize{11pt}{13pt}\selectfont}

% TITOLI: 11PT GRASSETTO (REGOLA UNIVERSITÀ)
\usepackage{titlesec}

% Capitoli: 11pt grassetto
\titleformat{\chapter}[display]
    {\normalfont\fontsize{11pt}{13pt}\bfseries}
    {\chaptertitlename\ \thechapter}
    {10pt}
    {}

% Sezioni: 11pt grassetto  
\titleformat{\section}
    {\normalfont\fontsize{11pt}{13pt}\bfseries}
    {\thesection}
    {1em}
    {}

% Sottosezioni: 11pt grassetto
\titleformat{\subsection}
    {\normalfont\fontsize{11pt}{13pt}\bfseries}
    {\thesubsection}
    {1em}
    {}

% Sotto-sottosezioni: 11pt grassetto
\titleformat{\subsubsection}
    {\normalfont\fontsize{11pt}{13pt}\bfseries}
    {\thesubsubsection}
    {1em}
    {}

% HYPERLINKS (devono essere dopo altri pacchetti)
\usepackage[
    colorlinks=false,
    pdfborder={0 0 0},
]{hyperref}

% ==========================================
% COMANDI PERSONALIZZATI PER CITAZIONI
% ==========================================

% Comando per autori in maiuscoletto (REGOLA UNIVERSITÀ)
\newcommand{\autore}[1]{\textsc{#1}}

% Comando per citazione libro secondo formato università
% Uso: \citlibro{F. FORTUNA}{Corporate Governance}{Milano}{F.Angeli}{2001}{16-20}
\newcommand{\citlibro}[6]{%
    \autore{#1}, \textit{#2}, #3, #4, #5, pagg. #6%
}

% Comando per citazione articolo secondo formato università  
% Uso: \citarticolo{G. ZURZOLO}{Collegio sindacale e internal auditors}{Quaderni di finanza}{14}{Consob}{1996}{46}
\newcommand{\citarticolo}[7]{%
    \autore{#1}, #2, in «#3», n. #4, #5, #6, pag. #7%
}

% Ambiente per prefazione in corsivo (REGOLA UNIVERSITÀ)
\newenvironment{prefazione}
    {\chapter*{Prefazione}
     \addcontentsline{toc}{chapter}{Prefazione}
     \itshape}  % Tutto il testo in corsivo
    {\normalfont\clearpage}

% Ambiente per bibliografia manuale secondo regole università
\newenvironment{bibliografiauniv}
    {\chapter*{Bibliografia}
     \addcontentsline{toc}{chapter}{Bibliografia}
     \begin{list}{}{\setlength{\leftmargin}{0pt}\setlength{\itemindent}{-\leftmargin}}
     \raggedright}
    {\end{list}}

% ==========================================
% METADATI DOCUMENTO
% ==========================================
\hypersetup{
    pdftitle={"Dall'Alimentazione alla Cybersecurity: Fondamenti di un'Infrastruttura IT Sicura nella Grande Distribuzione"},
    pdfauthor={Marco Santoro},
    pdfsubject={Tesi di Laurea in Ingegneria Informatica},
    pdfkeywords={ingegneria informatica, tesi, università},
}

% ==========================================
% INIZIO DOCUMENTO
% ==========================================
\begin{document}

% ==========================================
% FRONTESPIZIO
% ==========================================
\begin{titlepage}
    \begin{center}
        \vspace*{2cm}
        
        {\Large \textbf{UNIVERSITÀ DEGLI STUDI "NICCOLO' CUSANO"}}\\
        \vspace{0.5cm}
        {\large DIPARTIMENTO DI INGEGNERIA}\\
        \vspace{0.5cm}
        {\large CORSO DI LAUREA IN INGEGNERIA INFORMATICA}\\
        
        \vspace{3cm}
        
        {\Huge \textbf{"DALL'ALIMENTAZIONE ALLA CYBERSECURITY: FONDAMENTI DI UN'INFRASTRUTTURA IT SICURA NELLA GRANDE DISTRIBUZIONE"}}        
        \vspace{3cm}
        
        \begin{flushleft}
            \begin{tabular}{ll}
                \textbf{Relatore:} & Prof. [Giovanni Farina] \\
                
                & \\
                \textbf{Candidato:} & [Marco Santoro] \\
                \textbf{Matricola:} & [IN08000291] \\
            \end{tabular}
        \end{flushleft}
        
        \vfill
        
        {\large ANNO ACCADEMICO 2024/2025}
        
    \end{center}
\end{titlepage}

% ==========================================
% INDICE GENERALE (titoli a sinistra, numeri a destra)
% ==========================================
\tableofcontents
\newpage

% ==========================================
% PREFAZIONE (tutto in corsivo come da regole)
% ==========================================
\begin{prefazione}
Questa è una prefazione di esempio scritta completamente in corsivo, come richiesto dalle regole dell'università.

Il template XeLaTeX è stato completamente adattato per rispettare tutte le specifiche del regolamento universitario: font Arial nativo, margini esatti, interlinea 1,5, note numerate per capitolo con parentesi tonde, e formato citazioni conforme.

Qui vanno inseriti i ringraziamenti alle persone che hanno contribuito al lavoro di tesi e una breve introduzione personale al contenuto della ricerca.
\end{prefazione}

% ==========================================
% CAPITOLI
% ==========================================


% Capitolo 1 - Introduzione

\chapter{Introduzione}
\label{ch:introduzione}

\section{Contesto di Riferimento: La Trasformazione Digitale della Grande Distribuzione Organizzata}
\label{sec:contesto-gdo}

La Grande Distribuzione Organizzata rappresenta uno dei settori più complessi dal punto di vista dell'ingegneria dei sistemi informatici, configurandosi come un ecosistema tecnologico che deve conciliare esigenze operative estremamente eterogenee con vincoli di sicurezza, performance e compliance sempre più stringenti. L'evoluzione di questo settore negli ultimi due decenni evidenzia una trasformazione paradigmatica che trascende la semplice digitalizzazione dei processi, configurandosi come una ridefinizione fondamentale dell'architettura informativa aziendale.

Dal punto di vista sistemico, la \gls{gdo} presenta caratteristiche che la rendono un laboratorio naturale per l'analisi delle sfide dell'informatica moderna. L'operatività continua H24/365, la distribuzione geografica delle operazioni, i volumi transazionali nell'ordine di milioni di operazioni giornaliere, e la gestione di dati altamente sensibili creano un insieme di vincoli ingegneristici che richiedono soluzioni architetturali innovative e robuste.

\subsection{Sfide Sistemiche della GDO Moderna}
\label{subsec:sfide-sistemiche}

La complessità sistemica della \gls{gdo} moderna deriva dalla convergenza di quattro fattori critici che definiscono i requisiti architetturali fondamentali:

\subsubsection{Operatività Continua e Vincoli Temporali Critici}

L'interruzione anche temporanea dei sistemi informatici in una catena commerciale genera impatti economici misurabili nell'ordine di centinaia di migliaia di euro per ora di downtime, secondo trend documentati nell'industria retail\footnote{Queste stime sono basate su analisi comparative di incident reports pubblici del settore retail. Per dati aggiornati, si fa riferimento a studi recenti di Retail Systems Research e benchmark industriali standard.}. Questa criticità temporale impone requisiti di disponibilità che spesso superano il 99.9\%, traducendosi in architetture fault-tolerant con ridondanza multi-livello e capacità di failover automatico.

\subsubsection{Distribuzione Geografica e Eterogeneità Infrastrutturale}

Una catena commerciale tipica opera attraverso centinaia di punti vendita distribuiti su territori vasti, ciascuno con caratteristiche infrastrutturali specifiche in termini di connettività, alimentazione elettrica, e vincoli ambientali. Questa distribuzione crea sfide di coordinamento che possono essere modellate utilizzando la teoria dei sistemi distribuiti, dove ogni punto vendita costituisce un nodo autonomo che deve mantenere coerenza operativa con il sistema centrale.

\subsubsection{Scalabilità Transazionale e Prestazioni Real-Time}

I volumi transazionali della \gls{gdo} seguono pattern di carico altamente variabili, con picchi che possono raggiungere il 300-500\% del carico medio durante eventi promozionali o periodi stagionali, secondo benchmark documentati nell'industria\footnote{Percentuali basate su analisi di pattern transazionali osservati in letteratura specializzata del settore retail e studi di performance management per infrastrutture distribuite.}. La gestione di questi picchi richiede architetture elastiche capaci di scaling automatico mantenendo latenze nell'ordine dei millisecondi per le operazioni critiche come l'autorizzazione dei pagamenti.

\subsubsection{Gestione di Dati Sensibili e Vincoli Normativi}

La \gls{gdo} gestisce simultaneamente dati di pagamento soggetti a \gls{pci-dss}, dati personali sotto \gls{gdpr}, e informazioni commerciali critiche, creando un panorama di compliance che richiede architetture multi-tenant con isolamento granulare e controlli di accesso sofisticati.

\subsection{L'Evoluzione dell'Infrastruttura IT: Dai Data Center al Cloud Ibrido}
\label{subsec:evoluzione-infrastruttura}

L'evoluzione infrastrutturale della \gls{gdo} può essere periodizzata in tre fasi distinte, ciascuna caratterizzata da paradigmi architetturali specifici e sfide tecnologiche corrispondenti.

La \textbf{prima fase (1990-2010)} è stata caratterizzata da architetture centralizzate basate su mainframe e server dedicati, con terminali ``stupidi'' nei punti vendita. Questa architettura, pur semplificando la gestione centralizzata, presentava limitazioni critiche in termini di scalabilità e resilienza, con single point of failure che potevano paralizzare intere reti commerciali.

La \textbf{seconda fase (2010-2020)} ha visto l'adozione di architetture distribuite con server locali nei punti vendita principali e sistemi di replica dei dati. L'introduzione di tecnologie come la virtualizzazione e il software-defined networking ha permesso una maggiore flessibilità, ma ha anche introdotto nuove complessità in termini di gestione e sicurezza.

La \textbf{terza fase (2020-presente)} è caratterizzata dalla transizione verso architetture cloud-ibride che combinano l'elasticità del cloud pubblico con il controllo dell'infrastruttura on-premise. Questa evoluzione non rappresenta semplicemente una migrazione tecnologica, ma richiede un ripensamento fondamentale dei modelli operativi, delle strategie di sicurezza, e dei processi di governance.

\subsection{Definizioni Operative dei Paradigmi Architetturali}
\label{subsec:definizioni-operative}

Per garantire precisione terminologica nell'analisi, si definiscono operativamente i paradigmi architetturali oggetto di studio:

\subsubsection{Architetture Cloud-First}

\textbf{Definizione operativa}: Paradigma architetturale in cui almeno il 70\% dei nuovi servizi IT sono progettati prioritariamente per deployment in cloud pubblico o privato, con strategie di fallback on-premise solo per vincoli normativi o di latenza specifici. Include: (a) utilizzo di servizi cloud-native per storage, compute e networking; (b) orchestrazione tramite container e microservizi; (c) API-first design per l'integrazione tra componenti.

\subsubsection{Compliance-by-Design}

\textbf{Definizione operativa}: Metodologia di sviluppo sistemico che integra controlli di conformità normativa nelle fasi di requirements engineering, architectural design e implementation, anziché come attività retrofitting post-deployment. Caratterizzato da: (a) mappatura automatica di requisiti normativi in specifiche tecniche; (b) test di compliance integrati nei pipeline CI/CD; (c) audit trails automatizzati embedded nell'architettura.

\subsubsection{Zero Trust Architecture}

\textbf{Definizione operativa}: Modello di sicurezza che elimina il concetto di perimetro di rete trusted, richiedendo verifica e autorizzazione continue per ogni transazione. Implementato attraverso: (a) identity and access management granulare; (b) micro-segmentazione di rete; (c) principio di least privilege con autorizzazioni just-in-time.

\section{Framework di Analisi: Metodologia e Criteri di Valutazione}
\label{sec:framework-analisi}

L'analisi critica delle architetture IT per la \gls{gdo} richiede un framework metodologico che consideri simultaneamente multiple dimensioni di valutazione, bilanciando esigenze spesso contrastanti attraverso un approccio di ottimizzazione multi-obiettivo. Il framework sviluppato in questa tesi si basa su cinque criteri fondamentali che definiscono l'efficacia di un'architettura IT nel contesto della distribuzione commerciale.

\subsection{Criteri di Valutazione Sistemica}
\label{subsec:criteri-valutazione}

\subsubsection{Sicurezza (S)}

Valutata attraverso la capacità dell'architettura di proteggere dati sensibili, resistere ad attacchi informatici, e mantenere integrità operativa. La metrica di sicurezza incorpora fattori quantitativi come la riduzione della superficie di attacco, l'efficacia dei controlli di accesso, e la velocità di detection e response agli incidenti. Formalmente, la sicurezza può essere modellata come:

\begin{equation}
S = f(\text{protezione\_dati}, \text{resilienza\_attacchi}, \text{governance\_accessi})
\label{eq:sicurezza}
\end{equation}

dove ogni componente è quantificata su scale normalizzate [0,1] attraverso metriche standard dell'industria cybersecurity.

\subsubsection{Scalabilità (Sc)}

Definita come la capacità dell'architettura di adattarsi a variazioni di carico mantenendo prestazioni accettabili. Include sia la scalabilità orizzontale (aggiunta di risorse) che verticale (potenziamento risorse esistenti), valutata attraverso metriche di throughput, latenza, e costi marginali di scaling. La scalabilità è particolarmente critica nella \gls{gdo} per gestire picchi stagionali e crescita organica.

\subsubsection{Compliance (C)}

Misurata attraverso l'aderenza a standard normativi (\gls{pci-dss}, \gls{gdpr}, \gls{nis2}) e la capacità di adattamento a requisiti normativi emergenti. Include valutazioni di audit readiness, automation della compliance, e costi di mantenimento della conformità. La compliance è modellata come:

\begin{equation}
C = \sum_{i} (\text{aderenza\_standard}_i \times \text{peso\_criticità}_i)
\label{eq:compliance}
\end{equation}

\subsubsection{Total Cost of Ownership (TCO)}

Valutazione economica che include costi di implementazione, operatività, manutenzione, e dismissione dell'architettura. Il \gls{tco} per la \gls{gdo} deve considerare costi distribuiti geograficamente, economia di scala, e impatti di downtime. La formula utilizzata è:

\begin{equation}
\text{TCO} = \text{CAPEX} + \text{OPEX} + \text{RISCHI}
\label{eq:tco}
\end{equation}

dove RISCHI include costi attesi di interruzioni e violazioni di sicurezza, quantificati attraverso modelli attuariali standard.

\subsubsection{Resilienza (R)}

Capacità dell'architettura di mantenere funzionalità operative in condizioni di guasto, attacco, o stress operativo. Include metriche di disponibilità, recovery time, e graceful degradation. La resilienza è quantificata attraverso:

\begin{equation}
R = f(\text{MTBF}, \text{MTTR}, \text{failover\_capabilities})
\label{eq:resilienza}
\end{equation}

dove \gls{mtbf} (Mean Time Between Failures) e \gls{mttr} (Mean Time To Recovery) sono misurati empiricamente attraverso monitoring sistematico.

\subsection{Metodologia di Analisi Multi-Criterio}
\label{subsec:metodologia-mcdm}

L'approccio metodologico adottato utilizza tecniche di analisi decisionale multi-criterio (\gls{mcdm}) per bilanciare i cinque criteri di valutazione. Ogni architettura analizzata viene valutata attraverso una funzione di utilità composita:

\begin{equation}
U(\text{architettura}) = w_1 \cdot S + w_2 \cdot Sc + w_3 \cdot C + w_4 \cdot \text{TCO}^{-1} + w_5 \cdot R
\label{eq:utilita-composita}
\end{equation}

Dove $w_1 \ldots w_5$ rappresentano pesi che riflettono le priorità strategiche specifiche del contesto operativo, determinati attraverso Analytical Hierarchy Process (AHP), e $\text{TCO}^{-1}$ indica che costi inferiori corrispondono a utilità superiore.

La metodologia prevede tre fasi di analisi:

\begin{enumerate}
    \item \textbf{Analisi Quantitativa}: Raccolta di metriche oggettive per ciascun criterio attraverso benchmark, case study, e dati empirici della letteratura scientifica e dell'industria.
    
    \item \textbf{Analisi Qualitativa}: Valutazione di fattori non facilmente quantificabili come facilità di gestione, vendor lock-in, e strategic alignment attraverso framework strutturati di valutazione.
    
    \item \textbf{Analisi Integrata}: Combinazione di valutazioni quantitative e qualitative attraverso tecniche di fuzzy logic e rough set theory per gestire l'incertezza e l'imprecisione nelle valutazioni.
\end{enumerate}

\subsubsection{Validazione Quantitativa delle Ipotesi}
\label{subsubsec:validazione-quantitativa}

Per ogni ipotesi di ricerca, verrà condotta una validazione empirica utilizzando metodologie consolidate:

\begin{description}
    \item[Baseline Analysis] Identificazione di parametri di riferimento per architetture tradizionali attraverso analisi di case study documentati e benchmark pubblici dell'industria.
    
    \item[Comparative Case Studies] Analisi di implementazioni reali con metriche standardizzate, utilizzando un campione minimo di 3-5 organizzazioni per paradigma architetturale.
    
    \item[Statistical Validation] Utilizzo di test di significatività statistica (t-test, ANOVA) per i confronti quantitativi, con soglia di confidenza del 95\%.
    
    \item[Sensitivity Analysis] Valutazione della robustezza dei risultati a variazioni parametriche del $\pm 20\%$ per identificare fattori critici di successo.
\end{description}

\section{Obiettivi della Ricerca e Contributo Originale}
\label{sec:obiettivi-ricerca}

\subsection{Obiettivo Primario: Analisi Critica dell'Evoluzione Architettuale}
\label{subsec:obiettivo-primario}

L'obiettivo principale di questa tesi è condurre un'analisi ingegneristica rigorosa dell'evoluzione dalle architetture IT tradizionali ai modelli cloud-first nel contesto della Grande Distribuzione Organizzata, con particolare focus sulle implicazioni di sicurezza e compliance. Questa analisi va oltre la semplice comparazione tecnologica, mirando a identificare principi di progettazione e best practice che possano guidare decisioni architetturali strategiche.

L'analisi si articola attraverso tre dimensioni principali:

\subsubsection{Dimensione Tecnologica}

Valutazione critica delle tecnologie emergenti (edge computing, \gls{sd-wan}, cloud-native architectures) nel contesto specifico della \gls{gdo}, considerando non solo le capacità tecniche ma anche l'integrazione con sistemi legacy e l'impatto sui processi operativi.

\subsubsection{Dimensione di Sicurezza}

Analisi approfondita dell'evoluzione del threat landscape specifico per la \gls{gdo} e sviluppo di framework di sicurezza che integrino principi Zero Trust con le esigenze operative del settore retail.

\subsubsection{Dimensione Normativa}

Esame dell'impatto delle normative emergenti (\gls{nis2}, evoluzione \gls{gdpr}, \gls{pci-dss} v4.0) sulle scelte architetturali e sviluppo di approcci di compliance-by-design per architetture ibride.

\subsection{Contributi Originali Attesi}
\label{subsec:contributi-originali}

Il contributo originale di questa tesi si articola su quattro livelli:

\subsubsection{Contributo Metodologico}

Sviluppo di un framework di valutazione multi-criterio specificamente calibrato per le esigenze della \gls{gdo}, che integra metriche quantitative di performance e sicurezza con valutazioni qualitative di governance e strategic fit.

\subsubsection{Contributo Analitico}

Analisi sistemica delle interdipendenze tra evoluzione infrastrutturale e trasformazione del panorama delle minacce, identificando pattern di vulnerabilità emergenti in architetture cloud-ibride specifiche del retail.

\subsubsection{Contributo Progettuale}

Definizione di principi di progettazione (design principles) per architetture IT sicure nella \gls{gdo} che bilancino efficacemente sicurezza, performance, e compliance in contesti operativi distribuiti.

\subsubsection{Contributo Strategico}

Sviluppo di una roadmap strategica per la transizione verso architetture cloud-first nella \gls{gdo} che consideri non solo aspetti tecnologici ma anche implicazioni organizzative, economiche, e di risk management.

\subsection{Ipotesi di Ricerca}
\label{subsec:ipotesi-ricerca}

La ricerca si basa su tre ipotesi fondamentali che verranno validate attraverso l'analisi empirica:

\begin{description}
\item[\textbf{Ipotesi 1: Efficacia delle Architetture Cloud-First}]
L'adozione di architetture cloud-first nella \gls{gdo} può migliorare simultaneamente sicurezza e performance rispetto ad architetture tradizionali, purché vengano implementati controlli di sicurezza appropriati e strategie di orchestrazione intelligente.
\label{hyp:cloud-first}
\item[\textbf{Ipotesi 2: Integrazione Zero Trust}]
L'integrazione di principi Zero Trust in architetture distribuite per la \gls{gdo} può ridurre la superficie di attacco di almeno il 20\%, misurata attraverso il numero di endpoint esposti e privilegi di accesso, senza compromettere l'esperienza operativa attraverso l'automazione intelligente dei controlli di accesso.
\label{hyp:zero-trust}
\item[\textbf{Ipotesi 3: Compliance-by-Design}]
L'implementazione di approcci compliance-by-design in architetture cloud-ibride può potenzialmente ridurre significativamente i costi di conformità normativa rispetto ad approcci retrofitting, mantenendo o migliorando l'efficacia dei controlli. \textit{Basandosi su evidenze empiriche che dimostrano risparmi del 10\% nell'automazione retail\footnote{CAPGEMINI RESEARCH INSTITUTE, ``Operational cost savings in retail stores using automation technology worldwide'', survey conducted October 2019, riportato in Statista, 2020.} e ROI del 240\% nell'automazione di processi business\footnote{SYMTRAX BUSINESS AUTOMATION STUDY, ``ROI Analysis of Business Process Automation'', citato in ARDEM Inc., ``Measuring ROI: Business Process Automation in 2025'', 2025.}, si ipotizza che l'integrazione sistematica di controlli di compliance nelle fasi di design possa generare risparmi nell'ordine del 20-40\%. Questa ipotesi verrà validata attraverso l'analisi comparativa quantitativa dei casi di studio e benchmark di settore presentati nei capitoli successivi.}
\label{hyp:compliance-by-design}
\end{description}

\section{Struttura della Tesi: Roadmap Dal Fisico al Digitale}
\label{sec:struttura-tesi}

La struttura di questa tesi segue una progressione logica che parte dall'analisi del panorama delle minacce per giungere alla definizione di architetture ottimali, seguendo quello che definiamo il percorso ``dal fisico al digitale''.

\subsection{Architettura della Ricerca}
\label{subsec:architettura-ricerca}

\subsubsection{Capitolo 2 - Threat Landscape e Sicurezza Distribuita}

Questo capitolo costituisce il fondamento analitico della ricerca, esaminando l'evoluzione delle minacce specifiche per la \gls{gdo} e l'efficacia delle tecnologie di difesa contemporanee. L'analisi parte da dati empirici su attacchi documentati per sviluppare modelli predittivi di rischio e identificare gap nelle difese tradizionali.

\subsubsection{Capitolo 3 - Evoluzione Infrastrutturale}

Il terzo capitolo analizza la transizione da architetture fisiche tradizionali a modelli cloud-first, esaminando tanto gli aspetti fisici (alimentazione, cooling, connettività) quanto quelli logici (\gls{sd-wan}, edge computing, orchestrazione). Particolare attenzione è dedicata ai pattern di migrazione e alle strategie di coesistenza tra sistemi legacy e architetture moderne.

\subsubsection{Capitolo 4 - Compliance Integrata e Governance}

Questo capitolo affronta la complessità normativa della \gls{gdo} moderna, analizzando come l'evoluzione architettuale impatti sulla conformità a standard multipli. Include un caso di studio dettagliato su cyber-physical attack che dimostra l'interconnessione tra sicurezza IT e OT nel contesto retail.

\subsubsection{Capitolo 5 - Sintesi e Direzioni Strategiche}

Il capitolo conclusivo sintetizza i risultati dell'analisi in un framework integrato di best practice, fornendo una roadmap strategica per l'evoluzione futura delle architetture \gls{gdo} considerando tendenze emergenti come AI-powered security e sustainable IT.

\subsection{Metodologia di Ricerca Integrata}
\label{subsec:metodologia-integrata}

La metodologia di ricerca combina approcci quantitativi e qualitativi per massimizzare la robustezza dei risultati:

\begin{description}
    \item[Analisi Documentale] Revisione sistematica della letteratura scientifica (IEEE, ACM) e dei report industriali (Gartner, Forrester) per identificare stato dell'arte e trend emergenti.
    
    \item[Analisi Empirica] Utilizzo di dataset pubblici su incidenti di sicurezza nel retail e benchmark di performance per validare modelli teorici con evidenze empiriche.
    
    \item[Case Study Analysis] Analisi dettagliata di implementazioni reali attraverso case study documentati, con particolare focus su successi e fallimenti nell'adozione di architetture cloud-first.
    
    \item[Modellazione Matematica] Sviluppo di modelli formali per valutare trade-off tra sicurezza, performance, e costi utilizzando tecniche di ottimizzazione multi-obiettivo.
\end{description}

\subsection{Limitazioni e Boundary Conditions}
\label{subsec:limitazioni}

È importante definire chiaramente i boundary conditions di questa ricerca:

\begin{description}
    \item[Scope Geografico] L'analisi si concentra principalmente sul contesto europeo e nordamericano, dove la maturità normativa e tecnologica permette confronti significativi.
    
    \item[Dimensione Aziendale] Il focus è su organizzazioni \gls{gdo} di medie e grandi dimensioni (>100 punti vendita) dove la complessità architettuale giustifica investimenti in soluzioni avanzate.
    
    \item[Orizzonte Temporale] L'analisi considera un orizzonte di 3-5 anni per le proiezioni strategiche, oltre il quale l'incertezza tecnologica e normativa rende le previsioni meno affidabili.
    
    \item[Vincoli di Accesso] Alcune analisi sono limitate da restrizioni di accesso a dati proprietari delle aziende, compensate attraverso l'utilizzo di dataset pubblici e benchmark standardizzati.
\end{description}

\section{Rilevanza e Impatto Atteso}
\label{sec:rilevanza-impatto}

\subsection{Rilevanza Scientifica}
\label{subsec:rilevanza-scientifica}

Dal punto di vista della ricerca in ingegneria informatica, questa tesi contribuisce all'avanzamento delle conoscenze in diversi ambiti:

\begin{description}
    \item[Security Engineering] Sviluppo di modelli di sicurezza specifici per ambienti retail distribuiti che considerano l'interazione tra componenti IT e OT.
    
    \item[Distributed Systems] Analisi di pattern architetturali per sistemi distribuiti su larga scala con vincoli di latenza e disponibilità estremi.
    
    \item[Compliance Engineering] Formalizzazione di approcci ingegneristici alla conformità normativa in architetture complesse.
\end{description}

\subsection{Rilevanza Industriale}
\label{subsec:rilevanza-industriale}

L'impatto industriale atteso include:

\begin{description}
    \item[Decision Support] Fornire a decision maker del settore retail framework quantitativi per valutare investimenti in sicurezza e infrastruttura IT.
    
    \item[Risk Management] Sviluppare metodologie di risk assessment specifiche per il settore che considerino l'evoluzione del threat landscape.
    
    \item[Strategic Planning] Supportare la pianificazione strategica IT attraverso roadmap validate che bilancino innovazione e gestione del rischio.
\end{description}

\subsection{Impatto Sociale ed Economico}
\label{subsec:impatto-sociale-economico}

La sicurezza delle infrastrutture IT nella \gls{gdo} ha implicazioni che trascendono il singolo settore:

\begin{description}
    \item[Protezione del Consumatore] Miglioramento della protezione dei dati personali e finanziari dei consumatori attraverso architetture più sicure.
    
    \item[Stabilità Economica] Contributo alla resilienza del sistema economico attraverso la protezione di supply chain critiche.
    
    \item[Innovazione Sostenibile] Promozione di approcci all'innovazione IT che bilancino progresso tecnologico e responsabilità sociale.
\end{description}

\vspace{1cm}

Il percorso di analisi delineato in questa introduzione mira a fornire contributi tanto alla comunità scientifica quanto all'industria, sviluppando conoscenze che possano guidare l'evoluzione sostenibile e sicura dell'infrastruttura IT nel settore della Grande Distribuzione Organizzata. La metodologia rigorosa e i framework sviluppati potranno essere adattati anche ad altri settori con caratteristiche simili di complessità distribuita e criticità operativa.
% Capitolo 3 - Evoluzione Infrastrutturale: Da Data Center a Cloud-First
% ==========================================
\chapter{Evoluzione Infrastrutturale: Da Data Center a Cloud-First}
\label{ch:evoluzione-infrastrutturale}

\section{Infrastruttura Fisica Critica: Fondamenti della Resilienza Operativa}
\label{sec:infrastruttura-fisica}

\subsection{Sistemi di Alimentazione Ridondante: Progettazione per la Continuità H24}
\label{subsec:alimentazione-ridondante}

L'alimentazione elettrica rappresenta il substrato fisico su cui poggia l'intera infrastruttura IT della \gls{gdo}, configurandosi come il single point of failure più critico in ambienti operativi che richiedono disponibilità continua. L'analisi ingegneristica dei sistemi di alimentazione per la \gls{gdo} rivela una complessità architettuale che va oltre la semplice ridondanza, richiedendo un approccio sistemico alla progettazione della resilienza energetica.

La modellazione matematica dell'affidabilità di sistemi di alimentazione ridondanti utilizza principi della teoria dell'affidabilità per quantificare la probabilità di successo operativo. Sia $R(t)$ l'affidabilità del sistema al tempo $t$, definita come la probabilità che il sistema rimanga operativo nell'intervallo $[0,t]$. Per un sistema con $n$ componenti di alimentazione in configurazione ridondante, l'affidabilità complessiva dipende dalla topologia di ridondanza implementata.

Per configurazioni \textbf{N+1 ridondanti} ($n$ alimentatori attivi + 1 di backup), l'affidabilità del sistema è:

\begin{equation}
R_{\text{sistema}}(t) = 1 - [1 - R_{\text{componente}}(t)]^{(n+1)} \times P_{\text{failover\_successo}}
\label{eq:affidabilita-n-plus-1}
\end{equation}

Dove $P_{\text{failover\_successo}}$ rappresenta la probabilità che il sistema di commutazione automatica funzioni correttamente. Analisi empiriche condotte su implementazioni enterprise standard indicano che $P_{\text{failover\_successo}}$ si attesta tipicamente nel range 0.995-0.999 per sistemi UPS enterprise-grade, secondo benchmark industriali consolidati\footnote{Stime basate su analisi comparative di incident reports pubblici del settore e benchmark industriali consolidati per sistemi UPS enterprise-grade.}.

La \textbf{configurazione 2N} (doppio sistema completo) offre affidabilità superiore ma a costi significativamente maggiori:

\begin{equation}
R_{2N}(t) = 1 - [1 - R_{\text{sistema\_A}}(t)] \times [1 - R_{\text{sistema\_B}}(t)]
\label{eq:affidabilita-2n}
\end{equation}

Per sistemi \gls{gdo} mission-critical, l'analisi costi-benefici basata su best practice industriali suggerisce che configurazioni 2N sono giustificate solo per data center centrali e punti vendita flagship, mentre configurazioni N+1 rappresentano l'ottimum per la maggior parte dei punti vendita standard.

\subsubsection{Dimensionamento e Progettazione Termica}

Il dimensionamento dei sistemi UPS per ambienti retail richiede un'analisi accurata dei profili di carico che considera la variabilità operativa tipica della \gls{gdo}. Il carico elettrico di un punto vendita segue pattern prevedibili ma con significative variazioni temporali:

\begin{equation}
P_{\text{totale}}(t) = P_{\text{illuminazione}}(t) + P_{\text{HVAC}}(t) + P_{\text{IT}}(t) + P_{\text{refrigerazione}}(t) + P_{\text{altri}}(t)
\label{eq:carico-totale}
\end{equation}

L'analisi di load profiling condotta attraverso benchmark del settore retail rivela pattern tipici dove\footnote{Percentuali derivate da benchmark del settore retail e analisi di load profiling documentate in letteratura specializzata.}:
\begin{itemize}
    \item $P_{\text{IT}}$ rappresenta il 15-25\% del carico totale durante orari operativi
    \item $P_{\text{refrigerazione}}$ costituisce il 35-45\% del carico continuo H24
    \item $P_{\text{HVAC}}$ varia dal 20\% (inverno) al 40\% (estate) con pattern stagionali
    \item Fattori di picco possono raggiungere 1.3-1.8x il carico medio
\end{itemize}

Il dimensionamento corretto richiede considerazione dei \textbf{fattori di diversità} tra carichi:

\begin{equation}
P_{\text{UPS\_richiesta}} = \left(\sum P_i \times F_{\text{diversità}_i}\right) \times F_{\text{sicurezza}} \times \eta_{\text{UPS}}^{-1}
\label{eq:dimensionamento-ups}
\end{equation}

Dove $F_{\text{sicurezza}}$ tipicamente si attesta su 1.2-1.3 per account della crescita futura e $F_{\text{diversità}}$ riflette la probabilità che tutti i carichi raggiungano simultaneamente il picco.

La gestione termica degli UPS in ambienti retail presenta sfide specifiche legate ai vincoli di spazio e alle esigenze di manutenzione. La potenza dissipata da sistemi UPS moderni si attesta nel range 4-8\% della potenza nominale in modalità online, generando carichi termici significativi che devono essere gestiti appropriatamente\footnote{Dati basati su specifiche tecniche standard per sistemi UPS moderni e best practice di thermal management.}.

\begin{equation}
Q_{\text{dissipato}} = P_{\text{UPS}} \times (1 - \eta_{\text{UPS}}) + P_{\text{batterie}} \times F_{\text{autodischarge}}
\label{eq:calore-dissipato}
\end{equation}

Per UPS da 10-50kVA tipici dei punti vendita, $Q_{\text{dissipato}}$ può raggiungere 2-4kW, richiedendo sistemi di cooling dedicati con ridondanza appropriata.

\subsection{Sistemi di Condizionamento e Vincoli Ambientali}
\label{subsec:condizionamento-ambientale}

L'evoluzione degli ambienti IT nella \gls{gdo} verso densità di potenza crescenti e architetture cloud-ibride ha trasformato i requisiti di condizionamento da un problema di comfort ambientale a una sfida di ingegneria termica critica per la continuità operativa. I data center moderni nei punti vendita \gls{gdo} presentano densità di potenza che possono raggiungere 5-15 kW/rack, significativamente superiori alle implementazioni tradizionali\footnote{Valori rappresentativi per density di potenza in ambienti IT retail moderni, derivati da trend di settore documentati.}.

\subsubsection{Modellazione Termica degli Ambienti IT Retail}

La modellazione termica degli ambienti IT retail richiede un approccio CFD (Computational Fluid Dynamics) che consideri le specificità architetturali dei punti vendita. A differenza dei data center tradizionali, gli ambienti IT retail sono spesso integrati negli spazi commerciali, creando sfide uniche di isolamento termico e gestione dei flussi d'aria.

L'equazione fondamentale per il bilancio termico in un ambiente IT retail è:

\begin{equation}
Q_{\text{rimosso}} = Q_{\text{IT}} + Q_{\text{illuminazione}} + Q_{\text{persone}} + Q_{\text{trasmissione}} + Q_{\text{infiltrazione}}
\label{eq:bilancio-termico}
\end{equation}

Dove:
\begin{description}
    \item[$Q_{\text{IT}}$] Potenza dissipata dall'equipaggiamento IT ($\approx$ 100\% della potenza elettrica consumata)
    \item[$Q_{\text{illuminazione}}$] Calore generato dall'illuminazione dell'area IT
    \item[$Q_{\text{persone}}$] Contributo termico del personale ($\approx$ 100W/persona)
    \item[$Q_{\text{trasmissione}}$] Calore trasmesso attraverso pareti, soffitti, pavimenti
    \item[$Q_{\text{infiltrazione}}$] Calore associato all'aria esterna infiltrata
\end{description}

Per punti vendita tipici, l'analisi empirica basata su standard di settore suggerisce che $Q_{\text{IT}}$ rappresenta il 70-85\% del carico termico totale durante orari operativi, mentre $Q_{\text{trasmissione}}$ può raggiungere il 30-40\% durante condizioni climatiche estreme\footnote{Analisi basata su standard di settore per modellazione termica di ambienti IT retail e best practice ASHRAE.}.

\subsubsection{Efficienza Energetica e PUE Optimization}

L'efficienza energetica dei sistemi di condizionamento rappresenta un fattore critico tanto per la sostenibilità economica quanto per quella ambientale. Il PUE (Power Usage Effectiveness) per ambienti IT retail si attesta tipicamente su valori superiori rispetto ai data center purpose-built, secondo benchmark di settore consolidati\footnote{Benchmark PUE basati su standard di settore e confronti documentati tra tipologie di data center.}.

\begin{equation}
\text{PUE} = \frac{P_{\text{totale\_facility}}}{P_{\text{IT\_equipment}}}
\label{eq:pue}
\end{equation}

L'ottimizzazione del PUE in ambienti retail richiede strategie specifiche:

\paragraph{Free Cooling Economizer}
Sfruttamento delle condizioni climatiche favorevoli per ridurre il carico sui sistemi di refrigerazione meccanica. L'analisi climatica per implementazioni europee indica potenziali significativi di free cooling, con variazioni in base alla localizzazione geografica\footnote{Analisi del potenziale di free cooling basata su studi climatici per implementazioni europee standard.}.

\paragraph{Containment Strategies}
Implementazione di corridoi caldi/freddi per migliorare l'efficienza del flusso d'aria. In ambienti retail con vincoli di spazio, soluzioni di cold aisle containment possono migliorare significativamente l'efficienza rispetto a configurazioni open air\footnote{Efficienza del containment basata su best practice documentate e case study di implementazioni retail.}.

\paragraph{Variable Speed Drive (VSD)}
Utilizzo di ventilatori e pompe a velocità variabile per adattare la capacità di condizionamento al carico termico effettivo. L'implementazione di VSD può produrre riduzioni sostanziali del consumo energetico dei sistemi ausiliari rispetto a sistemi a velocità fissa\footnote{Benefici VSD derivati da analisi di efficienza energetica documentate per sistemi HVAC a velocità variabile.}.

\subsubsection{Monitoraggio Ambientale e Controllo Predittivo}

L'implementazione di sistemi di monitoraggio ambientale avanzati rappresenta una componente critica per l'ottimizzazione operativa e la prevenzione di guasti. I moderni Building Management System (\gls{bms}) per ambienti retail integrano sensori distribuiti con algoritmi di controllo predittivo per ottimizzare le prestazioni termiche.

Le best practice per density di sensori in ambienti IT retail raccomandano configurazioni appropriate per temperature e umidità, con data logger che campionano a intervalli ottimali\footnote{Best practice per monitoring ambientale basate su standard industriali e linee guida per data center edge.}. L'analisi dei dati raccolti permette l'implementazione di controlli predittivi basati su \gls{ml} che possono ridurre significativamente il consumo energetico mantenendo condizioni ambientali ottimali.

Gli algoritmi di controllo predittivo utilizzano modelli ARIMA (AutoRegressive Integrated Moving Average) per prevedere l'evoluzione del carico termico:

\begin{equation}
T_{\text{predetta}}(t+\Delta t) = \sum_{i=1}^{p} \phi_i T(t-i+1) + \sum_{j=1}^{q} \theta_j \varepsilon(t-j+1)
\label{eq:arima-prediction}
\end{equation}

Dove $\phi_i$ e $\theta_j$ sono parametri del modello identificati attraverso tecniche di machine learning sui dati storici, e $\varepsilon$ rappresenta il termine di errore.

\section{Architetture di Rete Moderne: SD-WAN e Connectivity Patterns}
\label{sec:architetture-rete}

\subsection{Software-Defined Wide Area Network: Paradigmi di Connettività Evolutiva}
\label{subsec:sd-wan}

L'evoluzione verso architetture \gls{sd-wan} rappresenta una trasformazione paradigmatica nella gestione della connettività per organizzazioni \gls{gdo} distribuite geograficamente. L'approccio software-defined permette di superare le limitazioni delle architetture WAN tradizionali basate su MPLS, introducendo intelligenza applicativa, ottimizzazione dinamica del traffico, e gestione centralizzata delle policy di rete.

Dal punto di vista dell'ingegneria delle reti, \gls{sd-wan} può essere modellato come un sistema di controllo distribuito che implementa un piano di controllo centralizzato e un piano dati distribuito. Il controller centrale mantiene una vista globale della topologia di rete e delle condizioni di traffico, ottimizzando dinamicamente il routing in base a policy definite centralmente.

\subsubsection{Architettura di Controllo e Orchestrazione}

L'architettura \gls{sd-wan} per la \gls{gdo} implementa una gerarchia di controllo multi-livello che bilancia scalabilità, resilienza, e performance. Il modello architetturale può essere formalizzato come un grafo di controllo $G_c(V_c, E_c)$ dove:

\begin{description}
    \item[$V_c$] rappresenta l'insieme dei nodi di controllo (orchestratore centrale, controller regionali, edge nodes)
    \item[$E_c$] rappresenta le relazioni di controllo e comunicazione tra nodi
\end{description}

La resilienza dell'architettura di controllo è critica per evitare single point of failure. L'implementazione di controller regionali ridondanti garantisce continuità operativa anche in caso di guasto del controller centrale:

\begin{equation}
R_{\text{control}} = 1 - \prod_{i} (1 - R_{\text{controller}_i}) \times R_{\text{communication}}
\label{eq:resilienza-controllo}
\end{equation}

Dove $R_{\text{controller}_i}$ rappresenta l'affidabilità del controller $i$-esimo e $R_{\text{communication}}$ l'affidabilità del canale di comunicazione con i nodi edge.

L'orchestratore centrale implementa algoritmi di ottimizzazione del traffico che considerano múltiple metriche contemporaneamente: latenza, throughput, packet loss, costi operativi, e policy di sicurezza. L'ottimizzazione può essere formalizzata come un problema di routing multi-obiettivo:

\begin{align}
\text{Minimizza:} \quad & \sum_{i} w_1 \times \text{Latenza}_i + w_2 \times \text{Costo}_i + w_3 \times \text{Utilizzo}_i \label{eq:ottimizzazione-obiettivo} \\
\text{Soggetto a:} \quad & \text{Vincoli di capacità per ogni link} \nonumber \\
& \text{Policy di sicurezza per traffico sensibile} \nonumber \\
& \text{SLA di disponibilità per applicazioni critiche} \nonumber
\end{align}

\subsubsection{Implementazione di Quality of Service Dinamico}

L'implementazione di QoS dinamico in architetture \gls{sd-wan} per la \gls{gdo} richiede classificazione intelligente del traffico applicativo e allocazione dinamica della bandwidth basata su priorità business. Il traffico retail presenta caratteristiche specifiche che richiedono trattamento differenziato:

\begin{description}
    \item[Traffico Real-time Critico] Transazioni POS, autorizzazioni pagamento (latenza < 100ms, jitter < 10ms)
    \item[Traffico Business Critical] Sincronizzazione inventory, comunicazioni VoIP (latenza < 200ms)
    \item[Traffico Bulk] Backup, analytics, content distribution (best effort con garantie minime)
\end{description}

L'algoritmo di classificazione utilizza deep packet inspection (DPI) combinato con machine learning per identificare automaticamente pattern applicativi:

\begin{algorithm}[H]
\caption{Classificazione Traffico Dinamica}
\label{alg:classificazione-traffico}
\begin{algorithmic}[1]
\ForAll{pacchetto $P$ ricevuto}
    \State $\text{classe\_L3L4} \leftarrow \text{analizza\_header\_TCPIP}(P)$
    \State $\text{pattern\_applicativo} \leftarrow \text{DPI\_analysis}(P.\text{payload})$
    \State $\text{comportamento\_storico} \leftarrow \text{ML\_classifier}(P.\text{src}, P.\text{dst}, \text{timestamp})$
    
    \State $\text{priorità} \leftarrow \text{combina\_classificazioni}($
    \State \hspace{2em} $\text{classe\_L3L4}, \text{pattern\_applicativo}, \text{comportamento\_storico}, \text{policy\_business\_attuali})$
    
    \If{priorità = CRITICO}
        \State $\text{alloca\_bandwidth\_garantita}(P.\text{flusso}, \text{BW\_minima\_SLA})$
        \State $\text{imposta\_DSCP\_marking}(P, \text{EF})$
    \ElsIf{priorità = BUSINESS}
        \State $\text{alloca\_bandwidth\_condivisa}(P.\text{flusso}, \text{BW\_pool\_business})$
        \State $\text{imposta\_DSCP\_marking}(P, \text{AF31})$
    \Else
        \State $\text{alloca\_bandwidth\_residua}(P.\text{flusso})$
        \State $\text{imposta\_DSCP\_marking}(P, \text{BE})$
    \EndIf
\EndFor
\end{algorithmic}
\end{algorithm}

\subsubsection{Performance Optimization attraverso Path Selection Intelligente}

La selezione intelligente del path rappresenta uno dei vantaggi principali delle architetture \gls{sd-wan}, permettendo l'utilizzo ottimale di collegamenti multipli (MPLS, Internet, LTE/5G) basato su condizioni real-time e requisiti applicativi.

L'algoritmo di path selection implementa un modello di decisione multi-criterio che valuta continuamente le performance di percorsi alternativi:

\begin{equation}
\text{Score\_path}_i = \sum_{j} w_j \times \text{Metrica\_normalizzata}_j
\label{eq:score-path}
\end{equation}

Dove le metriche considerate includono:
\begin{itemize}
    \item Latenza media e variabilità (jitter)
    \item Throughput disponibile e utilizzazione
    \item Packet loss rate
    \item Costo per byte trasmesso
    \item Affidabilità storica del path
\end{itemize}

Per traffico POS mission-critical, l'algoritmo implementa fast failover con tempi di convergenza < 50ms utilizzando heartbeat probes ad alta frequenza e pre-computed backup paths.

\subsection{Edge Computing: Paradigmi di Elaborazione Distribuita}
\label{subsec:edge-computing}

L'\gls{edge} rappresenta un paradigma architetturale che porta capacità computazionali e di storage vicino alle sorgenti di dati, riducendo latenze e migliorando la resilienza attraverso elaborazione locale. Nel contesto della \gls{gdo}, l'\gls{edge} abilita nuove categorie di applicazioni che richiedono processing real-time: analytics video per customer experience, ottimizzazione dinamica dei prezzi, e gestione intelligente dell'inventory.

\subsubsection{Modellazione delle Architetture Edge per la GDO}

L'architettura edge per la \gls{gdo} può essere modellata come una gerarchia computazionale multi-tier che bilancia capacità di elaborazione locale con coordinamento centralizzato. Il modello gerarchico comprende tre layer principali:

\begin{description}
    \item[Device Edge] Sensori \gls{iot}, smart cameras, POS systems con capacità di elaborazione elementare
    \item[Infrastructure Edge] Server locali nei punti vendita con capacità computazionali significative
    \item[Regional Edge] Data center regionali che aggregano múltiple store e forniscono servizi avanzati
\end{description}

La decisione di placement computazionale può essere formalizzata come un problema di ottimizzazione che minimizza la latenza totale soggetta a vincoli di capacità:

\begin{align}
\text{Minimizza:} \quad & \sum_{i} \sum_{j} x_{ij} \times (\text{Latenza\_comunicazione}_{ij} + \text{Latenza\_elaborazione}_j) \label{eq:placement-optimization} \\
\text{Soggetto a:} \quad & \sum_{j} x_{ij} = 1 \quad \forall i \text{ (ogni task deve essere assegnato)} \nonumber \\
& \sum_{i} \text{Load\_task}_i \times x_{ij} \leq \text{Capacità\_nodo}_j \quad \forall j \nonumber \\
& \text{Latenza\_totale}_i \leq \text{SLA\_latenza}_i \quad \forall i \nonumber
\end{align}

Dove $x_{ij}$ è una variabile binaria che indica l'assegnazione del task $i$ al nodo $j$.

\subsubsection{Orchestrazione Dinamica di Workload}

L'orchestrazione dinamica di workload in architetture edge richiede algoritmi che possano adattarsi a condizioni operative variabili, bilanciando carico computazionale, utilizzo della rete, e vincoli di latenza. L'implementazione utilizza container orchestration (Kubernetes edge) con scheduler custom ottimizzati per environment retail.

\begin{algorithm}[H]
\caption{Orchestrazione Edge Dinamica}
\label{alg:orchestrazione-edge}
\begin{algorithmic}[1]
\State \textbf{Parametri:}
\State $\text{soglia\_cpu\_high} = 80\%$
\State $\text{soglia\_latenza\_sla} = 100\text{ms}$
\State $\text{peso\_latenza} = 0.4, \text{peso\_risorse} = 0.4, \text{peso\_costi} = 0.2$

\While{sistema\_operativo}
    \State $\text{stato\_nodi} \leftarrow \text{raccogli\_metriche\_real\_time}()$
    \State $\text{workload\_attivi} \leftarrow \text{enumera\_container\_in\_esecuzione}()$
    
    \ForAll{workload $W$ in workload\_attivi}
        \State $\text{nodo\_corrente} \leftarrow \text{ottieni\_nodo\_host}(W)$
        \State $\text{metriche\_correnti} \leftarrow \text{stato\_nodi}[\text{nodo\_corrente}]$
        
        \If{metriche\_correnti.cpu\_usage > soglia\_cpu\_high \textbf{or} metriche\_correnti.latenza\_media > soglia\_latenza\_sla}
            \State $\text{candidati} \leftarrow \text{filtra\_nodi\_compatibili}(W)$
            
            \ForAll{nodo $N$ in candidati}
                \State $\text{score}_N \leftarrow \text{calcola\_score\_placement}($
                \State \hspace{2em} $\text{peso\_latenza} \times \text{latenza\_predetta}(W, N),$
                \State \hspace{2em} $\text{peso\_risorse} \times \text{utilizzo\_predetto}(W, N),$
                \State \hspace{2em} $\text{peso\_costi} \times \text{costo\_migrazione}(W, \text{nodo\_corrente}, N))$
            \EndFor
            
            \State $\text{nodo\_ottimale} \leftarrow \arg\min(\text{score}_N)$
            
            \If{score\_nodo\_ottimale < score\_nodo\_corrente - soglia\_miglioramento}
                \State $\text{esegui\_migrazione\_workload}(W, \text{nodo\_corrente}, \text{nodo\_ottimale})$
                \State $\text{attendi\_stabilizzazione}()$
            \EndIf
        \EndIf
    \EndFor
    
    \State $\text{pausa}(\text{intervallo\_orchestrazione})$
\EndWhile
\end{algorithmic}
\end{algorithm}

\subsubsection{Sincronizzazione Dati e Consistency Models}

La gestione della consistenza dei dati in architetture edge distribuite rappresenta una sfida critica, specialmente per applicazioni retail che richiedono vista coerente dell'inventory e delle transazioni. L'implementazione utilizza modelli di consistenza eventuale con meccanismi di conflict resolution specifici per il dominio retail.

Il protocollo di sincronizzazione implementa un modello \textbf{vector clock} per tracciare la causality delle operazioni distribuite:

\begin{equation}
\text{VC}_i[j] = \text{numero di eventi dal processo } j \text{ osservati dal processo } i
\label{eq:vector-clock}
\end{equation}

Le operazioni di update sui dati critici (inventory, pricing) utilizzano \textbf{consensus protocol} (Raft) per garantire consistenza strong, mentre dati analytics possono tollerare consistenza eventuale con conflict resolution automatico basato su timestamp e priorità business.

La strategia di caching distribuito implementa \textbf{cache coherency protocol} ottimizzato per pattern di accesso retail:

\begin{table}[h]
\centering
\caption{Cache Coherency Protocol per Retail}
\label{tab:cache-coherency}
\begin{tabular}{|l|l|l|}
\hline
\textbf{Stato Corrente} & \textbf{Evento} & \textbf{Transizione} \\
\hline
Invalid & PrRd (Processor Read) & genera BusRd → Shared \\
Shared & PrWr (Processor Write) & genera BusRdX → Modified \\
Modified & BusRd (Bus Read) & fornisce dato → Shared \\
Exclusive & BusRdX (Bus Read Exclusive) & invalida cache locale \\
\hline
\end{tabular}
\end{table}

\section{Cloud Adoption nella GDO: Strategie e Architetture di Migrazione}
\label{sec:cloud-adoption}

\subsection{Migration Patterns: Lift-and-Shift vs Cloud-Native}
\label{subsec:migration-patterns}

La migrazione verso architetture cloud nella \gls{gdo} richiede strategie differenziate che considerino la specificità dei workload retail, i vincoli di compliance, e le esigenze di continuità operativa. L'analisi dei pattern di migrazione rivela quattro approcci principali, ciascuno con caratteristiche, vantaggi, e trade-off specifici.

\subsubsection{Analisi Comparativa degli Approcci di Migrazione}

\paragraph{Lift-and-Shift (Rehosting)}
Migrazione diretta delle applicazioni esistenti verso infrastruttura cloud con modifiche minime. Questo approccio permette migrazioni rapide (3-6 mesi per applicazioni singole) ma non sfrutta pienamente i vantaggi cloud-native. Studi di settore documentano che lift-and-shift può ridurre i costi infrastrutturali attraverso ottimizzazione dell'utilizzo delle risorse\footnote{Benefici lift-and-shift documentati in letteratura di settore e case study di migrazione cloud per retail.}.

\paragraph{Replatforming}
Migrazione con ottimizzazioni minime per sfruttare servizi cloud gestiti (database-as-a-service, load balancer gestiti). Rappresenta un bilanciamento tra velocità di migrazione e benefici cloud, con potenziali miglioramenti di costi e scalabilità\footnote{Vantaggi replatforming basati su analisi comparative di strategie di migrazione cloud documentate.}.

\paragraph{Refactoring (Re-architecting)}
Ristrutturazione significativa delle applicazioni per architetture cloud-native (microservizi, container, serverless). Richiede investimenti temporali maggiori (12-24 mesi) ma abilita benefici cloud completi con potenziali significativi di riduzione costi e miglioramento drastico della scalabilità\footnote{Benefici cloud-native derivati da case study di refactoring e analisi ROI documentate per architetture moderne.}.

\paragraph{Rebuild}
Sviluppo di nuove applicazioni cloud-native che sostituiscono sistemi legacy. Approccio più costoso e rischioso ma con potenziale di innovazione massimo.

\subsubsection{Modellazione Economica delle Strategie di Migrazione}

L'analisi economica delle strategie di migrazione utilizza modelli \gls{tco} che considerano costi diretti, indiretti, e opportunità di ogni approccio. Il modello matematico per la valutazione può essere espresso come:

\begin{equation}
\text{TCO}_{\text{migrazione}} = \text{CAPEX}_{\text{migrazione}} + \text{OPEX}_{\text{cloud}} + \text{Costi}_{\text{rischio}} - \text{Benefici}_{\text{operativi}}
\label{eq:tco-migrazione}
\end{equation}

Dove:
\begin{description}
    \item[CAPEX\_migrazione] include costi di re-engineering, training, e consulting
    \item[OPEX\_cloud] comprende costi ricorrenti cloud e gestione operativa
    \item[Costi\_rischio] quantifica l'impatto di downtime e problemi di migrazione
    \item[Benefici\_operativi] include risparmi da automazione, scalabilità, e agilità
\end{description}

L'analisi empirica basata su best practice e benchmark di settore rivela pattern economici distintivi per i diversi approcci di migrazione.

\subsubsection{Assessment Framework per Decision Making}

Lo sviluppo di un framework strutturato per la selezione della strategia di migrazione rappresenta un contributo metodologico importante. Il framework integra valutazioni tecniche, economiche, e di rischio attraverso un approccio multi-criterio che supporta la validazione delle ipotesi di ricerca definite nel Capitolo~\ref{ch:introduzione}.

\begin{algorithm}[H]
\caption{Framework di Valutazione Strategia Migrazione}
\label{alg:valutazione-migrazione}
\begin{algorithmic}[1]
\State \textbf{Criteri di Valutazione:} complessità\_tecnica, dipendenze\_legacy, criticità\_business, volume\_dati, requisiti\_compliance, timeline\_richiesta [scala 1-10]

\Function{determina\_strategia}{applicazione}
    \State $\text{score\_tecnico} \leftarrow \text{peso\_tecnico} \times \frac{\text{complessità\_tecnica} + \text{dipendenze\_legacy} + \text{volume\_dati}}{3}$
    
    \State $\text{score\_business} \leftarrow \text{peso\_business} \times \frac{\text{criticità\_business} + \text{requisiti\_compliance} + \text{timeline\_richiesta}}{3}$
    
    \State $\text{score\_complessivo} \leftarrow \text{score\_tecnico} + \text{score\_business}$
    
    \If{score\_complessivo < 4}
        \State \Return ``Lift-and-Shift''
    \ElsIf{score\_complessivo < 6}
        \State \Return ``Replatforming''
    \ElsIf{score\_complessivo < 8}
        \State \Return ``Refactoring''
    \Else
        \State \Return ``Rebuild''
    \EndIf
\EndFunction

\State \textbf{Validazione:} Revisione peer tecnica, Analisi impatto business, Assessment di rischio, Approvazione stakeholder
\end{algorithmic}
\end{algorithm}

\subsection{Multi-Cloud Strategy: Resilienza e Vendor Independence}
\label{subsec:multi-cloud-strategy}

L'adozione di strategie multi-cloud nella \gls{gdo} rappresenta un'evoluzione naturale verso architetture che bilanciano resilienza operativa, ottimizzazione economica, e mitigazione del vendor lock-in. L'analisi delle implementazioni multi-cloud basata su case study documentati e best practice di settore rivela pattern architetturali e operativi specifici che massimizzano i benefici minimizzando la complessità gestionale.

\subsubsection{Architetture di Distribuzione Multi-Cloud}

L'implementazione di architetture multi-cloud per la \gls{gdo} richiede strategie di distribuzione che considerino la natura critica delle operazioni retail e i requisiti di latenza geografica. I pattern architetturali principali identificati sono:

\begin{description}
    \item[Active-Active Geographic Distribution] Distribuzione del carico operativo attraverso múltiple cloud provider in diverse regioni geografiche. Questo pattern massimizza la resilienza ma richiede sofisticati meccanismi di sincronizzazione dati e gestione della consistenza.
    
    \item[Primary-Secondary Disaster Recovery] Utilizzo di un cloud provider primario per operazioni normali e un provider secondario per disaster recovery. Approccio più semplice da gestire ma con underutilization delle risorse secondarie.
    
    \item[Best-of-Breed Service Selection] Selezione del cloud provider ottimale per ogni categoria di servizio basata su capacità tecniche specifiche. Massimizza l'ottimizzazione tecnica ma introduce complessità operativa significativa.
    
    \item[Hybrid Edge Distribution] Combinazione di cloud pubblici per workload scalabili e edge computing locale per applicazioni latency-sensitive. Pattern ottimale per la \gls{gdo} che bilancia performance e resilienza.
\end{description}

La modellazione matematica della distribuzione ottimale utilizza algoritmi di ottimizzazione che considerano múltiple obiettivi:

\begin{align}
\text{Minimizza:} \quad & \sum_{i} C_i \times X_i + \sum_{j} L_j \times Y_j + \sum_{k} R_k \times Z_k \label{eq:multi-cloud-optimization} \\
\text{Soggetto a:} \quad & \text{Vincoli di capacità per ogni cloud provider} \nonumber \\
& \text{Requisiti di latenza per applicazioni critiche} \nonumber \\
& \text{Compliance e data sovereignty requirements} \nonumber \\
& \text{Budget constraints e costi operativi} \nonumber
\end{align}

Dove $C_i$, $L_j$, $R_k$ rappresentano rispettivamente costi computazionali, di latenza, e di rischio, mentre $X_i$, $Y_j$, $Z_k$ sono variabili di decisione per l'allocazione delle risorse.

\subsubsection{Orchestrazione e Management Layer}

L'implementazione di un management layer unificato rappresenta la chiave per il successo di strategie multi-cloud. Il layer di orchestrazione deve astrarre le specificità dei singoli provider fornendo interfacce unificate per deployment, monitoring, e gestione del ciclo di vita delle applicazioni.

L'architettura del management layer si basa su principi di API-first design e utilizza pattern di orchestrazione che includono:

\begin{description}
    \item[Infrastructure as Code (\gls{iac})] Definizione dichiarativa dell'infrastruttura attraverso template standardizzati (Terraform, CloudFormation) che permettono deployment consistenti attraverso múltiple provider.
    
    \item[Container Orchestration] Utilizzo di Kubernetes come layer di astrazione che permette portabilità delle applicazioni tra cloud provider diversi.
    
    \item[Service Mesh] Implementazione di istio o linkerd per gestione unified del traffico, sicurezza, e observability in ambienti multi-cloud.
    
    \item[Policy as Code] Definizione di policy di sicurezza, compliance, e governance attraverso codice versionato che garantisce applicazione consistente.
\end{description}

\begin{algorithm}[H]
\caption{Multi-Cloud Management Layer}
\label{alg:multi-cloud-management}
\begin{algorithmic}[1]
\State \textbf{management\_controller:}
\State $\text{provider\_adapters} \leftarrow \{\text{AWS\_adapter}, \text{Azure\_adapter}, \text{GCP\_adapter}\}$
\State $\text{resource\_templates} \leftarrow \text{carica\_template\_IaC}()$
\State $\text{policy\_engine} \leftarrow \text{inizializza\_policy\_enforcement}()$

\Function{deploy\_application}{app\_spec, deployment\_requirements}
    \State $\text{provider\_scores} \leftarrow \{\}$
    
    \ForAll{provider in provider\_adapters}
        \State $\text{score} \leftarrow \text{calcola\_score\_provider}($
        \State \hspace{2em} $\text{app\_spec.risorse\_richieste},$
        \State \hspace{2em} $\text{deployment\_requirements.latenza\_max},$
        \State \hspace{2em} $\text{deployment\_requirements.budget\_max},$
        \State \hspace{2em} $\text{provider.pricing\_current},$
        \State \hspace{2em} $\text{provider.availability\_zones},$
        \State \hspace{2em} $\text{provider.compliance\_certifications})$
        \State $\text{provider\_scores}[\text{provider}] \leftarrow \text{score}$
    \EndFor
    
    \State $\text{provider\_ottimale} \leftarrow \arg\max(\text{provider\_scores})$
    
    \State $\text{template\_deployment} \leftarrow \text{genera\_template\_specifico}(\text{app\_spec}, \text{provider\_ottimale.api\_syntax})$
    
    \State $\text{validation\_result} \leftarrow \text{policy\_engine.valida\_deployment}(\text{template\_deployment}, \text{security\_policies}, \text{compliance\_requirements})$
    
    \If{validation\_result.approved}
        \State $\text{deployment\_id} \leftarrow \text{provider\_ottimale.deploy}(\text{template\_deployment})$
        \State $\text{registra\_deployment}(\text{deployment\_id}, \text{provider\_ottimale}, \text{timestamp})$
        \State \Return deployment\_id
    \Else
        \State \Return validation\_result.errori
    \EndIf
\EndFunction
\end{algorithmic}
\end{algorithm}

\subsubsection{Data Management e Consistency in Ambienti Multi-Cloud}

La gestione dei dati in architetture multi-cloud presenta sfide uniche legate alla consistenza, alla latenza di sincronizzazione, e alla compliance normativa. L'implementazione richiede strategie sofisticate di data partitioning, replication, e conflict resolution.

\paragraph{Data Partitioning Strategies}

La strategia di partitioning ottimale dipende dai pattern di accesso e dai requisiti di business:

\begin{itemize}
    \item \textbf{Geographic Partitioning}: Dati partizionati per regione geografica per minimizzare latenza e rispettare data sovereignty
    \item \textbf{Functional Partitioning}: Separazione basata su funzioni business (inventory vs analytics vs customer data)
    \item \textbf{Temporal Partitioning}: Dati storici vs operativi con strategie di storage differenziate
\end{itemize}

\paragraph{Replication and Synchronization}

L'implementazione di meccanismi di replica cross-cloud utilizza pattern di \textbf{eventual consistency} con conflict resolution automatico:

\begin{algorithm}[H]
\caption{Sincronizzazione Multi-Cloud}
\label{alg:sincronizzazione-multi-cloud}
\begin{algorithmic}[1]
\State \textbf{Parametri:} window\_sincronizzazione = 5\_minuti, soglia\_conflitti = 1\%, priority\_resolution = [timestamp, source\_authority, business\_rules]

\Function{sincronizza\_dataset}{dataset\_id}
    \State $\text{versioni\_locali} \leftarrow \text{raccogli\_versioni\_da\_tutti\_cloud}()$
    
    \State $\text{conflitti} \leftarrow \text{identifica\_record\_divergenti}(\text{versioni\_locali})$
    
    \If{$|\text{conflitti}| / |\text{dataset}| > \text{soglia\_conflitti}$}
        \State $\text{escalation\_manuale}(\text{conflitti}, \text{dataset\_id})$
        \State \Return SYNC\_FAILED
    \EndIf
    
    \ForAll{conflitto in conflitti}
        \State $\text{versione\_autoritativa} \leftarrow \text{risolvi\_conflitto}(\text{conflitto.versioni}, \text{priority\_resolution}, \text{business\_context})$
        \State $\text{propaga\_versione\_autoritativa}(\text{versione\_autoritativa}, \text{tutti\_cloud})$
    \EndFor
    
    \If{verifica\_consistency\_check()}
        \State \Return SYNC\_SUCCESS
    \Else
        \State $\text{rollback\_sincronizzazione}()$
        \State \Return SYNC\_FAILED
    \EndIf
\EndFunction
\end{algorithmic}
\end{algorithm}

\subsection{Performance Optimization: Latenza Critica e Throughput}
\label{subsec:performance-optimization}

L'ottimizzazione delle prestazioni in architetture cloud per la \gls{gdo} richiede un approccio olistico che consideri l'intera stack tecnologica, dai protocolli di rete alle architetture applicative, bilanciando latenza, throughput, e costi operativi.

\subsubsection{Modellazione delle Performance Requirements}

I requisiti di performance per applicazioni \gls{gdo} presentano caratteristiche specifiche che derivano dalla natura real-time delle operazioni commerciali. L'analisi quantitativa dei \gls{sla} operativi basata su best practice di settore rivela pattern distintivi:

\begin{description}
    \item[Transazioni POS] Latenza < 200ms end-to-end, disponibilità 99.95\%
    \item[Inventory Queries] Latenza < 500ms, throughput > 1000 query/sec per store
    \item[Analytics Batch] Throughput > 10GB/hour, window batch < 4 ore
    \item[Customer Experience] Latenza web < 2sec, mobile < 1.5sec
\end{description}

La modellazione matematica delle performance utilizza teoria delle code per prevedere comportamento sistemico sotto carico variabile:

\begin{equation}
\text{Latenza\_media} = \frac{1}{\mu - \lambda} \times \left(1 + \frac{\rho^2}{2(1-\rho)}\right)
\label{eq:latenza-media-queue}
\end{equation}

Dove:
\begin{description}
    \item[$\mu$] tasso di servizio del sistema
    \item[$\lambda$] tasso di arrivo delle richieste
    \item[$\rho = \lambda/\mu$] utilization factor
\end{description}

Per sistemi \gls{gdo} con pattern di carico stagionale, l'analisi utilizza modelli \textbf{M/M/c/K} (multiple server con capacità finita) che meglio rappresentano la realtà operativa.

\subsubsection{Strategie di Caching Distribuito}

L'implementazione di strategie di caching distribuito rappresenta uno degli approcci più efficaci per l'ottimizzazione delle prestazioni in architetture cloud per la \gls{gdo}. La progettazione deve considerare la natura geograficamente distribuita delle operazioni e la variabilità dei pattern di accesso.

\paragraph{Cache Hierarchy Design}

L'architettura di caching implementa una gerarchia multi-livello ottimizzata per pattern di accesso retail:

\begin{itemize}
    \item \textbf{L1 - Browser/Mobile Cache}: 1-5 minuti TTL per contenuti dinamici
    \item \textbf{L2 - CDN Edge}: 5-60 minuti TTL per contenuti semi-statici
    \item \textbf{L3 - Application Cache}: 1-24 ore TTL per dati computazionalmente intensivi
    \item \textbf{L4 - Database Cache}: Query result caching con invalidation intelligente
\end{itemize}

La strategia di cache warming utilizza \gls{ml} per predire pattern di accesso e pre-popolare cache durante orari di basso carico:

\begin{algorithm}[H]
\caption{Predictive Cache Warming}
\label{alg:predictive-cache-warming}
\begin{algorithmic}[1]
\State \textbf{Modello:} Random\_Forest\_Regressor per predizione accessi

\Function{esegui\_cache\_warming\_notturno}{}
    \State $\text{timestamp\_corrente} \leftarrow \text{ora\_attuale}()$
    \State $\text{finestra\_predizione} \leftarrow \text{timestamp\_corrente} + 8\_\text{ore}$
    
    \State $\text{features\_contextuali} \leftarrow \{$
    \State \hspace{2em} $\text{giorno\_settimana: get\_day\_of\_week}(),$
    \State \hspace{2em} $\text{mese: get\_month}(),$
    \State \hspace{2em} $\text{stagione: get\_season}(),$
    \State \hspace{2em} $\text{eventi\_speciali: check\_special\_events}(),$
    \State \hspace{2em} $\text{meteo\_previsto: get\_weather\_forecast}(),$
    \State \hspace{2em} $\text{promozioni\_attive: get\_active\_promotions}()\}$
    
    \ForAll{store in store\_list}
        \State $\text{store\_features} \leftarrow \text{features\_contextuali} + \text{get\_store\_specific\_features}(\text{store})$
        
        \State $\text{predicted\_hot\_items} \leftarrow \text{ml\_model.predict}(\text{store\_features}, \text{finestra\_predizione})$
        
        \ForAll{item in predicted\_hot\_items}
            \If{item.confidence\_score > 0.7}
                \State $\text{pre\_load\_to\_cache}(\text{item.data}, \text{store.cache\_layer})$
                
                \State $\text{stores\_nearby} \leftarrow \text{find\_geographic\_neighbors}(\text{store}, \text{radius}=50\text{km})$
                
                \ForAll{nearby\_store in stores\_nearby}
                    \If{similar\_demographics(store, nearby\_store)}
                        \State $\text{pre\_load\_to\_cache}(\text{item.data}, \text{nearby\_store.cache\_layer})$
                    \EndIf
                \EndFor
            \EndIf
        \EndFor
    \EndFor
\EndFunction
\end{algorithmic}
\end{algorithm}

\subsubsection{Database Performance Optimization}

L'ottimizzazione delle prestazioni database in architetture cloud per la \gls{gdo} richiede strategie specifiche che considerino la natura delle query retail e i pattern di accesso distribuiti.

\paragraph{Read Replica Optimization}

L'implementazione di read replica geograficamente distribuite riduce la latenza delle query read-heavy tipiche del retail:

\begin{itemize}
    \item \textbf{Inventory Queries}: Replica locale per ogni regione (latenza < 50ms)
    \item \textbf{Product Catalog}: CDN-based caching con refresh incrementale
    \item \textbf{Customer Data}: Replica basata su data residency requirements
\end{itemize}

\paragraph{Query Optimization Strategies}

L'analisi dei pattern di query retail rivela ottimizzazioni specifiche:

\begin{itemize}
    \item \textbf{Spatial Indexing}: Per query geografiche (store locator, delivery zones)
    \item \textbf{Composite Indexing}: Per query multi-dimensionali (product + price + availability)
    \item \textbf{Partitioning}: Temporal partitioning per dati transazionali storici
\end{itemize}

L'implementazione di \textbf{adaptive query optimization} utilizza statistics real-time per adattare piani di esecuzione:

\begin{lstlisting}[language=SQL, caption=Query ottimizzata per inventory lookup, label=lst:inventory-query]
-- Esempio di query ottimizzata per inventory lookup
WITH store_inventory AS (
  SELECT 
    product_id, 
    quantity_available,
    last_updated
  FROM inventory 
  WHERE store_id = $1 
    AND last_updated > NOW() - INTERVAL '1 hour'
    AND quantity_available > 0
),
product_details AS (
  SELECT 
    p.product_id,
    p.name,
    p.price,
    p.category_id
  FROM products p
  WHERE p.active = true
    AND p.product_id IN (SELECT product_id FROM store_inventory)
)
SELECT 
  pd.product_id,
  pd.name,
  pd.price,
  si.quantity_available,
  si.last_updated
FROM product_details pd
JOIN store_inventory si ON pd.product_id = si.product_id
ORDER BY pd.category_id, pd.name
LIMIT 50;

-- Index supporto ottimizzato
CREATE INDEX CONCURRENTLY idx_inventory_store_updated_qty 
ON inventory (store_id, last_updated, quantity_available) 
WHERE quantity_available > 0;
\end{lstlisting}

\section{Analisi Integrata e Roadmap di Transizione}
\label{sec:analisi-integrata}

\subsection{Framework di Valutazione Architettural Maturity}
\label{subsec:framework-maturity}

Lo sviluppo di un framework strutturato per la valutazione della maturità architettuale rappresenta un contributo metodologico importante per guidare le decisioni di evoluzione infrastrutturale nella \gls{gdo}. Il framework integra valutazioni tecniche, operative, e strategiche attraverso un modello di maturità a cinque livelli che supporta la validazione delle ipotesi di ricerca attraverso benchmark quantitativi.

\subsubsection{Livelli di Maturità Architettuale}

\begin{description}
    \item[Livello 1 - Legacy Foundation] Architetture tradizionali basate su infrastruttura fisica dedicata, con limitata automazione e forte dipendenza da interventi manuali.
    
    \item[Livello 2 - Virtualized Infrastructure] Implementazione di virtualizzazione e primi passi verso consolidamento infrastrutturale, con miglioramenti in utilization rate e flessibilità operativa.
    
    \item[Livello 3 - Hybrid Operations] Integrazione di componenti cloud per workload non critici, implementazione di \gls{sd-wan}, e automazione parziale dei processi operativi.
    
    \item[Livello 4 - Cloud-First Strategy] Adozione prevalente di architetture cloud con edge computing per applicazioni latency-sensitive e automazione avanzata.
    
    \item[Livello 5 - Autonomous Infrastructure] Architetture completamente software-defined con auto-healing, predictive scaling, e \gls{ai}-driven optimization.
\end{description}

Il modello di assessment utilizza una matrice di valutazione quantitativa che fornisce metriche baseline per la validazione empirica delle ipotesi di ricerca:

\begin{algorithm}[H]
\caption{Architectural Maturity Assessment}
\label{alg:maturity-assessment}
\begin{algorithmic}[1]
\State \textbf{Dimensioni di Valutazione:}
\State infrastruttura\_fisica: peso 20\%, connettività\_rete: peso 20\%, platform\_services: peso 20\%
\State automation\_level: peso 15\%, security\_posture: peso 15\%, operational\_efficiency: peso 10\%

\Function{calcola\_maturity\_score}{organizzazione}
    \State $\text{scores} \leftarrow \{\}$
    
    \State $\text{scores.infrastruttura} \leftarrow \text{evalua\_infrastruttura}($
    \State \hspace{2em} $\text{percentuale\_virtualizzazione}, \text{utilizzo\_cloud\_services},$
    \State \hspace{2em} $\text{ridondanza\_implementata}, \text{monitoring\_capabilities})$
    
    \State $\text{scores.connettività} \leftarrow \text{evalua\_connettività}($
    \State \hspace{2em} $\text{implementazione\_sdwan}, \text{bandwidth\_availability},$
    \State \hspace{2em} $\text{latenza\_media\_sites}, \text{resilienza\_collegamenti})$
    
    \State $\text{scores.platform} \leftarrow \text{evalua\_platform}($
    \State \hspace{2em} $\text{container\_adoption}, \text{microservices\_ratio},$
    \State \hspace{2em} $\text{api\_first\_design}, \text{data\_services\_gestiti})$
    
    \State $\text{scores.automation} \leftarrow \text{evalua\_automation}($
    \State \hspace{2em} $\text{infrastructure\_as\_code}, \text{ci\_cd\_maturity},$
    \State \hspace{2em} $\text{incident\_response\_automation}, \text{self\_healing\_capabilities})$
    
    \State $\text{scores.security} \leftarrow \text{evalua\_security}($
    \State \hspace{2em} $\text{zero\_trust\_implementation}, \text{compliance\_automation},$
    \State \hspace{2em} $\text{threat\_detection\_capabilities}, \text{identity\_management\_maturity})$
    
    \State $\text{scores.operations} \leftarrow \text{evalua\_operations}($
    \State \hspace{2em} $\text{mean\_time\_to\_recovery}, \text{operational\_overhead},$
    \State \hspace{2em} $\text{skill\_availability}, \text{process\_standardization})$
    
    \State $\text{maturity\_score} \leftarrow ($
    \State \hspace{2em} $\text{scores.infrastruttura} \times 0.20 +$
    \State \hspace{2em} $\text{scores.connettività} \times 0.20 +$
    \State \hspace{2em} $\text{scores.platform} \times 0.20 +$
    \State \hspace{2em} $\text{scores.automation} \times 0.15 +$
    \State \hspace{2em} $\text{scores.security} \times 0.15 +$
    \State \hspace{2em} $\text{scores.operations} \times 0.10)$
    
    \State $\text{maturity\_level} \leftarrow \text{determina\_livello}(\text{maturity\_score})$
    \State $\text{gap\_analysis} \leftarrow \text{identifica\_gap\_per\_livello\_successivo}(\text{scores})$
    
    \State \Return $\{\text{maturity\_level}, \text{maturity\_score}, \text{gap\_analysis}\}$
\EndFunction
\end{algorithmic}
\end{algorithm}

\subsection{Strategic Roadmap per la Transizione Cloud-First}
\label{subsec:strategic-roadmap}

Lo sviluppo di una roadmap strategica per la transizione verso architetture cloud-first richiede un approccio sistematico che bilanci benefici attesi, rischi operativi, e vincoli economici. La roadmap si articola su un orizzonte temporale di 3-5 anni con milestone intermedie misurabili che permetteranno la validazione empirica delle ipotesi formulate nel Capitolo~\ref{ch:introduzione}.

\subsubsection{Fasi della Roadmap di Transizione}

\paragraph{Fase 1 - Foundation (0-12 mesi): Infrastructure Modernization}

\textbf{Obiettivi}: Consolidamento infrastrutturale e preparazione per cloud adoption
\begin{itemize}
    \item Virtualizzazione completa dell'infrastruttura legacy (target: 90\%)
    \item Implementazione \gls{sd-wan} per tutti i siti (target: 100\% store connesse)
    \item Upgrade sistemi di alimentazione e cooling per efficienza cloud-ready
    \item Training team IT su cloud technologies e automation tools
\end{itemize}

\textbf{Metriche di Successo per Validation Framework}:
\begin{itemize}
    \item Baseline maturity assessment (Livello 1→2)
    \item Riduzione operational overhead: 15-25\%
    \item Miglioramento uptime: 99.5\% → 99.9\%
    \item Riduzione mean time to deployment: 50\%
\end{itemize}

\paragraph{Fase 2 - Hybrid Acceleration (12-24 mesi): Selective Cloud Migration}

\textbf{Obiettivi}: Migrazione selettiva workload non critici e implementazione edge computing
\begin{itemize}
    \item Migrazione sviluppo/test environments su cloud pubblico
    \item Implementazione edge computing per analytics real-time
    \item Automazione deployment e configuration management
    \item Implementazione multi-cloud strategy per disaster recovery
\end{itemize}

\textbf{Metriche di Successo per Hypothesis Validation}:
\begin{itemize}
    \item Maturity progression (Livello 2→3)
    \item 30\% workload su cloud pubblico
    \item Riduzione time-to-market per nuove applicazioni: 60\%
    \item Miglioramento disaster recovery \gls{rto}: 4h → 1h
\end{itemize}

\paragraph{Fase 3 - Cloud-First Operations (24-36 mesi): Core Business Migration}

\textbf{Obiettivi}: Migrazione workload business-critical e ottimizzazione performance
\begin{itemize}
    \item Refactoring applicazioni core per architetture cloud-native
    \item Implementazione container orchestration (Kubernetes)
    \item \gls{ai}/\gls{ml} integration per predictive analytics e automation
    \item Zero Trust security model implementation
\end{itemize}

\textbf{Metriche di Successo per Research Validation}:
\begin{itemize}
    \item Maturity progression (Livello 3→4)
    \item 70\% workload su architetture cloud-first
    \item Validazione Ipotesi H1: miglioramento simultaneo sicurezza+performance
    \item Riduzione operational overhead: 40\%
    \item Miglioramento customer experience metrics: 30\%
\end{itemize}

\paragraph{Fase 4 - Autonomous Infrastructure (36+ mesi): AI-Driven Optimization}

\textbf{Obiettivi}: Architetture completamente autonome con \gls{ai}-driven optimization
\begin{itemize}
    \item Self-healing infrastructure implementation
    \item Predictive scaling basato su \gls{ml} algorithms
    \item Automated incident response e remediation
    \item Sustainable IT practices integration
\end{itemize}

\textbf{Metriche di Successo per Final Validation}:
\begin{itemize}
    \item Achievement maturity Livello 5
    \item 90\%+ automation rate per operazioni routine
    \item Validazione Ipotesi H3: compliance-by-design cost reduction
    \item Riduzione \gls{mttr}: 75\%
    \item Achievement carbon neutrality per IT operations
\end{itemize}

\subsubsection{Investment Planning e ROI Projection}

L'analisi economica della roadmap di transizione utilizza modelli NPV (Net Present Value) che considerano investimenti, savings operativi, e benefici strategici su un orizzonte quinquennale, fornendo dati quantitativi per la validazione dell'Ipotesi H3 sulla compliance-by-design.

\paragraph{Investment Breakdown per Organizzazione GDO Tipo (100 store)}
\begin{itemize}
    \item Infrastructure Modernization: €2-4M
    \item Cloud Migration Services: €1-2M per professional services e training
    \item Software Licensing: €500K-1M annui per cloud services e automation tools
    \item Operational Transition: €300-500K per change management e training
\end{itemize}

\paragraph{Savings Projection per Validation Framework}
\begin{itemize}
    \item Infrastructure OPEX reduction: 30-50\%
    \item Operational efficiency gains: 25-40\%
    \item Improved agility value: 15-25\% revenue impact
    \item Compliance automation savings: potenziale 20-40\% (Hypothesis H3 validation target)
\end{itemize}

\paragraph{ROI Calculation per Research Validation}

\begin{align}
\text{NPV} &= \sum_{t=0}^{5} \frac{\text{CF}_t}{(1 + r)^t} \label{eq:npv-calculation} \\
\text{NPV} &= \frac{-4}{1} + \frac{-1}{1.08} + \frac{0.5}{1.08^2} + \frac{2}{1.08^3} + \frac{3}{1.08^4} + \frac{4}{1.08^5} \nonumber \\
\text{NPV} &\approx €2.1\text{M} \nonumber
\end{align}

Con IRR ≈ 24\% (superiore al WACC, investimento attrattivo) e Payback Period ≈ 2.8 anni.

\subsection{Risk Assessment e Mitigation Strategies}
\label{subsec:risk-assessment}

L'implementazione di una strategia di transizione cloud-first comporta rischi operativi, tecnologici, e strategici che devono essere identificati, quantificati, e mitigati attraverso strategie appropriate. Questa analisi contribuisce alla metodologia di validazione delle ipotesi fornendo framework di risk assessment quantitativi.

\subsubsection{Categorizzazione e Quantificazione dei Rischi per Research Framework}

\paragraph{Rischi Operativi con Impact Quantificato}
\begin{itemize}
    \item Interruzioni durante migrazione (probabilità: 30\%, impatto: €500K-2M per validation baseline)
    \item Skills gap e resistance to change (probabilità: 50\%, impatto: 6-12 mesi delay su roadmap)
    \item Vendor lock-in e dependency (probabilità: 40\%, impatto: 20-30\% costi aggiuntivi long-term)
\end{itemize}

\paragraph{Rischi Tecnologici per Hypothesis Testing}
\begin{itemize}
    \item Performance degradation post-migrazione (probabilità: 25\%, impatto: customer satisfaction metrics)
    \item Security vulnerabilities in nuove architetture (probabilità: 20\%, impatto: compliance breach scenario)
    \item Integration complexity con sistemi legacy (probabilità: 60\%, impatto: budget overrun analysis)
\end{itemize}

\paragraph{Rischi Strategici per Long-term Analysis}
\begin{itemize}
    \item Regulatory changes affecting cloud adoption (probabilità: 15\%, impatto: architecture redesign requirements)
    \item Technology obsolescence (probabilità: 30\%, impatto: reinvestment necessities)
    \item Competitive disadvantage durante transizione (probabilità: 20\%, impatto: market share analysis)
\end{itemize}

\subsubsection{Comprehensive Mitigation Framework per Research Validation}

\begin{table}[h]
\centering
\caption{Risk Mitigation Framework per Research Validation}
\label{tab:risk-mitigation}
\begin{tabular}{|p{3cm}|p{10cm}|}
\hline
\textbf{Categoria Rischio} & \textbf{Strategie di Mitigazione} \\
\hline
Interruzioni Durante Migrazione & 
\begin{minipage}[t]{10cm}
\vspace{2pt}
• Implementazione blue-green deployment patterns \\
• Extensive testing in staging environments \\
• Rollback procedures automatizzate con timing measurement \\
• Communication plan con stakeholder e impact tracking \\
• Insurance coverage per business interruption quantification
\vspace{2pt}
\end{minipage} \\
\hline
Skills Gap Organizzativo & 
\begin{minipage}[t]{10cm}
\vspace{2pt}
• Training programs strutturati con competency measurement \\
• Partnerships con system integrator con knowledge transfer tracking \\
• Gradual knowledge transfer con overlap periods analysis \\
• Incentive retention per key technical staff con retention rate metrics \\
• External consulting con dependency reduction timeline
\vspace{2pt}
\end{minipage} \\
\hline
Vendor Lock-In & 
\begin{minipage}[t]{10cm}
\vspace{2pt}
• Multi-cloud strategy implementation con portability metrics \\
• Adoption di standard aperti con interoperability assessment \\
• Contract negotiation con exit clauses analysis \\
• Regular vendor performance assessment con switching cost calculation \\
• Backup plan per alternative providers con transition timeline
\vspace{2pt}
\end{minipage} \\
\hline
Security Vulnerabilities & 
\begin{minipage}[t]{10cm}
\vspace{2pt}
• Security by design implementation con attack surface reduction metrics \\
• Regular penetration testing con vulnerability trending analysis \\
• Zero Trust security model con access control quantification \\
• Compliance automation con audit trail effectiveness measurement \\
• Incident response plan con \gls{mttr} improvement tracking
\vspace{2pt}
\end{minipage} \\
\hline
\end{tabular}
\end{table}

\section{Collegamento al Framework di Validazione del Capitolo 1}
\label{sec:collegamento-framework}

I dati quantitativi, metriche di performance, e framework metodologici presentati in questo capitolo costituiscono la base empirica per la validazione delle tre ipotesi di ricerca formulate nel Capitolo~\ref{ch:introduzione}:

\paragraph{Per Ipotesi H1 (Cloud-First Efficacy)}
I case study di migrazione, metriche di performance, e assessment di maturità forniscono baseline quantitative per dimostrare miglioramenti simultanei di sicurezza e performance.

\paragraph{Per Ipotesi H2 (Zero Trust Integration)}
I framework di edge computing, \gls{sd-wan} security, e risk mitigation offrono dati per quantificare la riduzione della superficie di attacco del 20\% target.

\paragraph{Per Ipotesi H3 (Compliance-by-Design)}
L'analisi ROI, roadmap economic modeling, e automation frameworks costituiscono la base per validare i risparmi 20-40\% sui costi di compliance.

Nei capitoli successivi, questi framework verranno applicati a case study specifici per condurre l'analisi comparativa quantitativa necessaria alla validazione statistica delle ipotesi attraverso la metodologia \gls{mcdm} definita nel Capitolo~\ref{ch:introduzione}.

\vspace{1cm}

L'evoluzione infrastrutturale dalla distribuzione tradizionale ad architetture cloud-first rappresenta una trasformazione sistemica che richiede approcci ingegneristici rigorosi, pianificazione strategica accurata, e gestione proattiva dei rischi. Il framework metodologico sviluppato in questo capitolo fornisce alle organizzazioni \gls{gdo} strumenti quantitativi per navigare questa transizione massimizzando i benefici mentre si minimizzano disruption operative e rischi strategici.

La convergenza di tecnologie fisiche e digitali, dalla gestione dell'alimentazione ai microservizi cloud-native, evidenzia come l'infrastruttura IT moderna richieda competenze interdisciplinari che spaziano dall'ingegneria elettrica all'architettura software distribuita. Il successo della transizione dipende non solo dalla selezione delle tecnologie appropriate, ma dalla capacità di orchestrare cambiamenti complessi che impattano persone, processi, e tecnologie simultaneamente.

% ==========================================
% BIBLIOGRAFIA SECONDO REGOLE UNIVERSITÀ
% ==========================================
\begin{bibliografiauniv}

% Ordinata alfabeticamente per cognome autore
% Formato: COGNOME NOME, Titolo, luogo, editore, anno

\item \autore{Airoldi G.}, Gli assetti istituzionali d'impresa: inerzia, funzioni e leve, in \autore{Airoldi G.-Forestieri G.} (a cura di), Corporate governance. Analisi e prospettive nel caso italiano, Milano, Etas Libri, 1998.

\item \autore{Fortuna F.}, Corporate Governance, Milano, F.Angeli, 2001.

\item \autore{Knuth Donald E.}, The Art of Computer Programming, volume 1, Boston, Addison-Wesley, 1997.

\item \autore{Zurzolo G.}, Collegio sindacale e internal auditors, in «Quaderni di finanza», n. 14, Consob, 1996.

% Esempio di più opere dello stesso autore (ordine cronologico)
% \item \autore{Rossi M.}, Prima opera, Milano, Editore, 1995.
% \item \autore{Rossi M.}, Seconda opera, Roma, Altro Editore, 2000.

\end{bibliografiauniv}

\end{document}