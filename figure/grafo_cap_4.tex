% =====================================
% GRAFICI E DIAGRAMMI - CAPITOLO 4 COMPLIANCE E GOVERNANCE
% "Compliance Integrata e Governance"
% Compatibile con XeLaTeX
% =====================================

\documentclass[12pt,a4paper]{article}
\usepackage{fontspec}
\usepackage{polyglossia}
\setmainlanguage{italian}
\usepackage{amsmath}
\usepackage{amsfonts}
\usepackage{amssymb}
\usepackage{tikz}
\usepackage{pgfplots}
\usepackage{pgfplotstable}
\usepackage{booktabs}
\usepackage{array}
\usepackage{multirow}
\usepackage{longtable}
\usepackage{xcolor}
\usepackage{geometry}
\usepackage{tabularx}


% Configurazioni PGFPlots
\pgfplotsset{compat=1.18}
\usetikzlibrary{positioning,shapes,arrows,calc,patterns,decorations.pathreplacing,fit,backgrounds,matrix,chains}

% Definizione colori tema
\definecolor{gdoblu}{RGB}{25,118,188}
\definecolor{gdorosso}{RGB}{220,38,58}
\definecolor{gdoverde}{RGB}{0,150,108}
\definecolor{gdoarancio}{RGB}{255,165,0}
\definecolor{gdogrigio}{RGB}{128,128,128}
\definecolor{gdovioletto}{RGB}{128,0,128}

\geometry{margin=2cm}

\begin{document}

% =====================================
% FIGURA 4.1: ARCHITETTURA CYBER-PHYSICAL REFRIGERAZIONE
% =====================================

\begin{figure}[htbp]
\centering
\begin{tikzpicture}[node distance=1.5cm, scale=0.85, transform shape]
% Definizione stili
\tikzset{
    sensor/.style={circle, minimum size=0.8cm, text centered, draw=gdoarancio, fill=gdoarancio!20, thick},
    controller/.style={rectangle, rounded corners, minimum width=2cm, minimum height=1.2cm, text centered, draw=gdoverde, fill=gdoverde!20, thick},
    gateway/.style={rectangle, rounded corners, minimum width=2.5cm, minimum height=1.5cm, text centered, draw=gdoblu, fill=gdoblu!20, thick},
    bms/.style={rectangle, rounded corners, minimum width=3cm, minimum height=2cm, text centered, draw=gdovioletto, fill=gdovioletto!20, thick},
    cloud/.style={ellipse, minimum width=3cm, minimum height=1.8cm, text centered, draw=gdogrigio, fill=gdogrigio!20, thick},
    attack/.style={->, thick, gdorosso, dashed},
    data/.style={<->, thick, gdoblu},
    control/.style={<->, thick, gdoverde}
}

% Layer 1: Sensori fisici
\node[sensor] (temp1) at (0,0) {\footnotesize T1};
\node[sensor] (temp2) at (1.5,0) {\footnotesize T2};
\node[sensor] (humid1) at (3,0) {\footnotesize H1};
\node[sensor] (press1) at (4.5,0) {\footnotesize P1};
\node[sensor] (temp3) at (6,0) {\footnotesize T3};
\node[sensor] (temp4) at (7.5,0) {\footnotesize T4};

% Layer 2: Controller locali (PLC)
\node[controller] (plc1) at (1,2) {
\begin{minipage}{1.8cm}
\centering
\textbf{PLC-1}
\footnotesize Zona A
\end{minipage}
};

\node[controller] (plc2) at (4,2) {
\begin{minipage}{1.8cm}
\centering
\textbf{PLC-2}  
\footnotesize Zona B
\end{minipage}
};

\node[controller] (plc3) at (7,2) {
\begin{minipage}{1.8cm}
\centering
\textbf{PLC-3}
\footnotesize Zona C
\end{minipage}
};

% Layer 3: Gateway IoT
\node[gateway] (gateway) at (4,4.5) {
\begin{minipage}{2.3cm}
\centering
\textbf{IoT Gateway}

• Protocol Translation
• Data Aggregation
• Local Analytics
\end{minipage}
};

% Layer 4: Building Management System
\node[bms] (bms) at (4,7) {
\begin{minipage}{2.8cm}
\centering
\textbf{Building Management System}

• HVAC Control
• Energy Management  
• Alarm Management
• Historical Data
\end{minipage}
};

% Layer 5: Cloud Analytics
\node[cloud] (cloud) at (4,9.5) {
\begin{minipage}{2.8cm}
\centering
\textbf{Cloud Analytics}

• Predictive Maintenance
• Energy Optimization
• Performance Analytics
\end{minipage}
};

% Connessioni normali
\draw[control] (temp1) -- (plc1);
\draw[control] (temp2) -- (plc1);
\draw[control] (humid1) -- (plc2);
\draw[control] (press1) -- (plc2);
\draw[control] (temp3) -- (plc3);
\draw[control] (temp4) -- (plc3);

\draw[data] (plc1) -- (gateway);
\draw[data] (plc2) -- (gateway);
\draw[data] (plc3) -- (gateway);

\draw[data] (gateway) -- (bms);
\draw[data] (bms) -- (cloud);

% Vettori di attacco
\node[rectangle, draw=gdorosso, fill=gdorosso!20, minimum width=1cm] (attacker) at (-2,4.5) {\footnotesize Attacker};

% Attack vectors
\draw[attack] (attacker) to[bend right=20] (temp1) node[midway, left] {\footnotesize 1. IoT Exploit};
\draw[attack] (attacker) to[bend right=10] (plc2) node[midway, left] {\footnotesize 2. Default Creds};
\draw[attack] (attacker) to[bend left=10] (gateway) node[midway, above] {\footnotesize 3. Network Sniffing};
\draw[attack] (attacker) to[bend left=20] (bms) node[midway, above] {\footnotesize 4. Lateral Movement};

% Impact zones
\node[ellipse, draw=gdorosso, fill=gdorosso!10, minimum width=8cm, minimum height=1.5cm] at (4,0.5) {};
\node[ellipse, draw=gdorosso, fill=gdorosso!10, minimum width=6cm, minimum height=1.5cm] at (4,2.5) {};

% Legenda attack vectors
\node at (10,6) {
\begin{minipage}{3cm}
\footnotesize
\textbf{Vettori di Attacco:}

1. Exploit dispositivi IoT
2. Credenziali default
3. Protocolli non cifrati
4. Movimento laterale

\textbf{Zone di Impatto:}
\tikz \fill[gdorosso!10] (0,0) ellipse (0.3cm and 0.1cm); Controllo fisico

\tikz \fill[gdorosso!20] (0,0) ellipse (0.3cm and 0.1cm); Supervisione
\end{minipage}
};

% Protocolli utilizzati
\node at (10,2) {
\begin{minipage}{3cm}
\footnotesize
\textbf{Protocolli OT:}
• Modbus TCP
• BACnet/IP
• MQTT
• Proprietary
\end{minipage}
};

\end{tikzpicture}
\caption{Architettura cyber-physical dei sistemi di refrigerazione: superficie di attacco e vettori di compromissione}
\label{fig:cyberphysical_refrigeration_architecture}
\end{figure}

% =====================================
% FIGURA 4.2: TIMELINE ATTACCO CYBER-PHYSICAL
% =====================================

\begin{figure}[htbp]
\centering
\begin{tikzpicture}[scale=1.0]
% Timeline principale
\draw[->] (0,0) -- (15,0) node[right] {\textbf{Giorni}};

% Fasi dell'attacco
\foreach \x in {1,2,...,14} {
  \draw (\x,-0.1) -- (\x,0.1);
  \node[below] at (\x,-0.3) {\footnotesize \x};
}

% Fasi principali
\node[rectangle, rounded corners, fill=gdoarancio!20, draw=gdoarancio, minimum width=2.5cm, minimum height=1cm] 
  at (2,2) {\begin{minipage}{2.2cm}\centering\footnotesize\textbf{Reconnaissance}\\ Shodan scanning\\ Device discovery\end{minipage}};

\node[rectangle, rounded corners, fill=gdoverde!20, draw=gdoverde, minimum width=2.5cm, minimum height=1cm] 
  at (5,2) {\begin{minipage}{2.2cm}\centering\footnotesize\textbf{Initial Access}\\ Default credentials\\ Backdoor install\end{minipage}};

\node[rectangle, rounded corners, fill=gdoblu!20, draw=gdoblu, minimum width=2.5cm, minimum height=1cm] 
  at (8,2) {\begin{minipage}{2.2cm}\centering\footnotesize\textbf{Lateral Movement}\\ Network mapping\\ BMS compromise\end{minipage}};

\node[rectangle, rounded corners, fill=gdovioletto!20, draw=gdovioletto, minimum width=2.5cm, minimum height=1cm] 
  at (11,2) {\begin{minipage}{2.2cm}\centering\footnotesize\textbf{Persistence}\\ Multiple backdoors\\ Stealth techniques\end{minipage}};

\node[rectangle, rounded corners, fill=gdorosso!20, draw=gdorosso, minimum width=2.5cm, minimum height=1cm] 
  at (13.5,2) {\begin{minipage}{2.2cm}\centering\footnotesize\textbf{Attack Execution}\\ Temperature manipulation\\ System disruption\end{minipage}};

% Frecce che collegano timeline a fasi
\draw[->] (2,0.2) -- (2,1.5);
\draw[->] (5,0.2) -- (5,1.5);
\draw[->] (8,0.2) -- (8,1.5);
\draw[->] (11,0.2) -- (11,1.5);
\draw[->] (13.5,0.2) -- (13.5,1.5);

% Eventi specifici
\node at (2,4) {\begin{minipage}{2.5cm}\centering\footnotesize\textbf{Giorno 2:}\\ 847 dispositivi identificati\\ 312 con cred. default\end{minipage}};

\node at (5,4) {\begin{minipage}{2.5cm}\centering\footnotesize\textbf{Giorno 5:}\\ Accesso controller\\ Mapping rete OT\end{minipage}};

\node at (8,4) {\begin{minipage}{2.5cm}\centering\footnotesize\textbf{Giorno 8:}\\ BMS compromesso\\ Zero-day exploit\end{minipage}};

\node at (11,4) {\begin{minipage}{2.5cm}\centering\footnotesize\textbf{Giorno 11:}\\ 156 unità controllate\\ 23 store impattati\end{minipage}};

\node at (13.5,4) {\begin{minipage}{2.5cm}\centering\footnotesize\textbf{Giorno 13-14:}\\ Esecuzione weekend\\ Perdita inventario\end{minipage}};

% Indicatori di severity
\foreach \x/\severity in {2/Low, 5/Medium, 8/High, 11/High, 13.5/Critical} {
  \node[circle, inner sep=1pt] at (\x,0.5) {\footnotesize \severity};
}

% Damage curve
\begin{scope}[yshift=-2cm]
\begin{axis}[
    width=12cm,
    height=3cm,
    xlabel={\footnotesize Giorni},
    ylabel={\footnotesize Damage Potential},
    xmin=0, xmax=14,
    ymin=0, ymax=10,
    grid=major,
    no marks
]
\addplot[gdorosso, thick] coordinates {
    (0,0) (3,1) (5,2) (8,6) (11,8) (13,10)
};
\end{axis}
\end{scope}

\end{tikzpicture}
\caption{Timeline dell'attacco cyber-physical: progressione delle fasi e escalation del rischio}
\label{fig:cyberphysical_attack_timeline}
\end{figure}

% =====================================
% TABELLA 4.1: CONFRONTO FRAMEWORK NORMATIVI
% =====================================
\begin{table}[htbp]
\centering
\caption{Confronto Framework Normativi per la GDO: Requisiti e Sovrapposizioni}
\label{tab:regulatory_frameworks_comparison}
\begin{tabular}{@{}llccp{3cm}@{}}
\toprule
\textbf{Framework} & \textbf{Scope} & \textbf{Applicabilità GDO} & \textbf{Penalità Max} & \textbf{Requisiti Chiave} \\
\midrule
\multirow{2}{*}{PCI-DSS 4.0} & Dati pagamento & Tutte le organizzazioni & \$500K + & MFA universale, Network validation \\
& & che processano carte & revoca certificazione & Customized approach \\
\addlinespace
\multirow{2}{*}{GDPR} & Dati personali & Tutte le organizzazioni & 4\% fatturato & Privacy by design \\
& & UE o clienti UE & o €20M & Data minimization, Consent \\
\addlinespace
\multirow{2}{*}{NIS2} & Infrastrutture & Catene GDO & €10M Essential & Supply chain security \\
& critiche & con fatturato >€10M & €7M Important & Incident reporting 24h \\
\addlinespace
\multirow{2}{*}{ISO 27001} & Sistemi informativi & Volontario & N/A & Risk management, ISMS implementation \\
& & (spesso richiesto) & (certificazione) & ISMS implementation \\
\bottomrule
\end{tabular}
\end{table}

% =====================================
% FIGURA 4.3: ANALISI IMPATTI MULTIDIMENSIONALI
% =====================================

\begin{figure}[htbp]
\centering
\begin{tikzpicture}
% Grafico a torta per distribuzione impatti
\begin{scope}[xshift=-6cm]
\node at (0,3) {\textbf{Distribuzione Impatti Totali}};

% Dati: Diretti 4.6%, Reputazionali 68.3%, Compliance 21.6%, Competitivi 5.5%
\def\slices{
  {"Diretti", 4.6, gdoverde},
  {"Reputazionali", 68.3, gdorosso},
  {"Compliance", 21.6, gdoarancio},
  {"Competitivi", 5.5, gdoblu}
}

\def\radius{2.5}
\def\startangle{0}

\foreach \name/\percentage/\color [count=\i] in \slices {
  % Calcolo angoli
  \pgfmathsetmacro{\endangle}{\startangle + \percentage * 3.6}
  
  % Disegno slice
  \fill[\color!70] (0,0) -- (\startangle:\radius) arc (\startangle:\endangle:\radius) -- cycle;
  
  % Etichetta
  \pgfmathsetmacro{\midangle}{(\startangle + \endangle) / 2}
  \node at (\midangle:\radius*0.7) {\footnotesize \percentage\%};
  
  % Legenda
  \node[rectangle, fill=\color!70, minimum width=0.3cm, minimum height=0.3cm] at (4, 2-\i*0.5) {};
  \node[right] at (4.2, 2-\i*0.5) {\footnotesize \name: €\pgfmathprintnumber[precision=1]{\percentage*0.473}M};
  
  \pgfmathsetmacro{\startangle}{\endangle}
}

\node at (0,-3.5) {\footnotesize\textbf{Totale: €47.3M su 3 anni}};
\end{scope}

% Breakdown temporale impatti
\begin{scope}[xshift=6cm]
\begin{axis}[
    title={\textbf{Evoluzione Temporale Impatti}},
    xlabel={Anni},
    ylabel={Impatto Cumulativo (M€)},
    width=8cm,
    height=6cm,
    xmin=0,
    xmax=3,
    ymin=0,
    ymax=50,
    legend style={at={(0.02,0.98)}, anchor=north west},
    stack plots=y
]

% Impatti diretti (immediati)
\addplot[fill=gdoverde!70, draw=gdoverde] coordinates {
    (0,2.2) (0.5,2.2) (1,2.2) (2,2.2) (3,2.2)
};

% Impatti compliance (0-1 anno)
\addplot[fill=gdoarancio!70, draw=gdoarancio] coordinates {
    (0,0) (0.5,10.2) (1,10.2) (2,10.2) (3,10.2)
};

% Impatti reputazionali (progressivi)
\addplot[fill=gdorosso!70, draw=gdorosso] coordinates {
    (0,0) (0.5,8) (1,16) (2,24) (3,32.3)
};

% Impatti competitivi (1-3 anni)
\addplot[fill=gdoblu!70, draw=gdoblu] coordinates {
    (0,0) (0.5,0) (1,1) (2,2) (3,2.6)
};

\legend{Diretti, Compliance, Reputazionali, Competitivi}
\end{axis}
\end{scope}

\end{tikzpicture}
\caption{Analisi multidimensionale degli impatti: distribuzione per categoria e evoluzione temporale}
\label{fig:multidimensional_impact_analysis}
\end{figure}

% =====================================
% FIGURA 4.4: FRAMEWORK GOVERNANCE INTEGRATO
% =====================================

\begin{figure}[htbp]
\centering
\begin{tikzpicture}[node distance=2cm, scale=0.9, transform shape]
% Definizione stili
\tikzset{
    governance/.style={rectangle, rounded corners, minimum width=3.5cm, minimum height=2cm, text centered, draw=gdoblu, fill=gdoblu!20, thick},
    process/.style={rectangle, rounded corners, minimum width=2.5cm, minimum height=1.5cm, text centered, draw=gdoverde, fill=gdoverde!20, thick},
    output/.style={rectangle, rounded corners, minimum width=2cm, minimum height=1cm, text centered, draw=gdorosso, fill=gdorosso!20, thick},
    feedback/.style={<->, thick, gdogrigio, dashed}
}

% Layer superiore: Governance strategica
\node[governance] (risk_gov) at (0,6) {
\begin{minipage}{3.2cm}
\centering
\textbf{Risk Governance}

• Risk appetite definition
• KRI monitoring
• Board reporting
\end{minipage}
};

\node[governance] (compliance_gov) at (5,6) {
\begin{minipage}{3.2cm}
\centering
\textbf{Compliance Governance}

• Multi-standard integration
• Audit coordination
• Policy harmonization
\end{minipage}
};

\node[governance] (crisis_gov) at (10,6) {
\begin{minipage}{3.2cm}
\centering
\textbf{Crisis Governance}

• Incident command
• Business continuity
• Communication management
\end{minipage}
};

% Layer intermedio: Processi operativi
\node[process] (risk_assess) at (-1,3) {
\begin{minipage}{2.2cm}
\centering
\textbf{Risk Assessment}

• Quantitative modeling
• Scenario analysis
\end{minipage}
};

\node[process] (monitor) at (2.5,3) {
\begin{minipage}{2.2cm}
\centering
\textbf{Continuous Monitoring}

• KRI tracking
• Threshold management
\end{minipage}
};

\node[process] (compliance_check) at (5,3) {
\begin{minipage}{2.2cm}
\centering
\textbf{Compliance Validation}

• Control testing
• Gap analysis
\end{minipage}
};

\node[process] (incident_mgmt) at (7.5,3) {
\begin{minipage}{2.2cm}
\centering
\textbf{Incident Management}

• Detection \& response
• Forensic analysis
\end{minipage}
};

\node[process] (recovery) at (11,3) {
\begin{minipage}{2.2cm}
\centering
\textbf{Recovery Management}

• Business continuity
• Disaster recovery
\end{minipage}
};

% Layer inferiore: Output e feedback
\node[output] (reports) at (1,0.5) {\begin{minipage}{1.8cm}\centering\footnotesize Risk Reports\end{minipage}};
\node[output] (dashboards) at (3.5,0.5) {\begin{minipage}{1.8cm}\centering\footnotesize KRI Dashboards\end{minipage}};
\node[output] (audit_results) at (6,0.5) {\begin{minipage}{1.8cm}\centering\footnotesize Audit Results\end{minipage}};
\node[output] (lessons) at (8.5,0.5) {\begin{minipage}{1.8cm}\centering\footnotesize Lessons Learned\end{minipage}};

% Connessioni principali
\draw[->] (risk_gov) -- (risk_assess);
\draw[->] (risk_gov) -- (monitor);
\draw[->] (compliance_gov) -- (compliance_check);
\draw[->] (crisis_gov) -- (incident_mgmt);
\draw[->] (crisis_gov) -- (recovery);

\draw[->] (risk_assess) -- (reports);
\draw[->] (monitor) -- (dashboards);
\draw[->] (compliance_check) -- (audit_results);
\draw[->] (incident_mgmt) -- (lessons);

% Feedback loops
\draw[feedback] (reports) to[bend right=20] (risk_gov);
\draw[feedback] (audit_results) to[bend left=20] (compliance_gov);
\draw[feedback] (lessons) to[bend left=30] (crisis_gov);

% Integrazione trasversale
\draw[<->, thick, gdovioletto] (risk_gov) -- (compliance_gov) node[midway, above] {\footnotesize Integration};
\draw[<->, thick, gdovioletto] (compliance_gov) -- (crisis_gov) node[midway, above] {\footnotesize Coordination};

% Key Performance Indicators
\node at (13,4) {
\begin{minipage}{3cm}
\footnotesize
\textbf{KPI Integrati:}

• Risk-adjusted ROI
• Compliance efficiency
• MTTR incidents
• Recovery capability
• Stakeholder confidence
\end{minipage}
};

\end{tikzpicture}
\caption{Framework di governance integrato: coordinamento di risk management, compliance e crisis management}
\label{fig:integrated_governance_framework}
\end{figure}

% =====================================
% TABELLA 4.2: METRICHE RISK MANAGEMENT
% =====================================

\begin{table}[htbp]
\centering
\caption{Key Risk Indicators (KRI) per Architetture Cloud-Ibride GDO}
\label{tab:key_risk_indicators}
\begin{tabular}{@{}lcccc@{}}
\toprule
\textbf{KRI} & \textbf{Soglia Verde} & \textbf{Soglia Gialla} & \textbf{Soglia Rossa} & \textbf{Frequenza} \\
\midrule
Availability Cumulative & < 2 ore/anno & 2-4 ore/anno & > 4 ore/anno & Real-time \\
\addlinespace
Security Incidents High & 0 & 1/anno & > 1/anno & Daily \\
\addlinespace
Vendor Concentration & < 30\% & 30-35\% & > 35\% & Monthly \\
\addlinespace
Compliance Score & > 95\% & 90-95\% & < 90\% & Weekly \\
\addlinespace
Recovery Time Test & < RTO target & RTO + 20\% & > RTO + 50\% & Quarterly \\
\addlinespace
Patch Currency & > 95\% & 90-95\% & < 90\% & Weekly \\
\addlinespace
Cyber Insurance Gap & 0\% & < 10\% & > 10\% & Annual \\
\bottomrule
\end{tabular}
\end{table}

% =====================================
% FIGURA 4.5: SIMULATION RISK MONTE CARLO
% =====================================

\begin{figure}[htbp]
\centering
\begin{tikzpicture}
\begin{axis}[
    title={Simulazione Monte Carlo: Distribuzione Annual Risk Exposure},
    xlabel={Annual Risk Exposure (M€)},
    ylabel={Probabilità Densità},
    width=12cm,
    height=7cm,
    xmin=0,
    xmax=25,
    ymin=0,
    ymax=0.25,
    grid=major,
    legend style={at={(0.98,0.98)}, anchor=north east},
]

% Distribuzione probability density
\addplot[color=gdoblu, line width=2pt, smooth] coordinates {
    (0.5,0.02) (1,0.05) (2,0.12) (3,0.18) (4,0.22) (5,0.23) (6,0.21) (7,0.18) 
    (8,0.15) (9,0.12) (10,0.09) (12,0.06) (15,0.03) (20,0.01) (25,0.005)
};

% Percentili di rischio
\draw[dashed, gdoverde, line width=2pt] (axis cs:3.2,0) -- (axis cs:3.2,0.25) node[above] {\footnotesize 50\% (€3.2M)};
\draw[dashed, gdoarancio, line width=2pt] (axis cs:8.1,0) -- (axis cs:8.1,0.25) node[above] {\footnotesize 95\% (€8.1M)};
\draw[dashed, gdorosso, line width=2pt] (axis cs:12.3,0) -- (axis cs:12.3,0.25) node[above] {\footnotesize 99\% (€12.3M)};

% Area sotto la curva per VaR 95%
\addplot[fill=gdorosso!20, draw=none, domain=8.1:25] {0.23*exp(-0.3*x)};

\legend{Risk Distribution, VaR 95\% Tail}

\end{axis}

% Box statistiche
\node at (10,-1) {
\begin{minipage}{10cm}
\footnotesize
\begin{tabular}{@{}p{2cm}p{2cm}p{2cm}p{2cm}@{}}
\textbf{Statistica} & \textbf{Valore} & \textbf{Statistica} & \textbf{Valore} \\
\midrule
Media & €4.7M & Expected Shortfall (95\%) & €11.2M \\
Mediana & €3.2M & Value at Risk (99\%) & €12.3M \\
Std Deviation & €3.1M & Maximum Loss & €24.8M \\
\end{tabular}
\end{minipage}
};

\end{tikzpicture}
\caption{Distribuzione probabilistica dell'esposizione al rischio annuale: risultati simulazione Monte Carlo}
\label{fig:monte_carlo_risk_simulation}
\end{figure}

% =====================================
% TABELLA 4.3: BUSINESS CONTINUITY TARGETS
% =====================================

\begin{table}[htbp]
\centering
\caption{Target di Business Continuity per Sistemi Critici GDO}
\label{tab:business_continuity_targets}
\begin{tabular}{@{}lcccc@{}}
\toprule
\textbf{Sistema} & \textbf{RTO Target} & \textbf{RPO Target} & \textbf{Criticità Business} & \textbf{Strategia Recovery} \\
\midrule
Core POS Systems & < 5 min & < 1 min & Critica & Hot standby \\
\addlinespace
Payment Gateway & < 2 min & < 30 sec & Critica & Active-active \\
\addlinespace
Inventory Management & < 30 min & < 15 min & Alta & Warm standby \\
\addlinespace
Customer Database & < 1 ora & < 5 min & Alta & Sync replication \\
\addlinespace
Analytics Platform & < 4 ore & < 1 ora & Media & Async replication \\
\addlinespace
ERP Systems & < 2 ore & < 30 min & Alta & Cluster failover \\
\addlinespace
E-commerce Platform & < 15 min & < 5 min & Alta & Multi-region \\
\addlinespace
Development/Test & < 24 ore & < 8 ore & Bassa & Backup restore \\
\bottomrule
\end{tabular}
\end{table}

\end{document}