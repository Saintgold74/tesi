% =====================================
% GRAFICI E TABELLE PER TESI GDO
% Sezioni 2.1, 2.2, 2.3
% Compatibile con XeLaTeX
% =====================================

\documentclass[12pt,a4paper]{article}
\usepackage{fontspec}
\usepackage{polyglossia}
\setmainlanguage{italian}
\usepackage{amsmath}
\usepackage{amsfonts}
\usepackage{amssymb}
\usepackage{tikz}
\usepackage{pgfplots}
\usepackage{pgfplotstable}
\usepackage{booktabs}
\usepackage{array}
\usepackage{multirow}
\usepackage{longtable}
\usepackage{xcolor}
\usepackage{geometry}

% Configurazioni PGFPlots
\pgfplotsset{compat=1.18}
\usetikzlibrary{positioning,shapes,arrows,calc,patterns,decorations.pathreplacing}

% Definizione colori tema
\definecolor{gdoblu}{RGB}{25,118,188}
\definecolor{gdorosso}{RGB}{220,38,58}
\definecolor{gdoverde}{RGB}{0,150,108}
\definecolor{gdoarancio}{RGB}{255,165,0}
\definecolor{gdogrigio}{RGB}{128,128,128}

\geometry{margin=2cm}

\begin{document}

% =====================================
% SEZIONE 2.1 - MINACCE E RISCHI
% =====================================

\section*{Sezione 2.1 - Minacce e Rischi Principali}

% Figura 2.1: Evoluzione del Threat Landscape GDO
\begin{figure}[htbp]
\centering
\begin{tikzpicture}
\begin{axis}[
    title={Evoluzione del Threat Landscape GDO - Q1 2024 vs Q1 2025},
    xlabel={Tipologia di Attacco},
    ylabel={Incremento Percentuale (\%)},
    ybar,
    bar width=0.6cm,
    width=14cm,
    height=8cm,
    symbolic x coords={Ransomware, Supply Chain, Data Breach, Phishing, Malware POS},
    xtick=data,
    x tick label style={rotate=45, anchor=east},
    ymin=0,
    ymax=160,
    legend style={at={(0.5,-0.15)}, anchor=north, legend columns=1},
    grid=major,
    nodes near coords,
    nodes near coords align={vertical},
]
\addplot[fill=gdorosso!80, draw=gdorosso] coordinates {
    (Ransomware, 149)
    (Supply Chain, 126)
    (Data Breach, 89)
    (Phishing, 73)
    (Malware POS, 45)
};
\legend{Incremento Q1 2025 vs Q1 2024}
\end{axis}
\end{tikzpicture}
\caption{La figura mostra l'incremento percentuale delle diverse tipologie di attacchi nel settore retail, evidenziando la crescita del 149\% per ransomware e del 126\% per attacchi supply chain}
\label{fig:threat_landscape_evolution}
\end{figure}

% Tabella 2.1: Evoluzione Tecniche Attacco POS
\begin{table}[htbp]
\centering
\caption{Evoluzione Tecniche Attacco POS}
\label{tab:pos_attack_evolution}
\begin{tabular}{@{}lcccp{4cm}c@{}}
\toprule
\textbf{Generazione} & \textbf{Periodo} & \textbf{Tasso Successo} & \textbf{Caratteristiche Principali} & \textbf{Contromisure Efficaci} \\
\midrule
Prima & 2019-2021 & 73\% & Malware semplice, vulnerabilità note & Antivirus aggiornati \\
\addlinespace
Seconda & 2022-2023 & 45\% & Offuscamento, comunicazioni cifrate & Analisi comportamentale \\
\addlinespace
Terza & 2024-2025 & 62\% & Adattamento dinamico, manipolazione protocolli & Architetture Zero Trust \\
\bottomrule
\end{tabular}
\end{table}

% Grafico 2.2: Timeline Incidente Applebee's
\begin{figure}[htbp]
\centering
\begin{tikzpicture}
\begin{axis}[
    title={Timeline Incidente Applebee's - Correlazione Tempo vs Impatto},
    xlabel={Giorni dall'Inizio Attacco},
    ylabel={Sistemi Compromessi (\%)},
    width=14cm,
    height=8cm,
    xmin=0,
    xmax=15,
    ymin=0,
    ymax=100,
    grid=major,
    legend style={at={(0.02,0.98)}, anchor=north west},
]
\addplot[color=gdorosso, mark=*, line width=2pt] coordinates {
    (0, 1)
    (2, 1)
    (5, 5)
    (7, 15)
    (9, 45)
    (12, 85)
    (14, 95)
};

% Annotazioni eventi chiave
\draw[<-, thick] (axis cs:2,1) -- (axis cs:3,10) node[above] {\footnotesize Ricognizione};
\draw[<-, thick] (axis cs:5,5) -- (axis cs:6,15) node[above] {\footnotesize Escalation};
\draw[<-, thick] (axis cs:7,15) -- (axis cs:8,25) node[above] {\footnotesize Propagazione};
\draw[<-, thick] (axis cs:14,95) -- (axis cs:13,85) node[left] {\footnotesize Rilevamento};

\legend{Percentuale Sistemi Compromessi}
\end{axis}
\end{tikzpicture}
\caption{Il grafico mostra come l'impatto dell'incidente sia cresciuto esponenzialmente con il tempo, evidenziando l'importanza del rilevamento precoce}
\label{fig:applebees_timeline}
\end{figure}

% Grafico 2.3: Crescita Attacchi Supply Chain
\begin{figure}[htbp]
\centering
\begin{tikzpicture}
\begin{axis}[
    title={Crescita Attacchi Supply Chain 2019-2025},
    xlabel={Anno},
    ylabel={Numero di Attacchi},
    width=14cm,
    height=8cm,
    xmin=2018.5,
    xmax=2025.5,
    ymin=0,
    ymax=800,
    grid=major,
    legend style={at={(0.02,0.98)}, anchor=north west},
    xtick={2019,2020,2021,2022,2023,2024,2025},
]
\addplot[color=gdoblu, mark=square*, line width=2pt] coordinates {
    (2019, 78)
    (2020, 145)
    (2021, 234)
    (2022, 356)
    (2023, 478)
    (2024, 612)
    (2025, 745)
};

% Freccia accelerazione
\draw[<->, thick, gdorosso] (axis cs:2024,612) -- (axis cs:2025,745) node[midway, right] {\footnotesize +21.7\%};

\legend{Attacchi Supply Chain Documentati}
\end{axis}
\end{tikzpicture}
\caption{Il grafico mostra la crescita esponenziale degli attacchi supply chain, con particolare accelerazione nel 2024-2025}
\label{fig:supply_chain_growth}
\end{figure}

% =====================================
% SEZIONE 2.2 - TECNOLOGIE DI DIFESA
% =====================================

\newpage
\section*{Sezione 2.2 - Tecnologie di Difesa Essenziali}

% Figura 2.4: Architettura Difesa Stratificata GDO
\begin{figure}[htbp]
\centering
\begin{tikzpicture}[node distance=1.5cm]
% Definizione stili
\tikzset{
    layer/.style={rectangle, rounded corners, minimum width=12cm, minimum height=1.5cm, text centered, draw=black, fill=blue!20},
    threat/.style={rectangle, rounded corners, minimum width=2cm, minimum height=0.8cm, text centered, draw=red, fill=red!20},
}

% Livelli di difesa
\node[layer, fill=gdorosso!20] (perimeter) at (0,0) {\textbf{Livello Perimetrale} - Firewall NGFW, IDS/IPS, WAF};
\node[layer, fill=gdoarancio!20] (network) at (0,-2.5) {\textbf{Livello Rete} - Segmentazione VLAN, Monitoring, DLP};
\node[layer, fill=gdoverde!20] (endpoint) at (0,-5) {\textbf{Livello Endpoint} - EDR, Antivirus, Device Control};
\node[layer, fill=gdoblu!20] (application) at (0,-7.5) {\textbf{Livello Applicazione} - WAF, API Security, Code Analysis};
\node[layer, fill=gdogrigio!20] (data) at (0,-10) {\textbf{Livello Dati} - Encryption, DLP, Access Control, Backup};

% Minacce esterne
\node[threat] (external) at (-8,0) {Minacce Esterne};
\node[threat] (malware) at (-8,-2.5) {Malware};
\node[threat] (insider) at (-8,-5) {Insider Threats};
\node[threat] (vulns) at (-8,-7.5) {Vulnerabilità App};
\node[threat] (breach) at (-8,-10) {Data Breach};

% Frecce di attacco
\draw[->, thick, red] (external) -- (perimeter);
\draw[->, thick, red] (malware) -- (network);
\draw[->, thick, red] (insider) -- (endpoint);
\draw[->, thick, red] (vulns) -- (application);
\draw[->, thick, red] (breach) -- (data);

% Asset protetti
\node[draw, fill=green!20, minimum width=3cm, minimum height=8cm] (assets) at (8,-5) {
\begin{minipage}{2.5cm}
\centering
\textbf{Asset Critici}

• Sistemi POS
• Database Clienti  
• Dati Pagamento
• Sistemi ERP
• Infrastruttura IT
\end{minipage}
};

\end{tikzpicture}
\caption{I cinque livelli principali di una difesa stratificata tipica per la GDO: perimetrale, rete, endpoint, applicazione, e dati}
\label{fig:layered_defense}
\end{figure}

% Tabella 2.3: Confronto Paradigmi Detection IDS/IPS
\begin{table}[htbp]
\centering
\caption{Confronto Paradigmi Detection IDS/IPS}
\label{tab:detection_paradigms}
\begin{tabular}{@{}lccc@{}}
\toprule
\textbf{Aspetto} & \textbf{Detection Firme} & \textbf{Detection Anomalie} & \textbf{Approccio Ibrido} \\
\midrule
Falsi Positivi & Molto Bassi & Medio-Alti & Bassi \\
Zero-Day Detection & No & Sì & Parziale \\
Overhead Computazionale & Basso & Alto & Medio \\
Facilità Tuning & Alta & Bassa & Media \\
Adattabilità & Bassa & Alta & Alta \\
\bottomrule
\end{tabular}
\end{table}

% Grafico 2.5: Tempi Deployment Patch
\begin{figure}[htbp]
\centering
\begin{tikzpicture}
\begin{axis}[
    title={Tempi Medi di Deployment Patch per Categoria Sistema},
    xlabel={Categoria Sistema},
    ylabel={Tempo Deployment (ore)},
    ybar,
    bar width=0.8cm,
    width=14cm,
    height=8cm,
    symbolic x coords={Sistemi Non Critici, Back Office, Workstation, Server Applicazioni, Sistemi POS},
    xtick=data,
    x tick label style={rotate=45, anchor=east},
    ymin=0,
    ymax=600,
    legend style={at={(0.5,-0.2)}, anchor=north, legend columns=1},
    grid=major,
    nodes near coords,
    nodes near coords align={vertical},
    log basis y=10,
]
\addplot[fill=gdoblu!80, draw=gdoblu] coordinates {
    (Sistemi Non Critici, 4)
    (Back Office, 24)
    (Workstation, 48)
    (Server Applicazioni, 168)
    (Sistemi POS, 504)
};
\legend{Tempo Medio Deployment}
\end{axis}
\end{tikzpicture}
\caption{Il grafico mostra come i tempi di deployment varino significativamente tra categorie di sistemi, da poche ore per sistemi non critici a diverse settimane per sistemi POS critici}
\label{fig:patch_deployment_times}
\end{figure}

% Tabella 2.4: Fattori Prioritizzazione CSPM
\begin{table}[htbp]
\centering
\caption{Fattori di Prioritizzazione Rischio CSPM}
\label{tab:cspm_risk_factors}
\begin{tabular}{@{}llcc@{}}
\toprule
\textbf{Fattore} & \textbf{Descrizione} & \textbf{Peso} & \textbf{Metrica} \\
\midrule
Severità CVSS & Score vulnerabilità standard & 25\% & 0-10 \\
Esposizione Internet & Accessibilità dall'esterno & 20\% & Binario \\
Sensitività Dati & Classificazione dati contenuti & 20\% & 1-5 \\
Criticità Business & Impatto operativo disruption & 15\% & 1-5 \\
Facilità Exploit & Disponibilità exploit pubblici & 10\% & Binario \\
Patch Disponibili & Esistenza di fix & 10\% & Binario \\
\bottomrule
\end{tabular}
\end{table}

% =====================================
% SEZIONE 2.3 - CONFORMITÀ
% =====================================

\newpage
\section*{Sezione 2.3 - Requisiti e Vincoli Architetturali}

% Tabella 2.5: Metriche Performance PCI-DSS
\begin{table}[htbp]
\centering
\caption{Metriche di Performance Monitoraggio PCI-DSS}
\label{tab:pci_performance_metrics}
\begin{tabular}{@{}lcccc@{}}
\toprule
\textbf{Componente} & \textbf{Latenza Aggiunta} & \textbf{CPU Overhead} & \textbf{Storage/Giorno} & \textbf{RAM Richiesta} \\
\midrule
Event Collection & 2-5ms & 3-5\% & 500MB-1GB & 512MB \\
Real-time Analysis & 10-20ms & 8-12\% & 1-2GB & 2GB \\
Correlation Engine & 50-100ms & 15-20\% & 2-3GB & 4GB \\
Audit Storage & N/A & 2-3\% & 2-5GB & 1GB \\
\bottomrule
\end{tabular}
\end{table}

% Figura 2.6: Architettura Privacy by Design
\begin{figure}[htbp]
\centering
\begin{tikzpicture}[node distance=2cm, auto]
% Definizione stili
\tikzset{
    process/.style={rectangle, rounded corners, minimum width=3cm, minimum height=1.5cm, text centered, draw=black, fill=blue!20},
    control/.style={diamond, minimum width=2cm, minimum height=1cm, text centered, draw=red, fill=red!20},
    storage/.style={cylinder, shape border rotate=90, minimum width=2cm, minimum height=1.5cm, text centered, draw=green, fill=green!20},
}

% Fasi del ciclo di vita dei dati
\node[process] (collect) at (0,0) {Raccolta Dati};
\node[control] (classify) at (0,-2.5) {Classificazione Privacy};
\node[process] (process) at (0,-5) {Elaborazione};
\node[control] (purpose) at (3,-5) {Verifica Finalità};
\node[process] (store) at (0,-7.5) {Archiviazione};
\node[storage] (db) at (3,-7.5) {Database};
\node[process] (retention) at (0,-10) {Gestione Retention};
\node[process] (delete) at (0,-12.5) {Cancellazione Sicura};

% Controlli privacy
\node[control, fill=gdoverde!20] (consent) at (-4,-2.5) {Verifica Consenso};
\node[control, fill=gdoverde!20] (minimize) at (-4,-5) {Minimizzazione};
\node[control, fill=gdoverde!20] (encrypt) at (-4,-7.5) {Cifratura};
\node[control, fill=gdoverde!20] (audit) at (-4,-10) {Audit Trail};

% Flusso principale
\draw[->] (collect) -- (classify);
\draw[->] (classify) -- (process);
\draw[->] (process) -- (store);
\draw[->] (store) -- (db);
\draw[->] (store) -- (retention);
\draw[->] (retention) -- (delete);

% Controlli laterali
\draw[->] (consent) -- (classify);
\draw[->] (minimize) -- (process);
\draw[->] (purpose) -- (process);
\draw[->] (encrypt) -- (store);
\draw[->] (audit) -- (retention);

\end{tikzpicture}
\caption{Il flusso di elaborazione dati dalla raccolta alla cancellazione, con controlli privacy integrati in ogni fase}
\label{fig:privacy_by_design}
\end{figure}

% Tabella 2.6: Tempi Risposta NIS2
\begin{table}[htbp]
\centering
\caption{Tempi di Risposta NIS2 per Categoria Incidente}
\label{tab:nis2_response_times}
\begin{tabular}{@{}lccccc@{}}
\toprule
\textbf{Categoria} & \textbf{Severità} & \textbf{Detection Time} & \textbf{Response Time} & \textbf{Recovery Time} & \textbf{Reporting} \\
\midrule
Critico & Alta & < 5 min & < 15 min & < 4 ore & 24 ore \\
Importante & Media & < 15 min & < 1 ora & < 8 ore & 72 ore \\
Standard & Bassa & < 1 ora & < 4 ore & < 24 ore & 7 giorni \\
\bottomrule
\end{tabular}
\end{table}

% Grafico 2.7: Curva Investimento-Disponibilità
\begin{figure}[htbp]
\centering
\begin{tikzpicture}
\begin{axis}[
    title={Curva Investimento-Disponibilità per Architetture NIS2},
    xlabel={Investimento in Resilienza (Unità Relative)},
    ylabel={Disponibilità Ottenibile (\%)},
    width=14cm,
    height=8cm,
    xmin=0,
    xmax=10,
    ymin=90,
    ymax=99.99,
    grid=major,
    legend style={at={(0.7,0.3)}, anchor=north west},
]

% Curva logaritmica
\addplot[color=gdoblu, line width=2pt, smooth] coordinates {
    (0.5, 95)
    (1, 98)
    (2, 99)
    (3, 99.5)
    (4, 99.7)
    (5, 99.8)
    (6, 99.85)
    (7, 99.9)
    (8, 99.93)
    (9, 99.95)
    (9.5, 99.97)
};

% Linea target NIS2
\draw[dashed, gdorosso, line width=2pt] (axis cs:0,99.9) -- (axis cs:10,99.9) node[right] {Target NIS2 (99.9\%)};

% Punto ottimale
\node[circle, fill=gdoverde, inner sep=3pt] at (axis cs:4,99.7) {};
\draw[<-, thick] (axis cs:4,99.7) -- (axis cs:5.5,99.6) node[right] {\footnotesize Punto Ottimale};

\legend{Disponibilità vs Investimento}
\end{axis}
\end{tikzpicture}
\caption{Il grafico mostra come la disponibilità aumenti logaritmicamente con gli investimenti in resilienza, con un punto di ottimizzazione intorno al 99.9\%}
\label{fig:investment_availability}
\end{figure}

% Figura 2.8: Architettura Motore Policy Multi-Standard
\begin{figure}[htbp]
\centering
\begin{tikzpicture}[node distance=2cm]
% Definizione stili
\tikzset{
    standard/.style={rectangle, rounded corners, minimum width=2.5cm, minimum height=1cm, text centered, draw=blue, fill=blue!20},
    engine/.style={rectangle, rounded corners, minimum width=4cm, minimum height=2cm, text centered, draw=black, fill=gray!20},
    control/.style={rectangle, rounded corners, minimum width=3cm, minimum height=1cm, text centered, draw=green, fill=green!20},
}

% Standard di input
\node[standard] (pci) at (-6,3) {PCI-DSS};
\node[standard] (gdpr) at (-6,1) {GDPR};
\node[standard] (nis2) at (-6,-1) {NIS2};
\node[standard] (iso) at (-6,-3) {ISO 27001};

% Motore centrale
\node[engine] (mapper) at (-2,0) {
\begin{minipage}{3.5cm}
\centering
\textbf{Mappatura Cross-Standard}

• Identificazione Overlap
• Risoluzione Conflitti  
• Ottimizzazione Controlli
\end{minipage}
};

\node[engine] (optimizer) at (2,0) {
\begin{minipage}{3.5cm}
\centering
\textbf{Ottimizzatore}

• Set Cover Algorithm
• Minimizzazione Costi
• Validazione Coverage
\end{minipage}
};

% Controlli di output
\node[control] (access) at (6,2) {Access Control};
\node[control] (crypto) at (6,0) {Crittografia};
\node[control] (monitor) at (6,-2) {Monitoring};

% Connessioni
\draw[->] (pci) -- (mapper);
\draw[->] (gdpr) -- (mapper);
\draw[->] (nis2) -- (mapper);
\draw[->] (iso) -- (mapper);

\draw[->] (mapper) -- (optimizer);

\draw[->] (optimizer) -- (access);
\draw[->] (optimizer) -- (crypto);
\draw[->] (optimizer) -- (monitor);

\end{tikzpicture}
\caption{Come i diversi standard vengano integrati in un motore unificato che ottimizza l'implementazione dei controlli}
\label{fig:multi_standard_engine}
\end{figure}

\end{document}