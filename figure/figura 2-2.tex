% ===================================
% FIGURA 2.2 - Curva di Propagazione Malware
% Modello SIR vs Dati Reali
% ===================================
\documentclass[tikz,border=10pt]{standalone}
\usepackage[T1]{fontenc}
\usepackage{pgfplots}
\pgfplotsset{compat=1.17}

\begin{document}
\begin{tikzpicture}
\begin{axis}[
    width=14cm,
    height=10cm,
    xlabel={Tempo (giorni)},
    ylabel={Sistemi Compromessi (\%)},
    xmin=0, xmax=14,
    ymin=0, ymax=100,
    grid=major,
    grid style={line width=.1pt, draw=gray!30},
    legend style={
        at={(0.95,0.95)},
        anchor=north east,
        draw=gray!50,
        fill=white,
        fill opacity=0.9
    },
    title={Curva di propagazione del malware: confronto tra modello SIR e dati reali dell'incidente 2023}
]

% Modello SIR teorico
\addplot[
    domain=0:14,
    samples=100,
    smooth,
    thick,
    color=blue!70,
    line width=1.5pt
] {100/(1 + exp(-0.8*(x-6)))};
\addlegendentry{Modello SIR (R₀ = 2.67)}

% Dati reali
\addplot[
    color=red!70,
    mark=*,
    mark size=2pt,
    line width=1.2pt
] coordinates {
    (0,1)
    (1,2.5)
    (2,6)
    (3,12)
    (4,22)
    (5,35)
    (6,48)
    (7,62)
    (8,73)
    (9,81)
    (10,86)
    (11,89)
    (12,91)
    (13,92)
    (14,92)
};
\addlegendentry{Dati Reali (Incidente 2023)}

% Intervento <24h
\draw[dashed, gray!60, line width=1pt] (axis cs:1,0) -- (axis cs:1,100);
\node[
    fill=yellow!20,
    draw=yellow!60,
    rounded corners=3pt,
    anchor=south west,
    font=\footnotesize
] at (axis cs:1.2,80) {Intervento <24h};

% Area critica
\fill[red!10, opacity=0.3] (axis cs:4,0) rectangle (axis cs:8,100);
\node[
    rotate=90,
    anchor=south,
    font=\footnotesize,
    text=red!60
] at (axis cs:6,50) {Fase Critica};

\end{axis}
\end{tikzpicture}
\end{document}